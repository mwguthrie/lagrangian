\documentclass{article}
%\documentclass[journal]{IEEEtran}
%\documentclass{report}
%\documentclass{acta}

\usepackage{hyperref}
\usepackage{amsmath}

\DeclareMathOperator{\dd}{\!d\!}


\begin{document}

\title{Derivation and Motivation of the Lagrangian Functional of Classical Mechanics}
\author{Gerd Wagner and Matt Guthrie}

\maketitle

\begin{abstract} %This is not quite an abstract in its current form. We should include the fact that the Lagrangian is almost always ``God-given,'' and how our paper remedies this.
The steps taken to arrive at the principle of stationary action, the Euler-Lagrange equations and the Lagrangian of classical mechanics are:
\begin{itemize}
\item Calculation of the minimal distance between to points in a plane to introduce the variation principle and to derive the Euler-Lagrange equation.
\item Proving the Euler-Lagrange equation is independent of arbitrary coordinate transformations and motivating that this independence is desirable for classical mechanics.
\item Straight forward rewrite of $F=ma$ in the form of Euler-Lagrange equations and formulation of the principle of stationary action.
\end{itemize}
\end{abstract}


\section{Introduction}

When the Lagrangian formalism of classical mechanics is introduced, it is often difficult to see the need for a new formalism of classical mechanics. It is even more difficult to understand why the Lagrangian takes the intriguing form $L=T-V$. In this paper, an introductory example from geometry simply requires the formalism. After the formalism is derived, a proof of the independence of the Euler-Lagrange equations under arbitrary coordinate transformations is the central argument to make the formalism desirable for physics. This gives a strong motivation to rewrite $F=ma$ in the form of Euler-Lagrange equations. As a result, the Lagrangian $L=T-V$ appears naturally and does not require further motivation or derivation.

\section{Minimal distance between two points in a plane and the Lagrangian formalism \cite{Klopper}}\label{distance}
The length %arclength?
of a function $y(x)$ between two points $(x_1,y_1=y_1(x_1))$ and $(x_2,y_2=y_2(x_2))$ %this notation seems awkward
is given by 

\begin{equation}
S=\int\limits_{x_1}^{x_2}\sqrt{\dd x^2 + \dd y^2} %which is equivalent to.... we should motivate the appearance of the derivative in this expression
= \int\limits_{x_1}^{x_2}\sqrt{1 + \left(\frac{\dd y}{\dd x}\right)^2} \dd x
= \int\limits_{x_1}^{x_2}\sqrt{1 + y'^2} \dd x.
\end{equation}

We generalize this formula by writing

\begin{equation}
S=\int\limits_{x_1}^{x_2} G(y,y',x) \dd x .
\end{equation}

The generalization is that in our case $G = \sqrt{1 + y'^2}$ only depends on $y'$ and not on $y$ or $x$.

To find the function $y(x)$ that minimizes $S$, we make the simplifying assumption that it is sufficient to find the $y(x)$ that makes $S$ stationary. To do this, we consider small but arbitrary variations $\delta y$ of $y$ and try to find a condition that causes $\delta S$ to vanish. During this, the endpoints $(x_1,y_1=y_1(x_1))$ %again, this notation seems slightly cumbersome to me, though it is not extremely important 
and $(x_2,y_2=y_2(x_2))$ are kept fixed; the variations $\delta y$ have the property $\delta y(x_1) = \delta y(x_2) = 0$. As a result,

\begin{equation}
\delta S = \int\limits_{x_1}^{x_2} \bigg(\frac{\partial G}{\partial y} \delta y 
+ \frac{\partial G}{\partial y'} \delta y' \bigg) \dd x.
\end{equation}

Using $\delta y' = y_2' - y_1' = \frac{\dd }{\dd x}(y_2 - y_1) = \frac{\dd}{\dd x} \delta y$ and integration by parts for the second term, we find

\begin{equation}
\delta S = \int\limits_{x_1}^{x_2} \bigg( \frac{\partial G}{\partial y} \delta y 
- \frac{\dd}{\dd x}\frac{\partial G}{\partial y'} \delta y \bigg) \dd x 
+ \bigg[\frac{\partial G}{\partial y'} \delta y \bigg]_{x_1}^{x_2}.
\end{equation}

The last term vanishes because $\delta y(x_1) = \delta y(x_2) = 0$, and we are left with


\begin{equation}
\delta S = \int\limits_{x_1}^{x_2} \bigg( \frac{\partial G}{\partial y} 
- \frac{\dd}{\dd x}\frac{\partial G}{\partial y'} \bigg) \delta y \; \dd x.
\end{equation}

For $S$ to be stationary, $\delta S$ must vanish. Since $\delta y$ is arbitrary the condition must be


\begin{equation}
\frac{\partial G}{\partial y} - \frac{\dd}{\dd x}\frac{\partial G}{\partial y'} = 0.
\end{equation}

This equation is called \emph{Euler-Lagrange equation}. The procedure of looking for a condition to make $S$ stationary under a function $G(y,y',x)$ is called the Lagrangian formalism.\\

If we substitute $G = \sqrt{1+y'^2}$ into this equation it is straightforward to show that $y'$ must to be constant and thus $y(x)$ is a straight line connecting $(x_1,y_1)$ and $(x_2,y_2)$. %should we show this?

\section{Invariance of the Euler-Lagrange equation under coordinate transformation \cite{Kleinert}}

Let $y=f(Y,x)$ be an invertible and differentiable coordinate transformation. We define the transformed function $\widetilde{G}$ through $G$ by

\begin{equation}
\widetilde{G}(Y,Y',x) := G(f,f',x).
\end{equation}

Using the derivation from section \ref{distance}, we find that making $S = \int\limits_{x_1}^{x_2} \widetilde{G}(Y,Y',x) \dd x$ stationary requires the Euler-Lagrange equation

\begin{equation}
\frac{\partial \widetilde{G}}{\partial Y} 
- \frac{\dd}{\dd x}\frac{\partial \widetilde{G}}{\partial Y'} = 0
\end{equation}

be satisfied. This result is again reached by considering a small but arbitrary variation $\delta Y$ which again vanishes at its endpoints.\\

Likewise, the variation of the same $S$ can be expressed by

\begin{equation}
\delta S = \int\limits_{x_1}^{x_2} \bigg( \frac{\partial G}{\partial f} \delta f 
+ \frac{\partial G}{\partial f'} \delta f' \bigg) \dd x
\end{equation}

where $\delta f$ is given by $\delta f = \frac{\partial f}{\partial Y} \delta Y$. \\


Using $\delta f' = f_2' - f_1' = \frac{\dd}{\dd x}(f_2 - f_1) = \frac{\dd}{\dd x} \delta f$ and integration by parts for the second term we find


\begin{equation}
\delta S = \int\limits_{x_1}^{x_2} \bigg( \frac{\partial G}{\partial f} 
- \frac{\dd}{\dd x} \frac{\partial G}{\partial f'} \bigg) \delta f \; \dd x \;
+ \; \bigg[\frac{\partial G}{\partial f'} \delta f \bigg]_{x_1}^{x_2}.
\end{equation}


As $\delta Y$ vanishes at the endpoints, so does $\delta f$ which causes the last term to vanish. From the arbitrariness of $\delta Y$ follows the arbitrariness of $\delta f$. The only mechanism for $\delta S$ to vanish is

\begin{equation}
0 = \frac{\partial G}{\partial f} - \frac{\dd}{\dd x} \frac{\partial G}{\partial f'} 
= \frac{\partial G}{\partial y} - \frac{\dd}{\dd x} \frac{\partial G}{\partial y'}.
\end{equation}

This proves that the Euler-Lagrange equation takes the same form under any coordinate transformation as long as the transformation of $G$ is given by $\widetilde{G}(Y,Y',x) := G(f,f',x)$. In physics functions that behave under coordinate transformations like $G$ are called scalar functions. %This is too important to be left as a note!

\section{Application to Newtonian mechanics \cite{Guthrie}} %My writeup has not been formally peer-reviewed, we should try to find another source.


It is desirable to write Newton's law $F=ma$ in the form of an Euler-Lagrange equation since this would hold for any general coordinate system. Doing so would also provide us with a functional $G$ that we could easily express in any coordinate system by using the definition of $\widetilde{G}$ above.\\

To do so, we consider Newton's law

\begin{equation}
0 = ma - F. %We should decide on a form for these E-L equations. It was customary in my classes to write P - F = 0. I've noticed you are switching the form around slightly as this document progresses. 
\end{equation}

The first term can be rewritten as follows:
\begin{equation}
ma = m \ddot{r} = \frac{\dd}{\dd t} (m \dot{r}) %The dot notation hasn't been used until now.
= \frac{\dd}{\dd t} \frac{\partial}{\partial \dot{r}} \Big(\frac{1}{2} m \dot{r}^2 \Big) 
= \frac{\dd}{\dd t} \frac{\partial T}{\partial \dot{r}},
\end{equation}

with $T:=\frac{1}{2} m \dot{r}^2$. \\

For the second term, we assume the force $F$ to be conservative. If so, there exists a potential $V$ such that

\begin{equation}
F = - \frac{\partial V}{\partial r} = \frac{\partial (-V)}{\partial r}.
\end{equation}

Using both rewritten terms, Newton's law becomes


\begin{equation}
0 = \frac{\dd}{\dd t} \frac{\partial T}{\partial \dot{r}} - \frac{\partial (-V)}{\partial r}.
\end{equation}


%If we assume that $\frac{\partial T}{\partial r} = 0$ and $\frac{\partial (V)}{\partial \dot{r}} = 0$ which is usually true in Newtonian mechanics we can convert the equation to 

If we assume that $\partial T/ \partial r = 0$ and $\partial V / \partial \dot{r} = 0$, which is nearly always true in Newtonian mechanics, we can convert the equation to 


\begin{equation}
0 = \frac{d}{dt} \frac{\partial (T-V)}{\partial \dot{r}} - \frac{\partial (T-V)}{\partial r}.
\end{equation}

If we look back at our original Euler-Lagrange equation and let time $t$ take the place of a general independent variable $x$, position $r$ take the place of a general coordinate $y$ and $T-V$ take the place of $G$, which from now on we will according to physics conventions call the Lagrangian $L$, we arrive at the following results:

Newton's law takes the form

\begin{equation}
0 = \frac{\dd}{\dd t} \frac{\partial L}{\partial \dot{r}} - \frac{\partial L}{\partial r}, 
\end{equation}

with

\begin{equation}
 L := T-V.
\end{equation}



This equation can be derived by requiring that the trajectory $r(t)$ a particle takes between two endpoints $(t_1,r_1)$ and $(t_2,r_2)$ makes the integral 
\begin{equation}
S=\int\limits_{t_1}^{t_2} L \; \dd t
\end{equation}
stationary. This is called the principle of stationary action in physics. \\

\section{Notes}
\begin{itemize}
\item $L$ has the dimension of energy, thus $S$ has dimensions of energy times time, a quantity called \emph{action}.
\item Since $F=ma$ is formulated in Cartesian coordinates, the Lagrangian $L$ we found is formulated in Cartesian coordinates, as well. The transformation law we found for $G$ provides us with a well defined method for the transformation $L$ to other coordinate systems. The equations of motion which are now Euler-Lagrange equations will likewise be the same in any transformed coordinate system.
\end{itemize}


\begin{thebibliography}{9}

\bibitem{Klopper} Juan Klopper, Understanding the Euler Lagrange Equation, \url{https://www.youtube.com/watch?v=08vJyA-XD3Q}

\bibitem{Kleinert} Hagen Kleinert, Path Integrals, section 1.1 Cassical Mechanics, page 5, \url{http://www.physik.fu-berlin.de/~kleinert/public_html/kleiner_reb5/psfiles/pthic01.pdf}

\bibitem{Guthrie} Matt Guthrie, The Origin of the Lagrangian, \url{https://web2.ph.utexas.edu/~mwguthrie/t.lagrangian.pdf}

\end{thebibliography}{9}



\end{document}
