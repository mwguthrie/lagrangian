\documentclass[prb,preprint]{revtex4-1} 
% The line above defines the type of LaTeX document.
% Note that AJP uses the same style as Phys. Rev. B (prb).

% The % character begins a comment, which continues to the end of the line.

\usepackage{amsmath}  % needed for \tfrac, \bmatrix, etc.
\usepackage{amsfonts} % needed for bold Greek, Fraktur, and blackboard bold
\usepackage{graphicx} % needed for figures

\DeclareMathOperator{\dd}{d\!}
\DeclareMathOperator{\ddd}{\mathrm{d}}

\begin{document}

% Be sure to use the \title, \author, \affiliation, and \abstract macros
% to format your title page.  Don't use lower-level macros to  manually
% adjust the fonts and centering.

\title{Demystifying the Lagrangian of Classical Mechanics} %Derivation and Motivation of the Lagrangian of Classical Mechanics
% In a long title you can use \\ to force a line break at a certain location.

\author{Gerd Wagner}
\email{} % optional
\altaffiliation[permanent address: ]{101 Main Street, 
  Anytown, USA} % optional second address
% If there were a second author at the same address, we would put another 
% \author{} statement here.  Don't combine multiple authors in a single
% \author statement.
\affiliation{mailing address}
% Please provide a full mailing address here.

\author{Matt Guthrie}
\email{matthew.guthrie@ucf.edu}
\affiliation{Department of Physics, University of Central Florida, Orlando, FL 32816}

% See the REVTeX documentation for more examples of author and affiliation lists.

\date{\today}

\begin{abstract} %This is not quite an abstract in its current form. We should include the fact that the Lagrangian is almost always ``God-given,'' and how our paper remedies this.
The Lagrangian formulation of classical mechanics is extremely useful for a vast array of physics problems encountered in the undergraduate and graduate physics curriculum. Unfortunately, many treatments of this topic lack explanations of the most basic details that make Lagrangian mechanics so practical. In this paper, we detail the steps taken to arrive at the principle of stationary action, the Euler-Lagrange equations, and the Lagrangian of classical mechanics. These steps are: 1) the calculation of the minimal distance between two points in a plane, to introduce the variation principle and to derive the Euler-Lagrange equation; 2) proving the Euler-Lagrange equations are independent of arbitrary coordinate transformations and motivating that this independence is desirable for classical mechanics; and 3) a straight-forward reformulation of Newton's second law in the form of Euler-Lagrange equations and formulation of the principle of stationary action. This paper is targeted toward the advanced undergraduate student who, like our own experiences, struggles with topics which are not seen as crucial to the utilization of the tools developed by Lagrangian mechanics, and is especially frustrated by the question ``\textit{why} is $L$ equal to $T$ minus $V$?'' We answer this question in a simple and approachable manner.
\end{abstract}  



\maketitle



\section{Introduction}\label{introduction}

Lagrangian mechanics is a powerful description of classical mechanics, which can be seemingly separate from the more ubiquitous Newtonian formulation. This paper is targeted to the advanced undergraduate student taking a course in Classical Mechanics who has learned about the Lagrangian formulation but has not quite understood how and why it works as an equivalent description of motion. We have observed, through our own experience and through discussions with other students and educators, that significant learning difficulty arises from the presentation of the material when students first encounter Lagrangian mechanics. For many students, Lagrangian mechanics presents an entirely new way to think about physics. Often through a lack of time to cover material, much of the treatment of the subject in university courses and popular textbooks relies on incomplete arguments and God-given principles, especially when considering the Lagrangian ($L$) itself: It is difficult to understand why the Lagrangian takes the intriguing form $L=T-V$, the difference between a particle's kinetic and potential energy (as a function of position $r$, velocity $\dot{r}$, and time $t$). The most popular undergraduate level treatments of the subject simply accept the definition as it is given and move on to using it to solve problems \cite{taylor, marion, fowles, gregory}.

In this paper, the Lagrangian formalism arises naturally through an introductory example from geometry. After the formalism is derived, we prove the invariance of the Euler-Lagrange equation under arbitrary coordinate transformations. To make clear why this property is important for physics we remind the reader that the laws of nature do not depend on the coordinates humans describe them in. On the other hand, physicists cannot formulate laws without coordinates and physical equations usually look different in different coordinates. This is a conflict and it is the reason why the relation of physical theories to coordinates should be as well defined and restricted as possible. The Lagrange formalism fulfills this demand the following way: 

The Euler-Lagrange equation, as we will show, is independent of coordinate transformations. Lagrangian functionals, as we will see, do depend on coordinates but in the simplest way physicists can think of - they transform like scalars. 

Thinking about the relation of the laws of physics to coordinates proved fruitful in the past. In this paper we follow this tradition, making it why it is desirable to reformulate Newton's second law in the form of Euler-Lagrange equations. As a result, the Lagrangian of classical mechanics will appear naturally without requiring further motivation or derivation. 


\section{The Lagrangian formalism and the minimal distance between two points in a plane \cite{Klopper}}\label{distance} %We really cannot cite a youtube video.

We will show that the minimal distance between two points in a plane is a straight line. This proof provides us with a well motivated way to introduce the calculus of variations which is the desired result in this section.\footnote{Some limitations are:
We limit ourselves to lines that can be written as one dimensional functions $y=f(x)$ while curves in two dimensions would be more general. We assume the shortest distance has to be a differentiable function although continuous would be sufficient. First and foremost we satisfy ourselves with finding a condition that makes the curve only stationary instead of minimal.}

The arc length of a function $y(x)$ between two points $(x_1,y_1)$ and $(x_2,y_2)$ is given by

\begin{equation}
S=\int\limits_{x_1}^{x_2}\sqrt{\dd x^2 + \dd y^2}.
\end{equation}

Factoring a $\dd x$ from the radical results in the following equation

\begin{equation}
S= \int\limits_{x_1}^{x_2}\sqrt{1 + \left(\frac{\dd y}{\dd x}\right)^2} \dd x,
\end{equation}

and defining $y' := \frac{\dd y}{\dd x}$ results in a convenient representation for the equation representing the arc length of $y(x)$ as a function of the way in which $y$ changes over its length, %``changes'' is not the best terminology.

\begin{equation}
S = \int\limits_{x_1}^{x_2}\sqrt{1 + y'^2} \dd x.
\end{equation}

We generalize this formula by writing

\begin{equation}
S=\int\limits_{x_1}^{x_2} G(y,y',x) \dd x
\end{equation}



and making the assumption that $G = \sqrt{1 + y'^2}$ only depends on $y'$ and not on $y$ or $x$. We simply give $\sqrt{1+y'^2}$ the new name $G(y')$ which we can replace back at any later stage. We further do as if $G$ would depend on more variables than it actually does. This does no harm to later back replacements of $G$ either. Should formulas we derive for $G$ contain derivatives of $G$ with respect to these new variables, we just replace these terms with zero as the derivative of a function with respect to a variable it does not depend on is always zero. %I'm struggling to find a situation where this would happen (but it is of course mathematically true)

To find the function $y(x)$ that minimizes $S$, we make the simplifying assumption that it is sufficient to find the $y(x)$ that makes $S$ stationary. To do this, we consider small but arbitrary variations $\delta y$ of $y$ and try to find a condition that causes $\delta S$ to vanish. During this, the endpoints $(x_1,y_1)$ and $(x_2,y_2)$ are kept fixed; the variations $\delta y$ have the property $\delta y(x_1) = \delta y(x_2) = 0$. As a result,

\begin{equation}
\delta S = \int\limits_{x_1}^{x_2} \left(\frac{\partial G}{\partial y} \delta y
+ \frac{\partial G}{\partial y'} \delta y' \right) \dd x.
\end{equation}

Using $\delta y' = y_2' - y_1' = \frac{\ddd }{\dd x}(y_2 - y_1) = \frac{\ddd}{\dd x} \delta y$ and integration by parts for the second term, we find

\begin{equation}
\delta S = \int\limits_{x_1}^{x_2} \left( \frac{\partial G}{\partial y} \delta y
- \frac{\ddd}{\dd x}\frac{\partial G}{\partial y'} \delta y \right) \dd x
+ \left[\frac{\partial G}{\partial y'} \delta y \right]_{x_1}^{x_2}.
\end{equation}

The last term vanishes because $\delta y(x_1) = \delta y(x_2) = 0$, and we are left with


\begin{equation}
\delta S = \int\limits_{x_1}^{x_2} \left( \frac{\partial G}{\partial y}
- \frac{\ddd}{\dd x}\frac{\partial G}{\partial y'} \right) \delta y \; \dd x.
\end{equation}

For $S$ to be stationary, $\delta S$ must vanish. Since $\delta y$ is arbitrary the condition must be


\begin{equation}\label{e-l}
\frac{\partial G}{\partial y} - \frac{\ddd}{\dd x}\frac{\partial G}{\partial y'} = 0.
\end{equation}

This equation is called an Euler-Lagrange equation. The procedure of looking for a condition to make $S$ stationary under a function $G(y,y',x)$ is called the Lagrangian formalism. If we substitute $G = \sqrt{1+y'^2}$ into equation \eqref{e-l}, it is straightforward to show that $y'$ must to be constant and thus $y(x)$ is a straight line connecting $(x_1,y_1)$ and $(x_2,y_2)$.

\section{Invariance of the Euler-Lagrange equation under coordinate transformations \cite{Kleinert}}

Let $y=f(Y,x)$ be an invertible and differentiable coordinate transformation. We define the transformed function $\widetilde{G}$ through $G$ by

\begin{equation}
\widetilde{G}(Y,Y',x) := G(f,f',x).
\end{equation}

Using the derivation from section \ref{distance}, we find that making $S = \int\limits_{x_1}^{x_2} \widetilde{G}(Y,Y',x) \dd x$ stationary requires the Euler-Lagrange equation

\begin{equation}
\frac{\partial \widetilde{G}}{\partial Y}
- \frac{\ddd}{\dd x}\frac{\partial \widetilde{G}}{\partial Y'} = 0
\end{equation}

be satisfied. This result is again reached by considering a small but arbitrary variation $\delta Y$ which again vanishes at its endpoints.

Likewise, the variation of the same $S$ can be expressed by

\begin{equation}
\delta S = \int\limits_{x_1}^{x_2} \left( \frac{\partial G}{\partial f} \delta f
+ \frac{\partial G}{\partial f'} \delta f' \right) \dd x
\end{equation}

where $\delta f$ is given by $\delta f = \frac{\partial f}{\partial Y} \delta Y$. 


Using $\delta f' = f_2' - f_1' = \frac{\ddd}{\dd x}(f_2 - f_1) = \frac{\dd}{\dd x} \delta f$ and integration by parts for the second term we find


\begin{equation}
\delta S = \int\limits_{x_1}^{x_2} \left( \frac{\partial G}{\partial f}
- \frac{\ddd}{\dd x} \frac{\partial G}{\partial f'} \right) \delta f \, \dd x \;
+ \; \left[\frac{\partial G}{\partial f'} \delta f \right]_{x_1}^{x_2}.
\end{equation}


As $\delta Y$ vanishes at the endpoints, so does $\delta f$ which causes the last term to vanish. From the arbitrariness of $\delta Y$ follows the arbitrariness of $\delta f$. The only mechanism for $\delta S$ to vanish is

\begin{equation}
0 = \frac{\partial G}{\partial f} - \frac{\ddd}{\dd x} \frac{\partial G}{\partial f'}
= \frac{\partial G}{\partial y} - \frac{\ddd}{\dd x} \frac{\partial G}{\partial y'}.
\end{equation}

This shows that the Euler-Lagrange equation takes the same form under any coordinate transformation as long as the transformation of the function $G$ is given by $\widetilde{G}(Y,Y',x) := G(f,f',x)$. Functions that behave this way under coordinate transformations as $G$ are called scalar functions, especially in physical contexts. 
In the following section $G$ will be interpreted as the Lagrangian we mentioned in section \ref{introduction}. As we now see, $G$ does indeed have the transformation properties we claimed in section \ref{introduction}.

\section{Application to Newtonian mechanics}% \cite{Guthrie}} %My writeup has not been formally peer-reviewed, we should try to find another source. Do we even need a source for this? It's our original work. 

As we discussed in detail in section \ref{introduction}, it is desirable to write Newton's law $F=ma$ in the form of an Euler-Lagrange equation. Doing so will provide us with a function like $G$ from the former section (\ref{e-l}). This function, which we derive, is called the Lagrangian of classical mechanics.

%Can you fix \ref{e-l}? It seems like we should reference a function but it's written like we want to reference a section. 

To proceed, we rearrange Newton's second law

\begin{equation}
0 = ma - F.
\end{equation}

The first term can be rewritten as follows:
\begin{equation}
ma = m \ddot{r} = \frac{\ddd}{\dd t} (m \dot{r})
= \frac{\ddd}{\dd t} \frac{\partial}{\partial \dot{r}} \left(\frac{1}{2} m \dot{r}^2 \right)
= \frac{\ddd}{\dd t} \frac{\partial T}{\partial \dot{r}},
\end{equation}

where $T:=\frac{1}{2} m \dot{r}^2$ is the classical kinetic energy of the system. For the second term, we assume the force $F$ to be conservative. If so, there exists a potential $V$ such that
\begin{equation}
F = - \frac{\partial V}{\partial r} = \frac{\partial (-V)}{\partial r}.
\end{equation}
Using both rewritten terms, Newton's law becomes
\begin{equation}
0 = \frac{\ddd}{\dd t} \frac{\partial T}{\partial \dot{r}} - \frac{\partial (-V)}{\partial r}.
\end{equation}

If we assume that $\partial T/ \partial r = 0$ and $\partial V / \partial \dot{r} = 0$, which is nearly always true in Newtonian mechanics, we can convert the equation to
\begin{equation}
0 = \frac{\ddd}{\dd t} \frac{\partial (T-V)}{\partial \dot{r}} - \frac{\partial (T-V)}{\partial r}.
\end{equation}

\section{Definition of the Lagrangian} %We can easily remove this section header if a reviewer suggests it, though for now I think defining the Lagrangian is important enough to warrant it

Looking back at the original Euler-Lagrange equation \eqref{e-l}, let time $t$ take the place of a general independent variable $x$, position $r$ take the place of a general coordinate $y$, and $T-V$ take the place of $G$, we arrive at the following results:

Newton's second law takes the form
\begin{equation}
0 = \frac{\ddd}{\dd t} \frac{\partial L}{\partial \dot{r}} - \frac{\partial L}{\partial r},
\end{equation}
where
\begin{equation}
 L := T-V.
\end{equation}

This $L$ is precisely the Lagrangian of classical mechanics that so many curricula imply is some divine quantity, while all we have shown is that $L$ is a simple change of variable to make Lagrange's equations more elegant. While elegance is certainly exhibited in this representation, it is often at the expense of increased abstraction.

Using this representation of $L$ allows the Lagrangian formulation of mechanics to be derived by requiring that the trajectory $r(t)$ which a particle takes between two endpoints $(t_1,r_1)$ and $(t_2,r_2)$ makes the integral
\begin{equation}\label{eqref:action}
S=\int\limits_{t_1}^{t_2} L \; \dd t
\end{equation}
stationary. From this, the equations of motion for the particle follows, being the particle's Euler-Lagrange equation. This is called the principle of stationary action in physics\footnote{The principle of stationary action is also sometimes called the principle of \emph{least} action. This can be confusing because the Euler-Lagrange equation finds instances where action is extremized (e.g. a saddle point or even maximized) \cite{action-not-least}.}. 

\section{Conclusion}

It is unnecessary to speciously explain the form of the classical Lagrangian as having an innate property of interest. We have shown that the Lagrangian takes its classical form
\begin{equation}
    L = T-V
\end{equation}
out of convenience.

Since $F=ma$ is formulated in Cartesian coordinates, the Lagrangian $L$ we derived is formulated in Cartesian coordinates, as well. The transformation law we found for $G$ provides us with a well defined method for the transformation of $L$ to other coordinate systems. The equations of motion which are now Euler-Lagrange equations are likewise the same in any coordinate system.  This way the relation of Newton's law to coordinates is restricted and well defined in the sense we mentioned in section \ref{introduction}.

The traditional continuation from equation \eqref{eqref:action} leads to the definition of a quantity called \emph{action}. $L$ has the dimension of energy, thus $S$ has dimensions of energy times time, a quantity also called action.

We hope that this has been useful for making at least one aspect of the Lagrangian formulation of classical mechanics less bewildering for students who are curious about the subject. An in-depth study of classical mechanics can be just as deep and rewarding as an extensive study of statistical or quantum mechanics.  


\begin{thebibliography}{9}

\bibitem{taylor} John R. Taylor, Classical Mechanics, 2005, page 238

\bibitem{marion} Jerry B. Marion, Classical Dynamics, 1965, page 217 %find a more recent copy of this

\bibitem{fowles} Grant R. Fowles, Analytical Mechanics, 1986, page ??? %find a more recent copy of this too

\bibitem{gregory} R. Douglas Gregory, Classical Mechanics, 2006, page 348

%\bibitem{Klopper} Juan Klopper, Understanding the Euler Lagrange Equation, \url{https://www.youtube.com/watch?v=08vJyA-XD3Q}

\bibitem{Kleinert} Hagen Kleinert, Path Integrals in Quantum Mechanics, Statistics, Polymer Physics, and Financial Markets, 2004, page 5, \url{https://books.google.com/books?id=dJ3FCgAAQBAJ}

\bibitem{action-not-least} Edwin Taylor and C.G. Gray, When Action is Not Least, doi: 10.1119/1.2710480

%\bibitem{Guthrie} Matt Guthrie, The Origin of the Lagrangian, \url{https://web2.ph.utexas.edu/~mwguthrie/t.lagrangian.pdf}

\end{thebibliography}



\end{document}
