\documentclass{article}
%\documentclass[journal]{IEEEtran}
%\documentclass{report}
%\documentclass{acta}

\usepackage{hyperref}
\usepackage{amsmath}

\DeclareMathOperator{\dd}{d\!}
\DeclareMathOperator{\ddd}{\mathrm{d}}


\begin{document}

\title{Demystifying the Lagrangian of Classical Mechanics} %Derivation and Motivation of the Lagrangian of Classical Mechanics
\author{Gerd Wagner and Matt Guthrie}

\maketitle

\begin{abstract} %This is not quite an abstract in its current form. We should include the fact that the Lagrangian is almost always ``God-given,'' and how our paper remedies this.
The steps taken to arrive at the principle of stationary action, the Euler-Lagrange equations and the Lagrangian of classical mechanics are:
\begin{itemize}
\item Calculation of the minimal distance between to points in a plane to introduce the variation principle and to derive the Euler-Lagrange equation.
\item Proving the Euler-Lagrange equation is independent of arbitrary coordinate transformations and motivating that this independence is desirable for classical mechanics.
\item Straight forward rewrite of $F=ma$ in the form of Euler-Lagrange equations and formulation of the principle of stationary action.
\end{itemize}
\end{abstract}


\section{Introduction}

Lagrangian mechanics is a powerful description of classical mechanics, which can be seemingly separate from the more ubiquitous Newtonian formulation. This paper is targeted to the advanced undergraduate student taking their second course in Classical Mechanics who has learned about the Lagrangian formulation but has not quite understood how and why it works as an equivalent description of motion. We have observed, through our own experience and through discussions with other students and educators, that significant learning difficulty arises from the presentation of the material when students first encounter Lagrangian mechanics. For many students, Lagrangian mechanics presents an entirely new way to think about physics. Often through a lack of time to cover material, much of the treatment of the subject in university courses and popular textbooks relies on incomplete arguments and God-given principles, especially when considering the Lagrangian ($L$) itself.

When the Lagrangian formalism of classical mechanics is introduced, it is often difficult for students to understand the need for a new formalism of classical mechanics. It is then more difficult to understand why the Lagrangian takes the intriguing form $L=T-V$, the difference between a particle's kinetic and potential energy (as a function of position $r$, velocity $\dot{r}$, and time $t$). In this paper, the Lagrangian formalism arises naturally through an introductory example from geometry. After the formalism is derived, a proof of the independence of the Euler-Lagrange equations under arbitrary coordinate transformations is the central argument to make the formalism desirable for physics. This gives a strong motivation to rewrite Newton's second law in the form of Euler-Lagrange equations. As a result, the Lagrangian functional appears naturally without requiring further motivation or derivation.

\section{The Lagrangian formalism and the minimal distance between two points in a plane \cite{Klopper}}\label{distance}

We will show that the minimal distance between two points in a plane is a straight line. This proof provides us with a well motivated way to introduce the calculus of variations which is the desired result in this section.\footnote{Some limitations are:
We limit ourselves to lines that can be written as one dimensional functions $y=f(x)$ while curves in two dimensions would be more general. We assume the shortest distance has to be a differentiable function although continuous would be sufficient. First and foremost we satisfy ourselves with finding a condition that makes the curve only stationary instead of minimal.}

The arc length of a function $y(x)$ between two points $(x_1,y_1)$ and $(x_2,y_2)$ is given by

\begin{equation}
S=\int\limits_{x_1}^{x_2}\sqrt{\dd x^2 + \dd y^2}.
\end{equation}

Factoring a $\dd x$ from the radical results in the following equation

\begin{equation}
S= \int\limits_{x_1}^{x_2}\sqrt{1 + \left(\frac{\dd y}{\dd x}\right)^2} \dd x,
\end{equation}

and defining $y' := \frac{\dd y}{\dd x}$ results in a convenient representation for the equation representing the arc length of $y(x)$ as a function of the way in which $y$ changes over its length, %``changes'' is not the best terminology.

\begin{equation}
S = \int\limits_{x_1}^{x_2}\sqrt{1 + y'^2} \dd x.
\end{equation}

We generalize this formula by writing

\begin{equation}
S=\int\limits_{x_1}^{x_2} G(y,y',x) \dd x
\end{equation}



and making the assumption that $G = \sqrt{1 + y'^2}$ only depends on $y'$ and not on $y$ or $x$. We simply give $\sqrt{1+y'^2}$ the new name $G(y')$ which we can replace back at any later stage. We further do as if $G$ would depend on more variables than it actually does. This does no harm to later back replacements of $G$ either. Should formulas we derive for $G$ contain derivatives of $G$ with respect to these new variables, we just replace these terms with zero as the derivative of a function with respect to a variable it does not depend on is always zero. %I'm struggling to find a situation where this would happen (but it is of course mathematically true)

To find the function $y(x)$ that minimizes $S$, we make the simplifying assumption that it is sufficient to find the $y(x)$ that makes $S$ stationary. To do this, we consider small but arbitrary variations $\delta y$ of $y$ and try to find a condition that causes $\delta S$ to vanish. During this, the endpoints $(x_1,y_1)$ and $(x_2,y_2)$ are kept fixed; the variations $\delta y$ have the property $\delta y(x_1) = \delta y(x_2) = 0$. As a result,

\begin{equation}
\delta S = \int\limits_{x_1}^{x_2} \left(\frac{\partial G}{\partial y} \delta y
+ \frac{\partial G}{\partial y'} \delta y' \right) \dd x.
\end{equation}

Using $\delta y' = y_2' - y_1' = \frac{\ddd }{\dd x}(y_2 - y_1) = \frac{\ddd}{\dd x} \delta y$ and integration by parts for the second term, we find

\begin{equation}
\delta S = \int\limits_{x_1}^{x_2} \bigg( \frac{\partial G}{\partial y} \delta y
- \frac{\ddd}{\dd x}\frac{\partial G}{\partial y'} \delta y \bigg) \dd x
+ \bigg[\frac{\partial G}{\partial y'} \delta y \bigg]_{x_1}^{x_2}.
\end{equation}

The last term vanishes because $\delta y(x_1) = \delta y(x_2) = 0$, and we are left with


\begin{equation}
\delta S = \int\limits_{x_1}^{x_2} \bigg( \frac{\partial G}{\partial y}
- \frac{\dd}{\dd x}\frac{\partial G}{\partial y'} \bigg) \delta y \; \dd x.
\end{equation}

For $S$ to be stationary, $\delta S$ must vanish. Since $\delta y$ is arbitrary the condition must be


\begin{equation}\label{e-l}
\frac{\partial G}{\partial y} - \frac{\dd}{\dd x}\frac{\partial G}{\partial y'} = 0.
\end{equation}

This equation is called the Euler-Lagrange equation. The procedure of looking for a condition to make $S$ stationary under a function $G(y,y',x)$ is called the Lagrangian formalism.\\

If we substitute $G = \sqrt{1+y'^2}$ into this equation it is straightforward to show that $y'$ must to be constant and thus $y(x)$ is a straight line connecting $(x_1,y_1)$ and $(x_2,y_2)$.

\section{Invariance of the Euler-Lagrange equation under coordinate transformation \cite{Kleinert}}

Let $y=f(Y,x)$ be an invertible and differentiable coordinate transformation. We define the transformed function $\widetilde{G}$ through $G$ by

\begin{equation}
\widetilde{G}(Y,Y',x) := G(f,f',x).
\end{equation}

Using the derivation from section \ref{distance}, we find that making $S = \int\limits_{x_1}^{x_2} \widetilde{G}(Y,Y',x) \dd x$ stationary requires the Euler-Lagrange equation

\begin{equation}
\frac{\partial \widetilde{G}}{\partial Y}
- \frac{\dd}{\dd x}\frac{\partial \widetilde{G}}{\partial Y'} = 0
\end{equation}

be satisfied. This result is again reached by considering a small but arbitrary variation $\delta Y$ which again vanishes at its endpoints.\\

Likewise, the variation of the same $S$ can be expressed by

\begin{equation}
\delta S = \int\limits_{x_1}^{x_2} \bigg( \frac{\partial G}{\partial f} \delta f
+ \frac{\partial G}{\partial f'} \delta f' \bigg) \dd x
\end{equation}

where $\delta f$ is given by $\delta f = \frac{\partial f}{\partial Y} \delta Y$. \\


Using $\delta f' = f_2' - f_1' = \frac{\dd}{\dd x}(f_2 - f_1) = \frac{\dd}{\dd x} \delta f$ and integration by parts for the second term we find


\begin{equation}
\delta S = \int\limits_{x_1}^{x_2} \left( \frac{\partial G}{\partial f}
- \frac{\ddd}{\dd x} \frac{\partial G}{\partial f'} \right) \delta f \, \dd x \;
+ \; \bigg[\frac{\partial G}{\partial f'} \delta f \bigg]_{x_1}^{x_2}.
\end{equation}


As $\delta Y$ vanishes at the endpoints, so does $\delta f$ which causes the last term to vanish. From the arbitrariness of $\delta Y$ follows the arbitrariness of $\delta f$. The only mechanism for $\delta S$ to vanish is

\begin{equation}
0 = \frac{\partial G}{\partial f} - \frac{\ddd}{\dd x} \frac{\partial G}{\partial f'}
= \frac{\partial G}{\partial y} - \frac{\ddd}{\dd x} \frac{\partial G}{\partial y'}.
\end{equation}

This shows that the Euler-Lagrange equation takes the same form under any coordinate transformation as long as the transformation of the function $G$ is given by $\widetilde{G}(Y,Y',x) := G(f,f',x)$. Functions that behave under coordinate transformations in the same way as $G$ are called scalar functions, especially in physical contexts.

\section{Application to Newtonian mechanics \cite{Guthrie}} %My writeup has not been formally peer-reviewed, we should try to find another source.


It is desirable to write Newton's law $F=ma$ in the form of an Euler-Lagrange equation since this would hold for any general coordinate system. Doing so would also provide us with a functional $G$ that we could easily express in any coordinate system by using the definition of $\widetilde{G}$ above.\\

To proceed, we rearrange Newton's law

\begin{equation}
0 = ma - F.
\end{equation}

The first term can be rewritten as follows:
\begin{equation}
ma = m \ddot{r} = \frac{\ddd}{\dd t} (m \dot{r})
= \frac{\ddd}{\dd t} \frac{\partial}{\partial \dot{r}} \left(\frac{1}{2} m \dot{r}^2 \right)
= \frac{\ddd}{\dd t} \frac{\partial T}{\partial \dot{r}},
\end{equation}

where $T:=\frac{1}{2} m \dot{r}^2$ is the familiar kinetic energy of the system. \\

For the second term, we assume the force $F$ to be conservative. If so, there exists a potential $V$ such that

\begin{equation}
F = - \frac{\partial V}{\partial r} = \frac{\partial (-V)}{\partial r}.
\end{equation}

Using both rewritten terms, Newton's law becomes


\begin{equation}
0 = \frac{\ddd}{\dd t} \frac{\partial T}{\partial \dot{r}} - \frac{\partial (-V)}{\partial r}.
\end{equation}


If we assume that $\partial T/ \partial r = 0$ and $\partial V / \partial \dot{r} = 0$, which is nearly always true in Newtonian mechanics, we can convert the equation to


\begin{equation}
0 = \frac{\ddd}{\dd t} \frac{\partial (T-V)}{\partial \dot{r}} - \frac{\partial (T-V)}{\partial r}.
\end{equation}

Looking back at the original Euler-Lagrange equation \eqref{e-l}, let time $t$ take the place of a general independent variable $x$, position $r$ take the place of a general coordinate $y$, and $T-V$ take the place of $G$, we arrive at the following results:

Newton's second law takes the form

\begin{equation}
0 = \frac{\ddd}{\dd t} \frac{\partial L}{\partial \dot{r}} - \frac{\partial L}{\partial r},
\end{equation}

with

\begin{equation}
 L := T-V.
\end{equation}

This $L$ is precisely the Lagrangian of classical mechanics that so many curricula say is some divine quantity. This representation shows that $L$ is a simple change of variable to make Lagrange's equations more elegant. While elegance is certainly exhibited in this representation, it is often at the expense of increased abstraction.

This allows the Lagrangian formulation of mechanics to be derived by requiring that the trajectory $r(t)$ which a particle takes between two endpoints $(t_1,r_1)$ and $(t_2,r_2)$ makes the integral
\begin{equation}
S=\int\limits_{t_1}^{t_2} L \; \dd t
\end{equation}
stationary. From this, the equations of motion for the particle follows, being the particle's Euler-Lagrange equation. This is called the principle of stationary action in physics. \\

\section{Conclusion}

\begin{itemize}
\item $L$ has the dimension of energy, thus $S$ has dimensions of energy times time, a quantity called \emph{action}.
\item Since $F=ma$ is formulated in Cartesian coordinates, the Lagrangian $L$ we found is formulated in Cartesian coordinates, as well. The transformation law we found for $G$ provides us with a well defined method for the transformation of $L$ to other coordinate systems. The equations of motion which are now Euler-Lagrange equations will likewise be the same in any transformed coordinate system.
\end{itemize}

We hope that this has been useful for making the Lagrangian formulation of Classical Mechanics less bewildering for students beginning to study the subject. And if you are a more advanced student, please continue to study the subject as it can be just as deep and rewarding of an extensive study of statistical or quantum mechanics. Physics is cool!


\begin{thebibliography}{9}

\bibitem{Klopper} Juan Klopper, Understanding the Euler Lagrange Equation, \url{https://www.youtube.com/watch?v=08vJyA-XD3Q}

\bibitem{Kleinert} Hagen Kleinert, Path Integrals, section 1.1 Classical Mechanics, page 5, \url{http://www.physik.fu-berlin.de/~kleinert/public_html/kleiner_reb5/psfiles/pthic01.pdf}

\bibitem{Guthrie} Matt Guthrie, The Origin of the Lagrangian, \url{https://web2.ph.utexas.edu/~mwguthrie/t.lagrangian.pdf}

\end{thebibliography}



\end{document}
