\section{Invariance of the Euler-Lagrange equation under coordinate transformations} \label{invariance}

This section was inspired by the first chapter of Hagen (2009)\cite{hagen2009path}.
Let $y=f(Y,x)$ be an invertible and differentiable coordinate transformation.
\footnote{The most effective tool in a physicist's toolbox when solving physics problems is picking an appropriate coordinate system.
A pendulum in Euclidean $(x,y)$ coordinates makes analyzing the problem cumbersome, but in circular $(r,\theta)$ analysis becomes simple.
Translating from the coordinate system of the problem statement to the coordinate system that best simplifies the system usually provides great insight, but you still have to translate back to the coordinate system of the problem statement to solve the problem.}
%This means for any $x$ the transformation $f$ is supposed to be an invertible and differentiable function of the new coordinate $Y$.
%The dependence on $x$ is optional, which is to say that we also well allow transformations $f$ that are only functions of $Y$.
%
We define the transformed function $\widetilde{G}$ through $G$ by
\begin{equation} \label{lagrangian-transform}
\widetilde{G}(Y,Y',x) := G(f,f',x).
\end{equation}
Using the derivation from section \ref{distance}, we find that making $S = \int\limits_{x_1}^{x_2} \widetilde{G}(Y,Y',x) \dd x$ stationary requires the Euler-Lagrange equation

\begin{equation}
\frac{\partial \widetilde{G}}{\partial Y}
- \frac{\ddd}{\dd x}\frac{\partial \widetilde{G}}{\partial Y'} = 0
\end{equation}
be satisfied. This result is again reached by considering a small but arbitrary variation $\delta Y$ which again vanishes at its endpoints.

Likewise, the variation of the same $S$ can be expressed by
\begin{equation}
\delta S = \int\limits_{x_1}^{x_2} \left( \frac{\partial G}{\partial f} \delta f
+ \frac{\partial G}{\partial f'} \delta f' \right) \dd x
\end{equation}
where $\delta f$ is given by $\delta f = \frac{\partial f}{\partial Y} \delta Y$.


Using $\delta f' = f_2' - f_1' = \frac{\ddd}{\dd x}(f_2 - f_1) = \frac{\dd}{\dd x} \delta f$ and integration by parts for the second term we find
\begin{equation}
\delta S = \int\limits_{x_1}^{x_2} \left( \frac{\partial G}{\partial f}
- \frac{\ddd}{\dd x} \frac{\partial G}{\partial f'} \right) \delta f \, \dd x \;
+ \; \left[\frac{\partial G}{\partial f'} \delta f \right]_{x_1}^{x_2}.
\end{equation}
As $\delta Y$ goes to $0$ at the endpoints, so does $\delta f$ which causes the last term to vanish. From the arbitrariness of $\delta Y$ follows the arbitrariness of $\delta f$. The only mechanism for $\delta S$ to vanish is

\begin{equation}
0 = \frac{\partial G}{\partial f} - \frac{\ddd}{\dd x} \frac{\partial G}{\partial f'}
= \frac{\partial G}{\partial y} - \frac{\ddd}{\dd x} \frac{\partial G}{\partial y'}.
\end{equation}
This shows that the Euler-Lagrange equation takes the same form under any coordinate transformation as long as the transformation of the function $G$ is given by $\widetilde{G}(Y,Y',x) := G(f,f',x)$. Functions that behave this way under coordinate transformations as $G$ are called scalar functions, especially in physical contexts.
%Can we give an example of a scalar function in undergraduate level mechanics? "Such as the gravitational potential or temperature over an object"
\footnote{An example for a scalar is air temperature.
To analyze this example we consider two coordinate systems with coordinates $x$ and $X$ and their transformation $x=f(X)$.
If $T=T(x)$ denotes temperature in the first coordinate system then as an analogue to definition \ref{lagrangian-transform} we define the temperature in the second coordinate system by
\begin{equation} \label{definitionOfTemperatureTransform}
  \tilde{T}(X) := T(f(x))
\end{equation}

Now the physical fact that temperature is the same in both coordinate systems leads to the condition $\tilde{T}(X) = T(x)$ which by using definition \ref{definitionOfTemperatureTransform} can be turned into $T(f(x)) = T(x)$.
Scalars that fulfill this condition are called invariant with respect to the transformation $f$.

An interesting point of the temperature example is that this condition becomes invalid if one of the two coordinate systems moves with a velocity that is not negligible compared to the average motion of the air molecules while the other stays relative to the air at rest.
}
In section \ref{lagrangian-def}, $G$ will be interpreted as the Lagrangian we mentioned in section \ref{introduction}. As we now see, $G$ does indeed have the transformation properties we claimed in section \ref{introduction}.
