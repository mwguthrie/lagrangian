\documentclass{article}
%\documentclass[journal]{IEEEtran}
%\documentclass{report}
%\documentclass{acta}

\usepackage{hyperref}
\usepackage{amsmath}
\usepackage{amssymb}
\usepackage{amsthm}
\usepackage{xcolor}

\DeclareMathOperator{\dd}{d\!}
\DeclareMathOperator{\ddd}{\mathrm{d}}


\begin{document}

\title{Special Relativity and Lagrangians}
\author{Gerd Wagner}

\maketitle

\begin{abstract}
In the first part of the paper we derive two foundations of special relativity.
Both are based on the invariance of the space time interval $c^2\Delta t^2 - \Delta x^2 - \Delta y^2 - \Delta z^2$,
i.e. the constance of the speed of light.
These foundations are:
\begin{itemize}
    \item[1.] The invariance for any space time interval not only for that of light.
    \item[2.] The Lorentz transformation.
\end{itemize}

The second part is about some aspects of the Lagrange formulation of particle dynamics:
\begin{itemize}
    \item[1.] We show how the energy of a particle can be derived from studying how its action $S$ changes with time translations.
    \item[2.] We do a detailed discussion of the relativistic particle in an electromagnetic field.
              The major result will be the transformation law of the electromagnetic potentials.
    \item[3.] We motivate the Lagrangian for the free particle in special relativity.
    \item[4.] We derive $E=mc^2$
\end{itemize}

The third part is about relativistic field Lagrangians and their transformation properties.
It's the continuation of our study of non relativistic field theory in \cite{LagrangeOfField}:
\begin{itemize}
    \item[1. (*)] We formulate the Lagrangian formlism of fields for special relativity where in contrast to \cite{LagrangeOfField} time is not a special coordinate anymore.
    \item[2. (**)] We show the invariance of the Euler-Lagrange equations under arbitray transformations of space and time as well as arbitrary transformations of the fields.
              We also explain how the Lagrangian changes under these transformations.
    \item[3.] We use (*) and (**) to turn the classical Lagrangian of electrodynamics into its relativistic form.
    \item[4.] We use (*) and (**) once more and show that the Lagrangian is invariant under Lorentz transformations.
              Which means we show the functional form of the Lagrangian doesn't change and thus that the equations of motion (Maxwell's equations) don't change.
\end{itemize}


\end{abstract}


\section{Introduction}

Lagrangians clear up the relation of equations of motion to coordinates.
This is especially important for special relativity which foundations lie in considering the laws of physics from within different coordinate systems i.e. inertial reference frames.
We want to make clear that once the transformation rules of the Lagrange formalism are derived, major results of relativistic electrodynamics can be found by simply applying these rules.
These results are:
\begin{itemize}
\item[1.] Turning the non relativistic Lagrangian of electrodynamics into its relativistic form by defining a simple coordinate transformation and applying the rules.
\item[2.] Proof that the Lagrangian of electrodynamics and the laws of electrodynamics (Maxwell's equations) are invariant under Lorentz transformations by applying the rules.
\end{itemize}

The second point requires some detailed discussion of relativistic particle physics, which is done in section \ref{particleLagrangian}.
In this discussion we will derive the way electromagnetic potentials transform under Lorentz transformations.

Once we stepped so deep into relativistic particle physics we felt we also had to include the relativistic free particle and derive its famous energy $E=m c^2$


\section{Two foundations of special relativity} \label{foundations}

\subsection{Invariance of the space time interval \\ $\Delta s = \sqrt{c^2\Delta t^2 - \Delta x^2 - \Delta y^2 - \Delta z^2}$ \cite{LandauInterval}} \label{sectionInvarianceSpaceTime}
We consider two inertial reference frames $T$, $T'$ with coordinates $t,x,y,z$ and $t',x',y',z'$.
Inertial reference frames are reference frames in which a body with zero net force acting upon it is not accelerating.
The possible transformations between inertial reference frames are translations in space and time, rotations in space and motion with constant velocity.

From within $T$ and $T'$ we observe a particle and a light pulse.
The light pulse we assume traveling in one direction with velocity $c$ such that it has nearly as well defined position as the particle.

The empiric fact, which started the theory of special relativity is that
\begin{equation} \label{constantsSpeedOfLight}
0 = c^2\Delta t^2 - \Delta x^2 - \Delta y^2 - \Delta z^2 = c^2\Delta t'^2 - \Delta x'^2 - \Delta y'^2 - \Delta z'^2
\end{equation}
holds the for the light pulse's coordinates in the two inertial reference frames $T$ and $T'$.
\footnote{This is the ususal mathematical way to state that the speed of light is the same in all inertial reference frames.
That the speed of light is the same in all inertial reference frames is also known as the second postulate of special relativity.
There are altogether two postulates of special relativity.
The two-postulate basis for special relativity is the one historically used by Einstein, and it remains the starting point today.
The first postulate, which is also called "principle of relativity" and postulates that the laws of physics are the same in all inertial frames of reference,
will become relevant in section \ref{sectionFirstPostulate} of this paper.}


If $T$ and $T'$ are relative to each other at rest the equality
\begin{equation}
c^2\Delta t^2 - \Delta x^2 - \Delta y^2 - \Delta z^2 = c^2\Delta t'^2 - \Delta x'^2 - \Delta y'^2 - \Delta z'^2
\end{equation}
is true for the particle, too.
This is because without relative motion the only transformations left are translations in space and time and rotations in space.
For these transformations the even more restrictive relations

\begin{equation} \label{noVelocityTransform}
\Delta t = \Delta t' \; \text{and} \; \Delta x^2 + \Delta y^2 + \Delta z^2 = \Delta x'^2 + \Delta y'^2 + \Delta z'^2
\end{equation}
hold. (Note 1: If these relations wouldn't hold when $T$ and $T'$ are relative to each other at rest, this would be a strong hint that space wasn't homogeneous and isotropic.
Note 2: The issues that led to special relativity arise only when $T$ and $T'$ are moving relative to each other.)\\

The question we now aim to answer is: Is it true that for any transformation connecting the two inertial reference frames the equation

\begin{equation} \label{invarianceSpaceTimeInterv}
    c^2\Delta t^2 - \Delta x^2 - \Delta y^2 - \Delta z^2 = c^2\Delta t'^2 - \Delta x'^2 - \Delta y'^2 - \Delta z'^2
\end{equation}
holds for the particle's coordinates in $T$ and $T'$, too?

As we already reasoned the only transformations this question isn't already answered for, are those which include a constant relative velocity $\vec{v}$ between $T$ and $T'$.
Since we already know that rotations in space have no effect on

\begin{align*}
    &\Delta s^2 := c^2\Delta t^2 - \Delta x^2 - \Delta y^2 - \Delta z^2 \\
    &\text{and} \\
    &\Delta s'^2 := c^2\Delta t'^2 - \Delta x'^2 - \Delta y'^2 - \Delta z'^2
\end{align*}
the direction of $\vec{v}$ can't have an effect either. So only the absolute value $|\vec{v}|$ of $\vec{v}$ could be responsible for $\Delta s^2 \neq \Delta s'^2 $.
Thus the possible relation between $\Delta s'$ and $\Delta s$ can be expressed by a family of functions $F$ parametrized by $|\vec{v}|$ such that

\begin{equation}
    \Delta s' = F_{|\vec{v}|} (\Delta s)
\end{equation}

Below we will show that $F$ can't depend on $|\vec{v}|$ at all.
If so, then $F$ relates $\Delta s'$ and $\Delta s$ the same way no matter what the transformation between $T$ and $T'$ is.
Thus we can use any special case of a transformation to determine $F$.
Since any of \ref{noVelocityTransform} and \ref{constantsSpeedOfLight} lead to

\begin{equation}
    \Delta s' = \Delta s
\end{equation}
$F$ must be the identity function.
So \ref{invarianceSpaceTimeInterv} holds for the particle, too.
\\


To prove $F$ cannot depend on $|\vec{v}|$ we consider three inertial reference frames $T_1$,$T_2$,$T_3$ with constant relative velocities

\begin{align*}
    &\vec{v_{12}} \; \text{between} \; T_1 \; \text{and} \; T_2 \\
    &\vec{v_{23}} \; \text{between} \; T_2 \; \text{and} \; T_3 \\
    &\vec{v_{13}} \; \text{between} \; T_1 \; \text{and} \; T_3 \\
\end{align*}
The equations for the relations between $\Delta s_1$, $\Delta s_2$, $\Delta s_3$ then would read

\begin{equation}\label{Fofv1}
\Delta s_2 = F_{|\vec{v_{12}}|}(\Delta s_1)
\end{equation}

\begin{equation} \label{Fofv2}
\Delta s_3 = F_{|\vec{v_{13}}|}(\Delta s_1)
\end{equation}

\begin{equation} \label{Fofv3}
\Delta s_3 = F_{|\vec{v_{23}}|}(\Delta s_2)
\end{equation}
Plugin \ref{Fofv1} into \ref{Fofv3} leads to $\Delta s_3 = F_{|\vec{v_{23}}|} \circ F_{|\vec{v_{12}}|} (\Delta s_1)$ which with \ref{Fofv2} leads to the function identity

\begin{equation} \label{FRelation}
F_{|\vec{v_{13}}|} = F_{|\vec{v_{23}}|} \circ F_{|\vec{v_{12}}|}
\end{equation}
at the same time the vector equation $\vec{v_{13}} = \vec{v_{12}} + \vec{v_{23}}$ must hold. For the absolute values this means

\begin{equation}
    |\vec{v_{13}}| = \sqrt{|\vec{v_{12}}|^2 + |\vec{v_{23}}|^2 + 2 |\vec{v_{12}}||\vec{v_{23}}| \cos(\alpha)}
\end{equation}
where $\alpha$ is the angle between $\vec{v_{12}}$ and $\vec{v_{23}}$.

This way the left hand side of equation \ref{FRelation} would depend on $\alpha$ but the right hand side would not. Because of this contradiction $F$ cannot depend on the relative velocity between inertial reference frames.




\subsection{The Lorentz transformation} \label{sectionLorentzTransformation}

\subsubsection{Non relativistic setup} \label{setup}
We observe a particle from within two inertial reference frames $T$ and $T'$.
Inertial reference frames are reference frames in which a body with zero net force acting upon it is not accelerating.
The axis of the frames point in the same direction and $T'$ moves with velocity $v$ along $T$'s $x$-axis such that their classical Galilean transformation is given by:

\begin{equation} \label{Galilei}
\left(\begin{array}{c}
t'
\\
x'
\end{array} \right)
=
\begin{pmatrix}
1 & 0
\\
-v & 1
\end{pmatrix}
\left(\begin{array}{c}
t
\\
x
\end{array} \right)
\;,\; y'=y \;,\; z'=z
\end{equation}

where $t,x,y,z$ are the coordinates of the particle in $T$ and $t',x',y',z'$ are the coordinates of the particle in $T'$.
As we can see from these equations we chose $t=t'=0$ to be the time when the coordinate frames are on top of each other.

\subsubsection{Ansatz for a transformation compatible with invariance of the space time interval}

Since at $t=t'=0$ the two reference frames $T$ and $T'$ are on top of each other \ref{invarianceSpaceTimeInterv} turns into

\begin{equation} \label{spaceTimeElement}
c^2t'^2-x'^2-y'^2-z'^2 = c^2t^2-x^2-y^2-z^2.
\end{equation}
As an ansatz for a coordinate transformation between the two reference frames which is consistent with \ref{spaceTimeElement} we write:
\begin{align} \label{ansatz}
\left(\begin{array}{c}
ct'
\\
x'
\end{array} \right)
&=
\begin{pmatrix}
a & d
\\
b & e
\end{pmatrix}
\left(\begin{array}{c}
ct
\\
x
\end{array} \right) \\
y' &= y \nonumber \\
z' &= z \nonumber
\end{align}


\subsubsection{Finding the matrix elements of the ansatz}

Because of $y'=y, z'=z$, equation \ref{spaceTimeElement} and the ansatz \ref{ansatz} result in the following matrix equation

\begin{equation} \label{matrixEquationLorentzTransform}
\left[
\begin{pmatrix}
a & d
\\
b & e
\end{pmatrix}
\left(\begin{array}{c}
ct
\\
x
\end{array} \right)
\right]^t
\begin{pmatrix}
1 & 0
\\
0 & -1
\end{pmatrix}
\left[
\begin{pmatrix}
a & d
\\
b & e
\end{pmatrix}
\left(\begin{array}{c}
ct
\\
x
\end{array} \right)
\right]
=
\left(\begin{array}{c}
ct
\\
x
\end{array} \right)^t
\begin{pmatrix}
1 & 0
\\
0 & -1
\end{pmatrix}
\left(\begin{array}{c}
ct
\\
x
\end{array} \right)
\end{equation}

Since $t$ and $x$ are arbitrary the only way for this equation to be true is

\begin{equation} \label{minkowski}
\begin{pmatrix}
a & b
\\
d & e
\end{pmatrix}
\begin{pmatrix}
1 & 0
\\
0 & -1
\end{pmatrix}
\begin{pmatrix}
a & d
\\
b & e
\end{pmatrix}
=
\begin{pmatrix}
1 & 0
\\
0 & -1
\end{pmatrix}
\end{equation}

\begin{equation}
\iff
\begin{pmatrix}
a & b
\\
d & e
\end{pmatrix}
\begin{pmatrix}
a & d
\\
-b & -e
\end{pmatrix}
=
\begin{pmatrix}
1 & 0
\\
0 & -1
\end{pmatrix}
\end{equation}

\begin{equation}
\iff
\begin{pmatrix}
a^2-b^2 & ad-be
\\
ad-be & d^2-e^2
\end{pmatrix}
=
\begin{pmatrix}
1 & 0
\\
0 & -1
\end{pmatrix}
\end{equation}




These are three equations for four parameters $a,b,d,e$. This means one free parameter will remain.
We are now going to solve these equations in such a way that $a,d,e$ are expressed through $b$:

\begin{equation}
a(b) = \pm \sqrt{1+b^2}
\end{equation}

\begin{equation}
-1 = d^2-e^2 = \frac{b^2e^2}{a^2} -e^2 = e^2 \bigg(\frac{b^2}{a^2} - 1 \bigg)
\end{equation}

\begin{equation}
\iff e^2 = \frac{1}{1-\frac{b^2}{a^2}} = \frac{1}{1-\frac{b^2}{1+b^2}} = \frac{1}{\frac{1+b^2-b^2}{1+b^2}} = 1+b^2
\end{equation}

\begin{equation}
\iff e(b) = \pm \sqrt{1+b^2}
\end{equation}


\begin{equation}
d^2 = e^2 -1 = b^2 \iff d(b) = \pm b
\end{equation}

\begin{equation} \label{matrixExpressedByB}
\implies
\begin{pmatrix}
a & d
\\
b & e
\end{pmatrix}
=
\begin{pmatrix}
\sqrt{1+b^2} & b
\\
b & \sqrt{1+b^2}
\end{pmatrix}
\end{equation}

In \ref{matrixExpressedByB} we chose the positive solutions only.
This can be rectified by checking that the matrix \ref{matrixExpressedByB} solves equation \ref{minkowski}.

The transformation law now reads

\begin{equation}
\left(\begin{array}{c}
ct'
\\
x'
\end{array} \right)
=
\begin{pmatrix}
\sqrt{1+b^2} & b
\\
b & \sqrt{1+b^2}
\end{pmatrix}
\left(\begin{array}{c}
ct
\\
x
\end{array} \right)
\;,\; y'=y \;,\; z'=z
\end{equation}

Next we want to move the speed of light $c$ from the vectors to the matrix.
To do so we write the above equation in the following form:

\begin{equation}
\begin{pmatrix}
c & 0
\\
0 & 1
\end{pmatrix}
\left(\begin{array}{c}
t'
\\
x'
\end{array} \right)
=
\begin{pmatrix}
\sqrt{1+b^2} & b
\\
b & \sqrt{1+b^2}
\end{pmatrix}
\begin{pmatrix}
c & 0
\\
0 & 1
\end{pmatrix}
\left(\begin{array}{c}
t
\\
x
\end{array} \right)
\end{equation}

\begin{equation}
\iff
\left(\begin{array}{c}
t'
\\
x'
\end{array} \right)
=
\begin{pmatrix}
1/c & 0
\\
0 & 1
\end{pmatrix}
\begin{pmatrix}
\sqrt{1+b^2} & b
\\
b & \sqrt{1+b^2}
\end{pmatrix}
\begin{pmatrix}
c & 0
\\
0 & 1
\end{pmatrix}
\left(\begin{array}{c}
t
\\
x
\end{array} \right)
\end{equation}

\begin{equation}
\iff
\left(\begin{array}{c}
t'
\\
x'
\end{array} \right)
=
\begin{pmatrix}
\sqrt{1+b^2} /c & b /c
\\
b & \sqrt{1+b^2}
\end{pmatrix}
\begin{pmatrix}
c & 0
\\
0 & 1
\end{pmatrix}
\left(\begin{array}{c}
t
\\
x
\end{array} \right)
\end{equation}

\begin{equation} \label{transformationBasedOnb}
\iff
\left(\begin{array}{c}
t'
\\
x'
\end{array} \right)
=
\begin{pmatrix}
\sqrt{1+b^2} & b /c
\\
c b  & \sqrt{1+b^2}
\end{pmatrix}
\left(\begin{array}{c}
t
\\
x
\end{array} \right)
\end{equation}

\subsubsection{Calculation of $b$} \label{calcOfB}

We would now like to give a physical interpretation of $b$.
To do so we consider \ref{spaceTimeElement} for the special case where the particle moves with the origin of reference frame $T'$.

First we use $y'=y, z'=z$ to simplify \ref{spaceTimeElement} to

\begin{equation} \label{spaceTimeElementReduced}
c^2t'^2-x'^2 = c^2t^2-x^2
\end{equation}

Since the particle resides in the origin of $T'$, $x'$ is zero which simplifies \ref{spaceTimeElementReduced} to

\begin{equation}
ct' = \sqrt{c^2 t^2 - x^2}
\end{equation}

Furthermore the velocity of the particle in $T$ is given by the relative speed $v$ between the two reference frames.
This provides us with a relation between $x$ and $t$: $x/t=v$. Using this we can write

\begin{equation}
ct' = \sqrt{c^2 t^2 - x^2} = \sqrt{1 - \frac{x^2}{c^2t^2}} \; \; \; ct = \sqrt{1 - \frac{v^2}{c^2}} \; \; \; ct
\end{equation}

\begin{equation} \label{timeDilation}
\iff t' = \sqrt{1 - \frac{v^2}{c^2}} \; \; \; t
\end{equation}

This special case of the transformation has to be consistent with our more general transformation given by \ref{transformationBasedOnb}.
Thus the following two equations must hold for our special case:

\begin{equation}
t' = \sqrt{1+b^2} \; \; t + \frac{b}{c} x = \sqrt{1+b^2} \; \; t + \frac{b}{c} v t
\end{equation}
and
\begin{equation}
t' = \sqrt{1 - \frac{v^2}{c^2}} \; \; \; t
\end{equation}

This results in the following condition for b:


\begin{equation}
\sqrt{1 - \frac{v^2}{c^2}} \; \; t = \sqrt{1+b^2} \; \; t + \frac{b}{c} v t
\end{equation}

\begin{equation} \label{find_b}
\iff \sqrt{1 - \frac{v^2}{c^2}} = \sqrt{1+b^2} + \frac{b}{c} v
\end{equation}


It is certainly possible, but tedious to solve \ref{find_b} for $b$. So we provide the solution

\begin{equation}
b = \frac{-v/c}{\sqrt{1 - \frac{v^2}{c^2}}}
\end{equation}

and check it is right:

\begin{equation}
\sqrt{1+b^2} + \frac{b}{c} v
= \sqrt{1+\frac{v^2/c^2}{1 - \frac{v^2}{c^2}}} - \frac{v^2/c^2}{\sqrt{1 - \frac{v^2}{c^2}}}
= \frac{1}{\sqrt{1 - \frac{v^2}{c^2}}}  - \frac{v^2/c^2}{\sqrt{1 - \frac{v^2}{c^2}}}
=  \sqrt{1 - \frac{v^2}{c^2}}
\end{equation}

q.e.d.


\subsubsection{Lorentz transformation} \label{sectLorentzTransform}

In preparation to plug $b$ into \ref{transformationBasedOnb} we first consider

\begin{equation}
\sqrt{1+b^2}
= \sqrt{1 + \frac{v^2/c^2}{1 - \frac{v^2}{c^2}}}
= \sqrt{\frac{1 - v^2/c^2 + v^2/c^2}{1 - \frac{v^2}{c^2}}}
= \frac{1}{\sqrt{1 - \frac{v^2}{c^2}}}
\end{equation}

Using this \ref{transformationBasedOnb} turns into

\begin{equation} \label{lorentz}
\left(\begin{array}{c}
t'
\\
x'
\end{array} \right)
=
\begin{pmatrix}
\frac{1}{\sqrt{1 - \frac{v^2}{c^2}}} & \frac{-v/c^2}{\sqrt{1 - \frac{v^2}{c^2}}}
\\
\frac{-v}{\sqrt{1 - \frac{v^2}{c^2}}}  & \frac{1}{\sqrt{1 - \frac{v^2}{c^2}}}
\end{pmatrix}
\left(\begin{array}{c}
t
\\
x
\end{array} \right)
=
\frac{1}{\sqrt{1 - \frac{v^2}{c^2}}}
\begin{pmatrix}
1 & -v/c^2
\\
-v & 1
\end{pmatrix}
\left(\begin{array}{c}
t
\\
x
\end{array} \right)
\end{equation}

which is called Lorentz transformation.

\subsubsection{Interpretation and generalization} \label{sectionGeneralizationLorentz}

This result with rigor is valid for the inertial reference frame $T'$ in which the particle is at rest at the frames origin.
Even if the particle is accelerated along the $x$ axis, for small time intervals there always exists such an inertial reference frame.
Those reference frames are called 'the particles momentary rest frame'.
\\
\\
If we look at \ref{lorentz} in its full 4 dimensional form

\begin{equation}
\left(\begin{array}{c}
              t'
              \\
              x'
              \\
              y'
              \\
              z'
\end{array} \right)
=
\frac{1}{\sqrt{1 - \frac{v^2}{c^2}}}
\begin{pmatrix}
    1 & -v/c^2 & 0 & 0
    \\
    -v & 1 & 0 & 0
    \\
    0 & 0 & 1 & 0
    \\
    0 & 0 & 0 & 1
\end{pmatrix}
\left(\begin{array}{c}
          t
          \\
          x
          \\
          y
          \\
          z
\end{array} \right)
\end{equation}
it is obvious that it can be multiplied by rotation matrices
\begin{equation}
\begin{pmatrix}
    1 & 0
    \\
    0 & R
\end{pmatrix}
\;,\; \text{where $R$ denotes a 3 dimensional rotation matrix}
\end{equation}
and that these products still fulfill \ref{spaceTimeElement}.
This in mind we can claim, that our derivation is valid for the momentary rest frame of a particle moving in an arbitrary direction relative to $T$.
\\
\\
With less rigor we can claim that for any two inertial reference frames $T$, $T'$ the particle's coordinates transform according to \ref{lorentz}.
But still, \ref{lorentz} fulfills the central proposition \ref{spaceTimeElement} and thus is a good candidate for being the right transformation.
If this was true, the generalizations we just discussed will be applicable, too.


\section{Concepts and notations}

\subsection{Proper time} \label{sectionProperTime}

Be $T'$ with coordinates $t', x_1',x_2',x_3'$ a particle's momentary rest inertial reference frame (IRF) and be $T$ with coordinates $t, x_1,x_2,x_3$ an IRF from which we observe the particle.
Then
\begin{equation}
    c^2dt'^2 - dx'^2 = c^2dt^2 - dx^2
\end{equation}
with $x' := (x_1',x_2',x_3')$ and $x := (x_1,x_2,x_3)$ holds.

Because in the particle's momentary rest IRF $dx'=0$ we are left with
\begin{equation} \label{defProperTime}
c dt' = \sqrt{c^2dt^2 - dx^2} = c \sqrt{1- \frac{v^2}{c^2}} \; dt \iff dt' = \sqrt{1- \frac{v^2}{c^2}} \; dt
\end{equation}
where $v=v(t)$ is the particles momentary velocity in the observer IRF.

According to its definition $dt' = \sqrt{1-\frac{v^2}{c^2}} \; dt$ is the time interval in the particles momentary rest IRF that corresponds to the time interval $dt$ in the observer IRF.
So any observer can calculate how long a time interval $dt$ in his IRF will be in the particles momentary rest IRF by $\sqrt{1-\frac{v^2}{c^2}} \; dt$.
Thus all observers in IRFs will agree upon the time interval $dt'$ in the particles momentary rest IRF.
Through this recipe $dt'$ is the same in any IRF. By convention time intervals in the particle's momentary rest IRF are named $d\tau$ and are called "proper time":
\begin{equation}
    d\tau = \sqrt{1-\frac{v^2}{c^2}} \; dt
\end{equation}
A more formal and even simpler way to see that $d\tau$ is the same in every IRF is: If in one observer IRF the particle travels the distance
\begin{equation}
    dx = \left(\begin{array}{c}
                   dx_1\\
                   dx_2\\
                   dx_3\\
    \end{array} \right)
\end{equation}
in time $dt$ then the value of $c^2 dt^2 - dx^2$ is the same in every other IRF, too.
The same is true for its square root:
\begin{equation}
    \sqrt{c^2 dt^2 - dx^2} = \sqrt{1-\frac{v^2}{c^2}} \; dt = d\tau
\end{equation}

\subsection{4-velocity} \label{section4Velocity}
The invariance of proper time $d\tau$ allows us to define a new object which under Lorentz transformations transforms like $(c \; dt,dx_1,dx_2,dx_3)$:
\begin{equation}
    u := \frac{1}{d \tau}
    \left(\begin{array}{c}
              c \; dt\\
              dx_1\\
              dx_2\\
              dx_3\\
    \end{array} \right)
    = \frac{1}{\sqrt{1-\frac{v^2}{c^2}}}
    \left(\begin{array}{c}
              c \\
              v_1\\
              v_2\\
              v_3\\
    \end{array} \right)
\end{equation}
$u$ is called the particles "4-velocity".

\subsection{Notations} \label{sectionNotations}
From section \ref{sectionGeneralizationLorentz} we know that Lorentz transformations can be written as $4 \times 4$ matrices.
In the following we consider $\Lambda$ as such a matrix:

For any Lorentz transformation $\Lambda$ of a particle's coordinates and time $X := (c t,x_1,x_2,x_3)$ the following equation must hold
\begin{equation} \label{generalConditionForLorentzTransformations}
\Lambda^t g \Lambda = g \;\; \text{where} \;\;
g :=
\begin{pmatrix}
    1 & 0 & 0 & 0
    \\
    0 & -1 & 0 & 0
    \\
    0 & 0 & -1 & 0
    \\
    0 & 0 & 0 & -1
\end{pmatrix}
\;\begin{array}{c}\text{In special relativity} \\ \text{$g$ is called metric tensor} \\ \text{or simply metric.}\end{array}
\end{equation}
This is the general form of \ref{matrixEquationLorentzTransform} and a direct consequence of \ref{invarianceSpaceTimeInterv}.
It is the formal definition of a general Lorentz transformation, i.e. any $4 \times 4$-matrix that fulfills this equation is a Lorentz transformation of $X$.
With these definitions the transformation of $X$ by $\Lambda$ is given by matrix-vector multiplication $\Lambda X$.
Up to now we know three 4-component objects that transform like $X$.
Those are $X$ itself, intervals $dX$ of $X$ and the 4-velocity $u$ from the former section.
Any 4-component object that transforms like $X$ is called a "4-vector".

\subsection{Invariance of the 4-vector product} \label{invarianceOf4VectorProduct}
The product $V^t g W$ of two 4-vectors $V,W$ is invariant under Lorentz-Transformations, i.e. is the same in any IRF.
Proof:
\begin{equation}
    (\Lambda V)^t g (\Lambda W) = V^t \Lambda^t g \Lambda W = V^t g W
\end{equation}

\subsubsection{Proper time revisited}
With the concept of the 4-vector product we can write
\begin{align}
    & dX^{t} g \; dX = c^2 dt^2 - dx^2 \\
    \iff & \sqrt{dX^{t} g \; dX} = c \; \sqrt{1-\frac{v^2}{c^2}} \; dt = c \; d\tau
\end{align}
This is another proof of the invariance of $d\tau$ is invariant under Lorentz-Transformations.
\footnote{
According to the upcoming section \ref{sectionInvariance} the invariance of $d\tau$ under Lorentz-Transformations is given by the fact that
\begin{equation}
    d\tau = \sqrt{1-\frac{v^2}{c^2}} \; dt = \sqrt{1-\frac{v'^2}{c^2}} \; dt'
\end{equation}
is true for any two IRFs $T$ and $T'$ where time is given by $t$ and $t'$ respectively and the particle's velocity is given by $v$ and $v'$ respectively.
}


\section{On the Lagrange formulation of particle dynamics} \label{particleLagrangian}

The main result of this section will be the derivation of the transformation law of the electromagnetic potentials $A,\phi$ as given in \cite{LagrangeOfField} from the invariance of the Lorentz force.
Another important result will be the derivation of $E=mc^2$.

Quite a few preparations in the field of classical non relativistic particle physics will be needed for this.
These preparations are done in the first subsection.

\subsection{On the Lagrange formulation of classical non relativistic particle dynamics}

\subsubsection{Energy conservation} \label{energyConservation}
The Lagrange formalism for particle physics as described in \cite{WagnerGuthrie}, as we will show, allows to derive energy conservation from analyzing how the action $S$ behaves under infinitesimal time translations. 
By doing so we will find a definition for the energy of any physical system that is described by a particle Lagrangian. \\

To kick off we consider the action

\begin{equation}
S = \int_{t_1}^{t_2} L(q(t), \dot{q}(t), t) \dd t
\end{equation}

where $L(q(t), \dot{q}(t), t)$ is the particle's Lagrange function, $q$ the particle's position coordinates and $t$ time. $t_1$,$t_2$ are fixed but arbitrary endpoints of a time interval of which we calculate the particle's action $S$. 
($S$ can be considered as a function of $t_1$ and $t_2$: $S = S(t_1,t_2)$)

We now ask by what amount $S$ changes if time is changed from $t$ to $t + \delta t$ with $\delta t$ being a small time interval.
The resulting change $\delta S$ of $S$ is given by

\begin{equation} \label{defDelS}
\delta S = \int_{t_1 + \delta t}^{t_2 + \delta t} L(q(t), \dot{q}(t), t) \dd t 
- \int_{t_1}^{t_2} L(q(t), \dot{q}(t), t) \dd t
\end{equation}

Next we use the substitution rule which is given by
\begin{equation}
\int_{\varphi(t_1)}^{\varphi(t_2)} f(x) \dd x = \int_{t_1}^{t_2} f(\varphi(t)) \; \dot{\varphi}(t) \dd t
\end{equation}

For $\varphi (t) = t + \delta t$ the rule reads
\begin{equation}
\int_{t_1 + \delta t}^{t_2 + \delta t} f(t) \dd t = \int_{t_1}^{t_2} f(t + \delta t) \dd t
\end{equation}

Since $\delta t$ is small we can write
\begin{equation}
\int_{t_1 + \delta t}^{t_2 + \delta t} f(t) \dd t 
= \int_{t_1}^{t_2} \bigg(f(t) + \frac{\dd f}{\dd t} \delta t \bigg) \dd t
\end{equation} 

Applying this to \ref{defDelS} we arrive at

\begin{equation}
\delta S = \int_{t_1}^{t_2} \bigg(L + \frac{\dd L}{\dd t} \delta t \bigg) \dd t - \int_{t_1}^{t_2} L \dd t
= \delta t \int_{t_1}^{t_2} \frac{\dd L}{\dd t} \dd t
\end{equation}

We will now assume that the particle's trajectory $q(t)$ is its physical trajectory which fulfills the Euler-Lagrange-Equation

\begin{equation} \label{ParticleEulLagEqu}
\frac{\dd}{\dd t} \frac{\partial L}{\partial \dot{q}} - \frac{\partial L}{\partial q} = 0
\end{equation}

To make use of this equation we reformulate $\delta S$ as follows

\begin{equation}
\delta S = \delta t \int_{t_1}^{t_2} \frac{\dd L}{\dd t} \dd t
= \delta t \int_{t_1}^{t_2} \bigg(
\frac{\partial L}{\partial q} \dot{q} + \frac{\partial L}{\partial \dot{q}} \ddot{q} + \frac{\partial L}{\partial t} 
\bigg) \dd t
\end{equation}

Integration by parts of the second term leads to

\begin{equation}
\delta S = \delta t \int_{t_1}^{t_2} \bigg(
\frac{\partial L}{\partial q} \dot{q} -\bigg( \frac{\dd }{\dd t}\frac{\partial L}{\partial \dot{q}} \bigg) \dot{q} 
 + \frac{\partial L}{\partial t}
\bigg) \dd t
+ \delta t \bigg[ \frac{\partial L}{\partial \dot{q}} \dot{q} \bigg]_{t_1}^{t_2}
\end{equation}

With \ref{ParticleEulLagEqu} we are left with

\begin{equation}
\delta S = \delta t \int_{t_1}^{t_2} \frac{\partial L}{\partial t} \dd t
+ \delta t \bigg[ \frac{\partial L}{\partial \dot{q}} \dot{q} \bigg]_{t_1}^{t_2}
= \delta t \int_{t_1}^{t_2} \frac{\partial L}{\partial t} \dd t
+ \delta t \int_{t_1}^{t_2} \frac{\dd}{\dd t} \bigg( \frac{\partial L}{\partial \dot{q}} \dot{q} \bigg) \dd t
\end{equation}

We now found two expressions for $\delta S$:

\begin{equation}
\delta S = \delta t \int_{t_1}^{t_2} \frac{\dd L}{\dd t} \dd t 
\;\; \text{and} \;\;
\delta S = \delta t \int_{t_1}^{t_2} \frac{\partial L}{\partial t} \dd t
+ \delta t \int_{t_1}^{t_2} \frac{\dd}{\dd t} \bigg( \frac{\partial L}{\partial \dot{q}} \dot{q} \bigg) \dd t
\end{equation}

Setting these to equal gives

\begin{align}
\int_{t_1}^{t_2} \frac{\dd L}{\dd t} \dd t
&=  \int_{t_1}^{t_2} \frac{\partial L}{\partial t} \dd t
+ \int_{t_1}^{t_2} \frac{\dd}{\dd t} \bigg( \frac{\partial L}{\partial \dot{q}} \dot{q} \bigg) \dd t \\
\iff
- \int_{t_1}^{t_2} \frac{\partial L}{\partial t} \dd t  
&=  \int_{t_1}^{t_2} \bigg(
\frac{\dd}{\dd t} \bigg( \frac{\partial L}{\partial \dot{q}} \dot{q} \bigg) - \frac{\dd L}{\dd t} \bigg) \dd t \\
\iff
- \int_{t_1}^{t_2} \frac{\partial L}{\partial t} \dd t  
&=  \int_{t_1}^{t_2} \frac{\dd}{\dd t} \bigg(\frac{\partial L}{\partial \dot{q}} \dot{q}  - L \bigg) \dd t
\end{align}


For the case $\partial L / \partial t = 0 $ this leads to
\begin{equation}
0 = \int_{t_1}^{t_2} \frac{\dd}{\dd t} \bigg(\frac{\partial L}{\partial \dot{q}} \dot{q}  - L \bigg) \dd t
\end{equation}

Since $t_1$ and $t_2$ are arbitrary this equation can only be true if

\begin{equation} \label{legendre}
0 = \frac{\dd}{\dd t} \bigg(\frac{\partial L}{\partial \dot{q}} \dot{q}  - L \bigg) 
\implies 
E := \frac{\partial L}{\partial \dot{q}} \dot{q}  - L = const
\end{equation}

\textbf{Interpretation}
\begin{itemize}
\item $E$ is called energy of the particle. $E$ is defined for any particle that can be described by the Lagrange formalism for particle physics. It is constant in time when $\partial L / \partial t = 0 $ i.e. when the particle's Lagrangian doesn't explicitly depend on time but depends on time through the particles coordinates $q(t)$ only.

\item 
$E$ was derived from analyzing how $S$ behaves under infinitesimal time translation. So energy conservation can be considered as a consequence of the behavior of a particle's action $S$ under infinitesimal time translations.

\item
There are more transformations, e.g. spacial translations and rotations, which also lead to conserved quantities. 
Those are easier to derive because for them $\delta S = 0$. They can be found in any classical mechanics text book.

\item
For the standard classical mechanics Lagrangian
\begin{equation}
L = T-V = \frac{1}{2} m v^2 - V 
\end{equation}
the energy calculates to
\begin{equation}
E = \frac{\partial}{\partial v} \bigg(\frac{1}{2} m v^2 - V \bigg) v - \bigg(\frac{1}{2} m v^2 - V\bigg) =  \frac{1}{2} m v^2 + V
\end{equation}

\end{itemize}

\subsubsection{Invariance of a Lagrangian} \label{sectionInvariance}
Let $q=f(\bar{q},t)$ be an invertible and differentiable transformation of coordinates $q=(q_1,q_2, ...,q_n)$ and let $L=L(q, \dot{q},t)$ be a Lagrangian of the $q$.
$L$ is called invariant under the transformation $f$ if

\begin{equation} \label{defInvarianceLagrange1}
    \bar{L}(\bar{q},\dot{\bar{q}},t) = L(\bar{q},\dot{\bar{q}},t)
\end{equation}
where $\bar{L}$ is as usual defined by $\bar{L}(\bar{q},\dot{\bar{q}},t) := L(f(\bar{q}), \dot{f}(\bar{q}),t)$.
From this definition another equivalent formulation of \ref{defInvarianceLagrange1} follows immediately:

\begin{equation} \label{defInvarianceLagrange2}
L(q,\dot{q},t) = L(\bar{q},\dot{\bar{q}},t)
\end{equation}

A good way to picture invariance is that the structure of the Lagrangian is such that the transformation cancels out.
\\ \\
\textbf{Example:}
An example is a Lagrangian of a particle in classical mechanics in cartesian coordinates of the form $L=\frac{1}{2}m v^2 - V(|x|)$ where $x$ and $v$ denote a particle's position and velocity respectively.
If in this case the transformation is a rotation of the particles coordinates $x = f(\bar{x}) := R\bar{x}$ with $1 = R^tR $ then $L$ is invariant under this transformation.
\\ \\
Since the Euler-Lagrange equations are never changed by transformations of the form $q=f(\bar{q},t)$ the following is true:
If a Lagrangian is invariant under $f$ the equations of motion in both coordinates will be the same with just the untransformed and transformed coordinates interchanged.

\subsubsection{Generalization of the invariance of a Lagrangian} \label{sectionGeneralizationInvariance}
We are going to generalize the ideas of the last section in such a way that we are able to include fields.
An example for a particle Lagrangian with fields is that of a charged particle in an electromagnetic field.

The generalized particle Lagrangian we assume to be of the form
\begin{equation}
    L = L \bigg(\psi,\frac{\partial \psi}{\partial q},\frac{\partial \psi}{\partial t},q,\dot{q},t\bigg)
\end{equation}
where $\psi$ denotes a field that generally consists of multiple components (like for example the electric field).
Be
\begin{align}
    &q=f(\bar{q}) \\
    &\psi = F(\bar{\psi})
\end{align}
transformations of the coordinates and the field, then the transformed Lagrangian is as usual defined by
\begin{equation} \label{defineGeneralizedParticleLagrangianTransform}
    \bar{L} \bigg(\bar{\psi},\frac{\partial \bar{\psi}}{\partial \bar{q}},\frac{\partial \bar{\psi}}{\partial t},\bar{q},\dot{\bar{q}},t\bigg)
    := L \bigg(F,\frac{\partial F}{\partial f},\frac{\partial F}{\partial t}, f,\dot{f},t\bigg)
\end{equation}
The Lagrangian is called invariant under the transformations $f$ and $F$, if
\begin{equation} \label{defInvarianceLagrange1Generalized}
    \bar{L} \bigg(\bar{\psi},\frac{\partial \bar{\psi}}{\partial \bar{q}},\frac{\partial \bar{\psi}}{\partial t},\bar{q},\dot{\bar{q}},t\bigg)
    = L \bigg(\bar{\psi},\frac{\partial \bar{\psi}}{\partial \bar{q}},\frac{\partial \bar{\psi}}{\partial t},\bar{q},\dot{\bar{q}},t\bigg)
\end{equation}
from which again immediately follows that
\begin{equation} \label{defInvarianceLagrange2Generalized}
    L \bigg(\bar{\psi},\frac{\partial \bar{\psi}}{\partial \bar{q}},\frac{\partial \bar{\psi}}{\partial t},\bar{q},\dot{\bar{q}},t\bigg)
    = L \bigg(\psi,\frac{\partial \psi}{\partial q},\frac{\partial \psi}{\partial t},q,\dot{q},t\bigg)
\end{equation}
Notes:
\begin{itemize}
    \item The concept of invariance may be generalized to any function which transforms according to \ref{defineGeneralizedParticleLagrangianTransform}.
    \item In the following we will learn that the Lagrangian of a particle in special relativity takes the form $L \sqrt{1 - \frac{v^2}{c^2}}$,
    where only the part $L$ will turn out to be invariant under Lorentz transformations, while $L \sqrt{1 - \frac{v^2}{c^2}}$ as a whole is not invariant under Lorentz transformations.
\end{itemize}



\subsubsection{The Lorentz force and its Lagrangian} \label{sectionLorentzForceLagrangian}

The Lorentz force on a particle with charge $e$ in cartesian coordinates is given by

\begin{equation} \label{lorentzForceLaw}
    F_L = e E + e v \times B
\end{equation}
where $v$ is the particles velocity and $E,B$ are respectively the electric and magnetic fields at the particle's coordinates.
The Lagrangian that corresponds to the Lorentz force is given by

\begin{equation} \label{lorentzForceLagrangian}
    L_L = - e (\phi - A \cdot v)
\end{equation}
where $\phi, A$ are respectively the electric and magnetic potentials as discussed in \cite{LagrangeOfField}

$L_L$ is the part of the Lagrangian of a charged particle in an electro magnetic field that represents the particle's interaction with the electro magnetic field.
In classical mechanics the whole Lagrangian is
\begin{equation}
    L = \frac{1}{2}mv^2 + L_L
\end{equation}

To show that $L_L$ reproduces the Lorentz force $F_L$ we prove

\begin{equation}
    F_L = -\bigg(\frac{\dd}{\dd t} \frac{\partial L_L}{\partial v} - \frac{\partial L_L}{\partial x} \bigg)
\end{equation}

We start with

\begin{equation}
    \frac{\partial L_L}{\partial v_i} = e A_i
\end{equation}

\begin{align}
    \implies & \frac{\dd}{\dd t} \frac{\partial L_L}{\partial v_i} = e \bigg(\frac{\partial A_i}{\partial t} + \frac{\partial A_i}{\partial x_j} v_j\bigg) \\
    & \text{with} \;\; i,j \in \{1,2,3\} \\
    & \text{and implicit sum over duplicate indices} \nonumber
\end{align}
To understand the term $\frac{\partial A_i}{\partial x_j} v_j$ we consider that during some small time interval $\dd t$ the particles coordinates change by $\dd x_j = v_j \dd t$.
Hence $A_i$'s change resulting from the change $\dd x_j$ is given by $\frac{\partial A_i}{\partial x_j} \dd x_j = \frac{\partial A_i}{\partial x_j} v_j \dd t$.
So the coordinate wise contribution of $A_i$ to the total time derivative of $\frac{\partial L_L}{\partial v_i}$ is $\frac{\partial A_i}{\partial x_j} v_j$.

We continue with $\frac{\partial L_L}{\partial x_i}$:

\begin{equation}
    \frac{\partial L_L}{\partial x_i} = -e \frac{\partial \phi}{\partial x_i} + e \frac{\partial A_j}{\partial x_i} v_j
\end{equation}
Thus
\begin{align}
    \frac{\dd}{\dd t} \frac{\partial L_L}{\partial v_i} - \frac{\partial L_L}{\partial x_i}
    & = e \bigg(\frac{\partial A_i}{\partial t} + \frac{\partial A_i}{\partial x_j} v_j\bigg)
    - \bigg( -e \frac{\partial \phi}{\partial x_i} + e \frac{\partial A_j}{\partial x_i} v_j \bigg) \nonumber \\
    & = - e \bigg( -\frac{\partial \phi}{\partial x_i} - \frac{\partial A_i}{\partial t} \bigg)
    - e \bigg( v_j \frac{\partial A_j}{\partial x_i}  - v_j \frac{\partial A_i}{\partial x_j}  \bigg)
\end{align}
We next use the relations $B = \nabla \times A$ and $E = - \nabla \phi - \frac{\partial A}{\partial t}$ which were introduced in  \cite{LagrangeOfField}.

As an intermediate step we consider
\begin{align}
[v \times B]_i &= [v \times (\nabla \times A)]_i \nonumber \\
&=\epsilon_{ijk} v_j [\nabla \times A]_k \nonumber \\
&=\epsilon_{ijk} v_j \epsilon_{kln} \frac{\partial A_n}{\partial x_l} \nonumber \\
&=\epsilon_{kij} \epsilon_{kln} v_j  \frac{\partial A_n}{\partial x_l} \nonumber \\
&=(\delta_{il} \delta_{jn} - \delta_{in} \delta_{jl} ) v_j  \frac{\partial A_n}{\partial x_l} \nonumber \\
&=v_j \frac{\partial A_j}{\partial x_i} - v_j \frac{\partial A_i}{\partial x_j} \nonumber
\end{align}
where $\epsilon_{ijk}$ is the Levi-Civita symbol, $\delta_{ij}$ is the Kronecker delta and where we made use of the rule
$\epsilon_{ijk} \epsilon_{ilm} = \delta_{jl}\delta_{km} - \delta_{jm}\delta_{kl}$.

With that we arrive at
\begin{equation}
    \frac{\dd}{\dd t} \frac{\partial L_L}{\partial v_i} - \frac{\partial L_L}{\partial x_i}
    = - e E_i - e [v \times B]_i = -F_L_i
\end{equation}
which is the result we wanted to prove.


\section{Gauge and Lorentz transformations} \label{sectionGauge}
We will define gauge in the context of a particle Lagrangian and will explore how it is connected to gauge in the context of the electromagnetic potentials.
The aim of this section is to clarify the interaction beween gauges of the electromagnetic potentials and Lorentz transformation.
It will turn out that they have no influence,
i.e. the four components of $(\phi/c, A_1,A_2,A_3)$ will turn out to transform under Lorentz transformations like the components of $(ct,x_1,x_2,x_3)$ in any gauge.

\subsection{Gauge of a particle Lagrangian}
A particle Lagrangian can be changed the following way without changing its equation of motion:

\begin{equation}
    L(q,\dot{q}, t) \rightarrow L(q,\dot{q}, t) + \frac{\dd}{ \dd t} F(q,t)
\end{equation}
where F ist an arbitrary function of the coordinates and time.
A change of this kind is called a gauge.

To prove that the equations of motion are not changed by a gauge we calculate the Euler-Lagrange equations for the right hand side:

\begin{align}
    & \frac{\dd}{\dd t} \frac{\partial \big(L + \frac{\dd F}{\dd t}\big)}{\partial \dot{q}} - \frac{\partial \big(L + \frac{\dd F}{\dd t}\big)}{\partial q} \nonumber \\
    & = \frac{\dd}{\dd t} \frac{\partial L}{\partial \dot{q}} - \frac{\partial L }{\partial q}
        + \frac{\dd}{\dd t} \frac{\partial}{\partial \dot{q}} \frac{\dd F}{\dd t}  - \frac{\partial}{\partial q} \frac{\dd F}{\dd t} \nonumber \\
    & = \frac{\dd}{\dd t} \frac{\partial L}{\partial \dot{q}} - \frac{\partial L }{\partial q}
        + \frac{\dd}{\dd t} \frac{\partial}{\partial \dot{q}} \bigg( \frac{\partial F}{\partial q} \dot{q} + \frac{\partial F}{\partial t} \bigg)
        - \frac{\partial}{\partial q} \bigg( \frac{\partial F}{\partial q} \dot{q} + \frac{\partial F}{\partial t} \bigg) \nonumber \\
    & = \frac{\dd}{\dd t} \frac{\partial L}{\partial \dot{q}} - \frac{\partial L }{\partial q}
        + \frac{\dd}{\dd t} \frac{\partial F}{\partial q}
        - \bigg(\frac{\partial^2 F}{\partial q^2} \dot{q} + \frac{\partial }{\partial t} \frac{\partial F}{\partial q} \bigg) \nonumber \\
    & = \frac{\dd}{\dd t} \frac{\partial L}{\partial \dot{q}} - \frac{\partial L }{\partial q}
        + \frac{\dd}{\dd t} \frac{\partial F}{\partial q}
        - \frac{\dd}{\dd t} \bigg(\frac{\partial F}{\partial q} \bigg) \nonumber \\
    & = \frac{\dd}{\dd t} \frac{\partial L}{\partial \dot{q}} - \frac{\partial L }{\partial q} + 0
\end{align}
This is the same term as without gauge, which proves that the gauge doesn't change the equations of motion.

\subsection{Connection between the gauges of the particle Lagrangian and the electromagnetic field}
A gauge of the electromagnetic potentials is given by

\begin{align}
    A &\rightarrow A' = A + \nabla \lambda(x,t) \\
    \phi &\rightarrow \phi' = \phi - \frac{\partial}{\partial t} \lambda(x,t)
\end{align}
where $\lambda$ is an arbitrary function of the spacial coordinates and time.
This gauge is defined in such a way that it has no effect on the fields $E,B$.
To prove this we calculate the fields from the gauged potentials $A', \phi'$ using the formulas $B = \nabla \times A$ and $E = - \nabla \phi - \frac{\partial A}{\partial t}$ as defined for example in section 4 of \cite{LagrangeOfField}:
\begin{align}
  B' &= \nabla \times A' = \nabla \times (A + \nabla \lambda) = \nabla \times A + 0 = \nabla \times A = B \\
  E' &= -\nabla \phi - \frac{\partial A'}{\partial t} = -\nabla \bigg( \phi - \frac{\partial \lambda}{\partial t} \bigg) -\frac{\partial}{\partial t} ( A + \nabla \lambda) \nonumber \\
   &= -\nabla \phi - \frac{\partial A}{\partial t} + \frac{\partial \nabla \lambda}{\partial t} - + \frac{\partial \nabla \lambda}{\partial t} = -\nabla \phi - \frac{\partial A}{\partial t} = E
\end{align}

Next we will calculate in which way this gauge will affect the Lagrangian $L_L$ of the Lorentz force:
The way $L_L$ changes by a gauge of the electromagnetic potentials is given by

\begin{align}
    L_L = - e (\phi - A \cdot v) \rightarrow & -e(\phi-\frac{\partial \lambda}{\partial t}) - (A + \nabla \lambda) \cdot v) \nonumber \\
    & = - e (\phi - A \cdot v) - e (\frac{\partial \lambda}{\partial t} - \nabla \lambda \cdot v) \nonumber \\
    & = - e (\phi - A \cdot v) - e \frac{\dd \lambda}{\dd t} \nonumber \\
    & = - e (\phi - A \cdot v) - \frac{\dd \; (e \lambda)}{\dd t}
\end{align}
This shows that changing the gauge of the electromagnetic potentials changes the Lagrangian by a total time derivative.
As we saw above this change won't affect the equations of motion.
Both gauges are consistent in so far as they don't effect physical, i.e. measurable, phenomena.

\subsection{The effect of Lorentz transfomations on physical phenomena in the case of a charged particle in an electromagnetic field}

We consider two IRFs $T, \bar{T}$ with coordinates $X=(ct, x_1, x_2, x_3)$ and $\bar{X}=(c \bar{t}, \bar{x_1}, \bar{x_2}, \bar{x_3})$ which move relative to each other with non zero speed.
The equations of motion for a charged particle in an electromagnetic field in these two coordinate systems are given by
\begin{align}
  \frac{\dd}{\dd t} \frac{\partial L_{free}}{\partial v} - \frac{\partial L_{free}}{\partial x} &= e E + e v \times B \\
  \text{and} \nonumber \\
  \frac{\dd}{\dd \bar{t}} \frac{\partial \bar{L}_{free}}{\partial \bar{v}} - \frac{\partial \bar{L}_{free}}{\partial \bar{x}} &= e \bar{E} + e \bar{v} \times \bar{B}
\end{align}
where $L_{free}$ denotes the free partical Lagrangian $L_{free} = - m c^2 \sqrt{1 - \frac{v^2}{c^2}}$.
\footnote{If the particle in $T$ and $\bar{T}$ moves with a speed small compared to the speed of light we may also choose $L_{free} = \frac{1}{2} m v^2$.}

Although the structure of both sets of equations is the same as the first postulate of special relativity requires the ingredients aren't at all, i.e.
\begin{equation}
  \bar{v} \neq v \;,\; \bar{E} \neq E \;,\; \bar{B} \neq B \;,\; \bar{L}_{free} \neq L_{free}
\end{equation}
That these quantities aren't the same in $T$ and $\bar{T}$ is most famously discussed in the introduction of Einstein's first paper on special relativity\cite{EinsteinSpecialRelativity}.

As the change of $E$ and $B$ results from Lorentz transformation's of the potentials $\phi, A$ it must be impossible to express a Lorentz transformation by a gauge transformation of $\phi, A$.
This is because by definition and construction gauge transformations never change the fields.

\subsection{On the Lagrange formulation of relativistic particle dynamics} \label{sectionOnRelativisticParticles}

\subsubsection{The first postulate of special relativity} \label{sectionFirstPostulate}
The first postulate of special relativity says that the laws of physics are the same in all IRFs.
The way to translate this into the Lagrange formalism is to require that a particle's action integral
\begin{equation} \label{relativisticAction}
    S = \int\limits_{\tau_1}^{\tau_2} L \dd \tau
\end{equation}
over it's proper time $\tau$ must be invariant
\footnote{For brevity in the sub sections of \ref{sectionOnRelativisticParticles} by invariant we mean invariant under Lorentz transformations.}.
Notable aspects of this are:
\begin{itemize}
    \item The invariance of $S$ is to be understood in the sense of sections \ref{sectionInvariance} and \ref{sectionGeneralizationInvariance}.
    \item Since the interval $d  \tau$ of proper time is invariant then for $S$ to be invariant the Lagrangian $L$ must be invariant, too.
    \item When $S$ is invariant any stationary point of $S$ is invariant, too.
    \item The equations of motion will be the same in every IRF.
\end{itemize}
In an observer IRF with time $t$ and where the particle's velocity is $v$ we may write \ref{relativisticAction} in the form
\begin{equation} \label{relativisticActionWithObserver}
    S = \int\limits_{\tau_1}^{\tau_2} L \dd \tau = \int\limits_{t_1}^{t_2} L \; \sqrt{1-\frac{v^2}{c^2}} \; \dd t
\end{equation}
where in the last equation we used the substitution rule $\int_{\varphi(t_1)}^{\varphi(t_2)} f(x) \dd x = \int_{t_1}^{t_2} f(\varphi(t)) \; \dot{\varphi}(t) \dd t$ with $\dot{\varphi}(t)= \frac{d \tau}{d t} = \sqrt{1-\frac{v(t)^2}{c^2}}$.


\subsubsection{Application to the Lorentz force law} \label{sectionConsequencesOfInvarianceLorentzForce}

Einstein assumed that the first postulate is true for the laws of electrodynamics.
Here we make the weaker assumption that the Lorentz force law $F_L = eE + ev \times B$ (see \ref{lorentzForceLaw}) is the same in every IRF.
We used to write the part of the action that represents $F_L$ in the following form
\begin{equation}
    S_L = \int\limits_{t_1}^{t_2} L_L \; \dd t \;\; \text{with} \;\; L_L = - e (\phi - A \cdot v)
\end{equation}
which can be given the form \ref{relativisticActionWithObserver} by writing
\begin{equation}
    S_L = \int\limits_{t_1}^{t_2} \frac{L_L}{\sqrt{1-\frac{v^2}{c^2}}} \; \sqrt{1-\frac{v^2}{c^2}} \; \dd t
        = \int\limits_{\tau_1}^{\tau_2} \frac{L_L}{\sqrt{1-\frac{v^2}{c^2}}} \;  \dd \tau
\end{equation}
This equation now requires
\begin{equation}
    \frac{L_L}{\sqrt{1-\frac{v^2}{c^2}}} = \frac{- e (\phi - A \cdot v)}{\sqrt{1-\frac{v^2}{c^2}}}
\end{equation}
to be invariant.
This is equivalent to
\begin{align}
    \frac{L_L}{\sqrt{1-\frac{v^2}{c^2}}} &= \frac{- e (\phi - A \cdot v)}{\sqrt{1-\frac{v^2}{c^2}}}  = - \frac{1}{\sqrt{1-\frac{v^2}{c^2}}} \; e (\frac{\phi}{c} , A_1, A_2, A_3) \; g
    \left(\begin{array}{c}
      c \\
      v_1\\
      v_2\\
      v_3\\
    \end{array} \right) \nonumber \\
    &= - e (\frac{\phi}{c} , A_1, A_2, A_3) \; g \; u \label{invariantLorentzForceLagrangian}
\end{align}
being invariant.
Note that in this formula we used the 4-velocity $u$ from section \ref{section4Velocity}.

To explicitly study the meaning of this we consider the following:
Be $T$ and $T'$ two IRFs.
From both we observe a particle.
Let the particles space and time coordinates in $T$ and $T'$ be given by $X = (ct,x_1,x_2,x_3)$ and $X' = (ct',x_1',x_2',x_3')$ respectively.
Then there exists a Lorentz transformation $\Lambda$ such that $X' = \Lambda X$.
(Note that $X$ and $X'$ describe the same position in space and time, namely the particle's position.)
What invariance of \ref{invariantLorentzForceLagrangian} means, is
\begin{align}
    &\;e (\frac{\phi'(X')}{c} , A_1'(X'), A_2'(X'), A_3'(X')) \; g \; u' \nonumber \\
    = &\;e (\frac{\phi(X)}{c} , A_1(X), A_2(X), A_3(X)) \; g \; u \;\;\; \text{see \ref{defInvarianceLagrange1Generalized} and \ref{defInvarianceLagrange2Generalized}} \nonumber \\
%    = &\;e (\frac{\phi(\Lambda X)}{c} , A_1(\Lambda X), A_2(\Lambda X), A_3(\Lambda X)) \; g \; u
\end{align}
From section \ref{section4Velocity} we know that $u'=\Lambda u$.
Thus the above equation turns into
\begin{align}
    &\;e (\frac{\phi'(X')}{c} , A_1'(X'), A_2'(X'), A_3'(X')) \; g \; \Lambda u \nonumber \\
    = &\;e (\frac{\phi(X)}{c} , A_1(X), A_2(X), A_3(X)) \; g \; u \label{invarianceLorentzLagrange}
\end{align}
The simplest way to fulfill this equation is that
\begin{equation}
    (\frac{\phi'(X')}{c} , A_1'(X'), A_2'(X'), A_3'(X'))
    = \left[ \Lambda
           \left(\begin{array}{c}
                     \phi(X) / c \\
                     A_1(X)\\
                     A_2(X)\\
                     A_3(X)\\
           \end{array} \right) \right]^t
\end{equation}
because then \ref{invarianceLorentzLagrange} turns into
\begin{align}
&\;e (\frac{\phi(X)}{c} , A_1(X), A_2(X), A_3(X)) \; \Lambda^t \; g \; \Lambda u \nonumber \\
= &\;e (\frac{\phi(X)}{c} , A_1(X), A_2(X), A_3(X)) \; g \; u \nonumber
\end{align}
which with \ref{generalConditionForLorentzTransformations} is true.

So we find that in the sense of section \ref{sectionNotations}
\begin{equation}
    \left(\begin{array}{c}
              \phi / c \\
              A_1\\
              A_2\\
              A_3\\
    \end{array} \right) \nonumber \\
\end{equation}
is a 4-vector.
This is an important result!


\subsubsection{The relativistic free particle} \label{sectionRelativisticFreeParticle}
Next we are going to find the Lagrangian for the free particle in special relativity.
Remember: A free particle is a particle with zero net force acting upon it.
\footnote{As in classical mechanics the Lagrangian of a particle in an electromagnetic field will be the sum of the Lagrangian of the free particle and $L_L$.}

The simplest guess that can be thought of is that $L$ is some constant $k$ which is just the same in any IRF.
The way we will proceed is to compare this guess with the classical limit and see if it works.
In case it works we will find out what the value of $k$ is.

We assume we observe the particle from within some observer IRF with time $t$.
In the observer IRF we assume particle's velocity to be given by $v$.
The particles Lagrangian in the observer IRF according to \ref{relativisticActionWithObserver} will then be
\begin{equation}
    L = k \sqrt{1-\frac{v^2}{c^2}}
\end{equation}

To compare $k \sqrt{1 - \frac{v^2}{c^2}}$ to its classical limit, i.e. for the limit $v << c$, we first consider some some mathematical preliminaries:\\

$(1+\epsilon)^\alpha$ for small $\epsilon$ can be approximated by the first two terms of its Taylor series:
\begin{equation}
    (1+\epsilon)^\alpha
    \approx (1+\epsilon)^\alpha \Big|_{\epsilon = 0}
    + \frac{\dd \; (1+\epsilon)^\alpha}{\dd \epsilon}\Big|_{\epsilon = 0} \cdot \epsilon
    = 1 + \alpha (1+\epsilon)^{\alpha -1} \Big|_{\epsilon = 0} \cdot \epsilon
    = 1 + \alpha \epsilon
\end{equation}

Applying this approximation to our relativistic Lagrangian with $\epsilon = - v^2/c^2$ results in

\begin{equation}
    L \approx k \bigg(1 + \frac{1}{2} \bigg(-\frac{v^2}{c^2} \bigg) \bigg)
    = k + \frac{1}{2} \bigg(-\frac{k}{c^2} \bigg) v^2
\end{equation}

Since additional constants (in our case $k$) do not have effect on the equation of motion (i.e. the Euler-Lagrange-Equation) it is sufficient to compare the second term of this approximation to the free particle Lagrangian of classical mechanics, which is given by $L=1/2 m v^2$.
The two become identical if we choose
\begin{equation}
    - \frac{k}{c^2} = m \iff k = -mc^2
\end{equation}

Thus our guess of the relativistic free particle Lagrangian is consistent with classical mechanics for small particle velocity $v$ if we write

\begin{equation} \label{freeRelativistivParticleLagrangian}
L = -mc^2 \sqrt{1-\frac{v^2}{c^2}}
\end{equation}


\subsubsection{Energy of the relativistic free particle ($E=mc^2$)}

With \ref{freeRelativistivParticleLagrangian} the energy of the free particle can according to \ref{legendre} be calculated as follows:

\begin{align}
    E &= \frac{\partial L}{\partial v} v - L \\
    &= -m c^2 \frac{1}{2 \sqrt{1-\frac{v^2}{c^2}}} \bigg(-\frac{2v}{c^2} \bigg) \cdot v
    -\bigg(-mc^2 \sqrt{1-\frac{v^2}{c^2}} \bigg) \\
    &= m c^2 \bigg(\frac{v^2/c^2}{\sqrt{1-\frac{v^2}{c^2}}} + \sqrt{1-\frac{v^2}{c^2}} \bigg) \\
    &= m c^2 \bigg(\frac{v^2/c^2 + 1 - v^2/c^2}{\sqrt{1-\frac{v^2}{c^2}}} \bigg) \\
    &= \frac{m c^2}{\sqrt{1-\frac{v^2}{c^2}}}
\end{align}

For a particle at rest ($v=0$) this takes Einstein's famous form, which tells us that in the theory of special relativity a particle with mass $m$ is assigned an energy $E = m c^2$.


\subsubsection{The equations of motion of the free relativistic particle} \label{sectionEquOfMotionFreeParticle}

The equations of motion for the free relativistic particle are given by

\begin{align}
    &0 = \frac{\dd}{\dd t} \frac{\partial }{\partial v_i} \bigg(-mc^2 \sqrt{1-\frac{v^2}{c^2}}\bigg) - \frac{\partial }{\partial x_i} \bigg(-mc^2 \sqrt{1-\frac{v^2}{c^2}}\bigg)
    \nonumber \\
    \iff &0 = \frac{\dd}{\dd t} \frac{\partial }{\partial v_i} \bigg(-mc^2 \sqrt{1-\frac{v^2}{c^2}}\bigg) \;\; \text{for} \;\; i \in \{1,2,3\}
\end{align}

We start with

\begin{equation}
    \frac{\partial L}{\partial v_i} = - m c^2 \frac{1}{2 \sqrt{1 - \frac{v^2}{c^2}}} \bigg(- \frac{2 v_i}{c^2}\bigg) = \frac{m v_i}{\sqrt{1 - \frac{v^2}{c^2}}}
\end{equation}


\begin{align}
   \implies \frac{\dd}{\dd t} \frac{\partial L}{\partial v_i}
   &= \frac{m \dot{v_i}}{\sqrt{1 - \frac{v^2}{c^2}}} + \frac{m v_i}{-2 \sqrt{1 - \frac{v^2}{c^2}}^3} \frac{-2 v_j \dot{v_j}}{c^2} \;\; \text{with imlicit sum over} \;\; j \in \{1,2,3\} \nonumber \\
   &= \frac{m \dot{v_i}}{\sqrt{1 - \frac{v^2}{c^2}}} + \frac{m v_i}{\sqrt{1 - \frac{v^2}{c^2}}^3} \frac{v_j \dot{v_j}}{c^2} \nonumber \\
   &= \frac{m}{\sqrt{1 - \frac{v^2}{c^2}}} \bigg( \dot{v_i} + \frac{v_i}{\frac{c^2 - v^2}{c^2}} \frac{v_j \dot{v_j}}{c^2} \bigg)\nonumber \\
   &= \frac{m}{\sqrt{1 - \frac{v^2}{c^2}}} \bigg( \dot{v_i} + \frac{v_i}{c^2 - v^2} v_j \dot{v_j} \bigg)\nonumber \\
   &= \frac{m}{\sqrt{1 - \frac{v^2}{c^2}}} \frac{1}{c^2 -v^2} \bigg( (c^2 -v^2) \dot{v_i} + v_i v_j \dot{v_j} \bigg)\nonumber \\
   &= \frac{m}{\sqrt{1 - \frac{v^2}{c^2}}} \frac{1}{c^2 -v^2} \bigg( c^2 \dot{v_i} - v_j v_j \dot{v_i} + v_i v_j \dot{v_j} \bigg)\nonumber \\
   &= \frac{m}{\sqrt{1 - \frac{v^2}{c^2}}} \frac{c^2}{c^2 -v^2} \bigg( \dot{v_i} + \frac{v_i v_j \dot{v_j} - v_j v_j \dot{v_i}}{c^2} \bigg)\nonumber \\
   &= \frac{m}{\sqrt{1 - \frac{v^2}{c^2}}} \frac{c^2}{c^2\big(1- \frac{v^2}{c^2}\big)} \bigg( \dot{v_i} + \frac{v_i v_j \dot{v_j} - v_j v_j \dot{v_i}}{c^2} \bigg)\nonumber \\
   &= \frac{m}{\sqrt{1 - \frac{v^2}{c^2}}^3} \bigg( \dot{v_i} + \frac{v_i v_j \dot{v_j} - v_j v_j \dot{v_i}}{c^2} \bigg) \label{relativisticFreeEquationOfMotion}
\end{align}


The identity


\begin{align}
    [ v \times (v \times \dot{v})]_i &= \epsilon _{ijk} v_j (v \times \dot{v})_k \nonumber \\
    &= \epsilon _{ijk} v_j \epsilon _{kln} v_l \dot{v}_n \nonumber \\
    &= \epsilon_{kij} \epsilon _{kln}  v_j v_l \dot{v}_n \nonumber \\
    &= (\delta_{il} \delta_{jn} - \delta_{in} \delta_{jl})  v_j v_l \dot{v}_n \nonumber \\
    &= v_j v_i \dot{v}_j - v_j v_j \dot{v}_i \nonumber \\
\end{align}

allows to write \ref{relativisticFreeEquationOfMotion} as

\begin{equation}
    \frac{\dd}{\dd t} \frac{\partial L}{\partial v} = \frac{m}{\sqrt{1 - \frac{v^2}{c^2}}^3} (\dot{v} + \frac{1}{c^2} v \times (v \times \dot{v}))
\end{equation}

So the equation of motion for the free relativistic particle reads

\begin{equation}
    0 = \frac{m}{\sqrt{1 - \frac{v^2}{c^2}}^3} (\dot{v} + \frac{1}{c^2} v \times (v \times \dot{v}))
\end{equation}

\subsubsection{Lagrangian of the relativistic particle in an electromagnetic field}

As a results from sections \ref{sectionConsequencesOfInvarianceLorentzForce} and \ref{sectionRelativisticFreeParticle} we can write down the Lagrangian of the relativistic particle in an electromagnetic field:
\begin{align} \label{lagrangianOfRelativisticParticleInEMField}
L &= -mc^2 \sqrt{1-\frac{v^2}{c^2}} \; - e (\phi - A \cdot v) \nonumber \\
&= ( -mc^2 - e (\frac{\phi}{c} , A_1, A_2, A_3) \; g \; u)) \sqrt{1-\frac{v^2}{c^2}}
\end{align}


\subsubsection{The equations of motion of the relativistic particle in an electromagnetic field}
The equations of motion of the relativistic particle in an electromagnetic field
can be calculate from \ref{lagrangianOfRelativisticParticleInEMField} and with the results from sections \ref{sectionLorentzForceLagrangian} and \ref{sectionEquOfMotionFreeParticle} reads:

\begin{align}
    &F_L = \frac{m}{\sqrt{1 - \frac{v^2}{c^2}}^3} (\dot{v} + \frac{1}{c^2} v \times (v \times \dot{v})) \nonumber \\
    \iff & eE + e v \times B = \frac{m}{\sqrt{1 - \frac{v^2}{c^2}}^3} (\dot{v} + \frac{1}{c^2} v \times (v \times \dot{v}))
\end{align}


\section{Relativistic field Lagrangians}

In \cite{LagrangeOfField} we derived the Lagrange formalism for classical fields.
Classical in the sence that time was a special coordinate which was treated well separated from spacial coordinates.
As we learned in section \ref{foundations}, in special relativity spacial coordinates and time are not clearly separated anymore.
This becomes obvious when we look at \ref{lorentz} where in contrast to \ref{Galilei} the spacial coordinate $x$ contributes to time.

That's the reason why we strive to modify the Lagrange formalism for fields in such a way that time and spacial coordinates are treated uniformly:

\begin{equation} \label{relativisticFielLagrangian}
    \mathcal{L} = \mathcal{L}\bigg(\psi,\frac{\partial \psi}{\partial q}\bigg)
\end{equation}

where $q$ denotes spacial coordinates including time and $\psi$ denotes the field.
The field may consist of multiple components.
A well known example for a multiple component field is the electric field which in classical non relativistic electrodynamics consists of three components, that make up its direction in space.

The most intuitive ansatz for an action $S$ created from $\mathcal{L}$ is

\begin{equation} \label{relativisticFielAction}
    S = \int\limits_{A} \mathcal{L}\bigg(\psi,\frac{\partial \psi}{\partial q}\bigg) \dd q^{n}
\end{equation}

where $n$ denotes the number (dimension) of the coordiantes $q$ and $A$ an arbitrary n-dimensional area in the space of the $q$.
\\

As laid out in \cite{WagnerGuthrie} and \cite{LagrangeOfField} we have clear criteria to decide if the Lagrangian formalism arising
from \ref{relativisticFielLagrangian} and \ref{relativisticFielAction} is useful. These criteria are:

\begin{itemize}
    \item[1.] Does the principle of stationary action lead to Euler-Lagrange equations?
    \item[2.] Are the Euler-Lagrange equations invariant under arbitrary differentiable and invertible transformations of the coordinates $q$ and the fields $\psi$?
    \item[3.] Does the Lagrangian $\mathcal{L}$ transform in a well defined way.
\end{itemize}

If these criteria are fulfilled we are well motivated to find Lagrangians for physical field theories such that their field equations
become the Euler-Lagrange equations of these Lagrangians.

\subsection{Euler Lagrange equations \cite{LagrangeOfField}} \label{sectionEulerLagrangeEquation}

We consider the variations $\delta S$ of $S$ that result from variations $\delta \psi$ of the fields.
The variation of the fields is arbitrary except from the condition that it vanishes on the border of $A$ which we denote by $\partial A$:

\begin{equation}
    \delta \psi(q) = 0 \; \text{when} \; q \in \partial A
\end{equation}


The variation of $S$ is given by

\begin{equation} \label{actionVariation}
    \delta S = \int\limits_{A} \frac{\partial \mathcal{L}}{\partial \psi} \delta \psi
               + \frac{\partial \mathcal{L}}{\partial \frac{\partial \psi}{\partial q}} \cdot  \delta \bigg(\frac{\partial \psi}{\partial q}\bigg) \dd q ^n
\end{equation}


If we consider the possibly multidimensional components of $\psi$ indexed by $j$ and the $q$ coordinates by $i$ these summands mean:
\begin{equation}
    \frac{\partial \mathcal{L}}{\partial \psi} \cdot \delta \psi
    = \sum_{j} \frac{\partial \mathcal{L}}{\partial \psi_{j}} \; \delta \psi_{j}
\end{equation}
\begin{equation}
    \frac{\partial \mathcal{L}}{\partial \frac{\partial \psi}{\partial q}} \cdot \delta \bigg(\frac{\partial \psi} {\partial q}\bigg)
    = \sum_{i,j} \frac{\partial \mathcal{L}}{\partial \frac{\partial \psi_{j}}{\partial q_{i}}} \; \delta \bigg(\frac{\partial \psi_{j}} {\partial q_{i}}\bigg)
\end{equation}

We integrate the second summand of \ref{actionVariation} by parts. To do so we use the identity
$\delta \big(\frac{\partial \psi} {\partial q}\big)
= \frac{\partial \psi_2} {\partial q} - \frac{\partial \psi_1} {\partial q}
= \frac{\partial (\psi_2 - \psi_1)} {\partial q}
= \frac{\partial \delta \psi} {\partial q}$

\begin{equation}
    \delta S = \int\limits_{A}
    \frac{\partial \mathcal{L}}{\partial \psi} \cdot \delta \psi
    -\bigg(\frac{\partial}{\partial q} \cdot \bigg( \frac{\partial \mathcal{L}}{\partial \frac{\partial \psi}{\partial q}} \bigg)\bigg) \cdot \delta \psi
    \dd q^n
    + \int\limits_{A} \frac{\partial}{\partial q} \cdot \bigg( \frac{\partial \mathcal{L}}{\partial \frac{\partial \psi}{\partial q}} \cdot \delta \psi \bigg) \dd q^n
\end{equation}
The second integral vanishes because of Gauss's theorem and $\delta \psi(q) = 0$ for any $q$ on the surface $\partial A$ of $A$.

\begin{equation}
\delta S = \int\limits_{A}
\frac{\partial \mathcal{L}}{\partial \psi} \cdot \delta \psi
-\bigg(\frac{\partial}{\partial q} \cdot \bigg( \frac{\partial \mathcal{L}}{\partial \frac{\partial \psi}{\partial q}} \bigg)\bigg) \cdot \delta \psi
\dd q^n
\end{equation}


If we use the same index conventions for the field $\psi$ and $q$ as we did above the last term means

\begin{equation}
    \bigg(\frac{\partial}{\partial q} \cdot \bigg( \frac{\partial \mathcal{L}}{\partial \frac{\partial \psi}{\partial q}} \bigg)\bigg) \cdot \delta \psi
    = \sum_j \bigg(\sum_i \frac{\partial}{\partial q_i} \; \bigg( \frac{\partial \mathcal{L}}{\partial \frac{\partial \psi_j}{\partial q_i}} \bigg)\bigg) \; \delta \psi_j
\end{equation}
where the sum over $i$ is called the divergence of $\partial \mathcal{L} / \partial \frac{\partial \psi_j}{\partial q}$.

The last rewrite of $\delta S$ we do is

\begin{equation}
    \delta S = \int\limits_{A}
    \bigg(
    \frac{\partial \mathcal{L}}{\partial \psi}
    -\frac{\partial}{\partial q} \cdot \bigg( \frac{\partial \mathcal{L}}{\partial \frac{\partial \psi}{\partial q}} \bigg)\bigg) \cdot \delta \psi
    \dd q^n
\end{equation}

Since $\delta \psi$ is arbitrary (except from its border conditions) the only way to make $S$ stationary (which is equivalent to require $\delta S = 0$) is that $\mathcal{L}$ fulfills the condition

\begin{equation} \label{EulerLagrangeField}
    0 = \frac{\partial \mathcal{L}}{\partial \psi}
    -\frac{\partial}{\partial q} \cdot \bigg( \frac{\partial \mathcal{L}}{\partial \frac{\partial \psi}{\partial q}} \bigg)
\end{equation}
This is the Euler-Lagrange equation we were looking for.
Of course this equation actually consist of multiple equations for the coordinates and the field components.
That is why it is common to use the plural and speak of the Euler-Lagrange equation\textbf{s}.

\subsection{Invariance of the Euler-Lagrange equations under transformations \cite{LagrangeOfField}} \label{LagrangeTranformation}

This section is very close to what we did in section 3 of \cite{LagrangeOfField}.
Nonetheless it is worth verifying that the arguments work without time as a special coordinate, too.
\\

Let $q=f(\bar{q})$ be an invertible and differentiable transformation of the coordinates and $\psi=F(\bar{\psi})$ be an invertible and differentiable transformation of the field.
We define the transformed Lagrangian  $\bar{\mathcal{L}}$ by

\begin{equation} \label{LagrTransform}
\bar{\mathcal{L}}\bigg(\bar{\psi}, \frac{\partial \bar{\psi}}{\partial \bar{q}}\bigg)
:= \mathcal{L}\bigg(F(\bar{\psi}) , \frac{\partial F(\bar{\psi})}{\partial f}\bigg)
\bigg| det \frac{\partial f}{\partial \bar{q}} \bigg|
\end{equation}
where $\big| det \frac{\partial f}{\partial \bar{q}} \big|$ is the absolute value of the determinant of the Jacobian matrix of $f$ with respect to the coordinates $\bar{q}$. \\

We will prove that from requiring $S$ to be stationary the two equations

\begin{equation} \label{ELGTransformed}
0 = \frac{\partial \bar{\mathcal{L}}}{\partial \bar{\psi}}
-\frac{\partial}{\partial \bar{q}} \cdot \bigg( \frac{\partial \mathcal{\bar{L}}}{\partial \frac{\partial \bar{\psi}}{\partial \bar{q}}} \bigg)
\end{equation}
and

\begin{equation} \label{ELGUntransformed}
0 = \frac{\partial \mathcal{L}}{\partial \psi}
-\frac{\partial}{\partial q} \cdot \bigg( \frac{\partial \mathcal{L}}{\partial \frac{\partial \psi}{\partial q}} \bigg)
\end{equation}
follow and thus that the Euler-Lagrange equations are independent of arbitrary coordinate and field transformations as long as the transformation of the Lagrangian is given by \ref{LagrTransform}. \\

To do so we consider arbitrary but small variations $\delta \bar{\psi}$ of the field $\bar{\psi}$ that vanish on the surface of an area of space $\bar{A}$.
These we use to find the condition for

\begin{equation}
    S = \int\limits_{\bar{A}} \bar{\mathcal{L}}\bigg(\bar{\psi}, \frac{\partial \bar{\psi}}{\partial \bar{q}}\bigg) \dd \bar{q}^n
    = \int\limits_{\bar{A}} \mathcal{L}\bigg(F(\bar{\psi}), \frac{\partial F(\bar{\psi})}{\partial f}\bigg)
    \bigg| det \frac{\partial f}{\partial \bar{q}} \bigg| \dd \bar{q}^n
\end{equation}
to become stationary.\\

\ref{ELGTransformed} just follows from repeating the considerations of section \ref{sectionEulerLagrangeEquation}. \\

To prove \ref{ELGUntransformed} we look at

\begin{equation}
    S = \int\limits_{\bar{A}} \mathcal{L}\bigg(F(\bar{\psi}), \frac{\partial F(\bar{\psi})}{\partial f}\bigg)
    \bigg| det \frac{\partial f}{\partial \bar{q}} \bigg| \dd \bar{q}^n
\end{equation}
which by using the transformation formula of multidimensional integrals can be turned into

\begin{equation}
    S = \int\limits_{f(\bar{A})} \mathcal{L}\bigg(F(\bar{\psi}), \frac{\partial F(\bar{\psi})}{\partial f}\bigg) \dd f^n
\end{equation}
where $f(\bar{A})$ is the picture of $\bar{A}$ under the coordinate transformation $f$.


Based on this formula the variation $\delta S$ of $S$ is given by

\begin{equation}
    \delta S = \int\limits_{f(\bar{A})}
    \frac{\partial \mathcal{L}}{\partial F} \cdot \delta F
    + \frac{\partial \mathcal{L}}{\partial \frac{\partial F}{\partial f}} \cdot \delta \bigg(\frac{\partial F} {\partial f}\bigg)
    \dd f^n
\end{equation}
where

\begin{equation} \label{deltaFDefinition}
\delta F = \frac{\partial F}{\partial \bar{\psi}} \delta \bar{\psi}
\end{equation}


Integration by parts of the second term leads to

\begin{equation} \label{calcDeltaSSection3}
\begin{split}
    \delta S = \int\limits_{f(\bar{A})}
    \frac{\partial \mathcal{L}}{\partial F} \cdot \delta F
    -\bigg(\frac{\partial}{\partial f} \cdot \bigg( \frac{\partial \mathcal{L}}{\partial \frac{\partial F}{\partial f}} \bigg)\bigg) \cdot \delta F
    \dd f^n
    + \int\limits_{f(\bar{A})} \frac{\partial}{\partial f} \cdot \bigg( \frac{\partial \mathcal{L}}{\partial \frac{\partial F}{\partial f}} \cdot \delta F \bigg) \dd f^n
\end{split}
\end{equation}
where the identity
$\delta \big(\frac{\partial F} {\partial f}\big)
= \frac{\partial F_2} {\partial f} - \frac{\partial F_1} {\partial f}
= \frac{\partial (F_2 - F_1)} {\partial f}
= \frac{\partial \delta F} {\partial f}$
was used. \\

The second integral of \ref{calcDeltaSSection3} can be transformed into an integral over the surface of $f(\bar{A})$ which we denote by $\partial (f(\bar{A}))$. This surface is the same as the picture of the surface of $\bar{A}$ under $f$:
\begin{equation}
    \partial (f(\bar{A})) = f(\partial \bar{A})
    \footnote {The simplest way to picture this equation is to imagine a real area in space, which is described from within two systems of coordinates.}
\end{equation}
To show that the third term vanishes, we will prove, that $\delta F$ is zero for any $q \in \partial (f(\bar{A}))$:
\\
\\

\noindent \textbf{Begin proof}
\\
Let $q$ be an element of $\partial (f(\bar{A}))$.
\footnote{The simplest way to picture this element is to imagine a real point on the surface of the area in space, which is described from within two systems of coordinates.}
Then for $q$ there exists an unique $\bar{q} \in \partial \bar{A}$ which is defined by $q=f(\bar{q})$.
We are going to use the fact from above that $\delta \bar{\psi}(\bar{q}) = 0$.
We recall that the variation $\delta \bar{\psi}$ is a difference between two fields which we name $\bar{\psi}_1$ and $\bar{\psi}_2$ such that
\begin{equation}
    \delta \bar{\psi} = \bar{\psi}_2 - \bar{\psi}_1
\end{equation}
The value of $F$ considered as a function of $q$ is given by
\begin{equation}
    F(q) = F(\bar{\psi}(\bar{q})) \;\; \text{with} \;\; \bar{q} \;\; \text{defined through} \;\; q=f(\bar{q})
    \iff \bar{q} = f^{-1}(q)
\end{equation}
The variation $\delta F$ that results from the difference $\delta \bar{\psi}$ between $\bar{\psi}_1$ and $\bar{\psi}_2$ is given by
\begin{equation}
    \delta F(q) = F(\bar{\psi}_2(\bar{q})) - F(\bar{\psi}_1(\bar{q}))
    = F(\bar{\psi}_1(\bar{q}) + \delta \bar{\psi} (\bar{q})) - F(\bar{\psi}_1(\bar{q}))
    = \frac{\partial F}{\partial \bar{\psi}} \delta \bar{\psi} (\bar{q})
\end{equation}
Since $\delta \bar{\psi}(\bar{q})$ is zero by assumption, $\delta F(q)$ is zero, too, which finishes the proof.
\\
\textbf{End proof}
\\
\\

\noindent As to $\delta S$ we are now left with


\begin{equation}
    \delta S = \int\limits_{f(\bar{A})}
    \bigg(
    \frac{\partial \mathcal{L}}{\partial F}
    -\frac{\partial}{\partial f} \cdot \bigg( \frac{\partial \mathcal{L}}{\partial \frac{\partial F}{\partial f}} \bigg)\bigg) \cdot \delta F
    \dd q^n
\end{equation}

Because of \ref{deltaFDefinition} $\delta F$ is equally arbitrary as $\delta \bar{\psi}$. Thus the only way for $\delta S$ to become zero is

\begin{equation}
    0 =
    \frac{\partial \mathcal{L}}{\partial F}
    -\frac{\partial}{\partial f} \cdot \bigg( \frac{\partial \mathcal{L}}{\partial \frac{\partial F}{\partial f}} \bigg)
\end{equation}

If we now replace $F$ and $f$ according to their definitions by $\psi$ and $q$ this equation turns into \ref{ELGUntransformed} and thus finishes the proof.


\subsection{Relativistic electrodynamics}

\subsubsection{Relativistic from of the Lagrangian of electrodynamics} \label{sectionRelativisticLagrangianElectrodynamics}
In literature on electrodynamics it is common to state that electrodynamics is a relativistic theory.
This is to say that the first postulate, see section \ref{sectionFirstPostulate}, applies to the laws of electrodynamics.

Based on the previous two sections we will show that by two simple transformations of the
non relativistic Lagrangian of electrodynamics it can be made clear what this statement exactly means and in which sense it is true.

In fact based on the results from section \ref{sectionConsequencesOfInvarianceLorentzForce} and appendix \ref{appendixConinuity} we are able to prove that the laws of electrodynamics are the same in every inertial reference frame.
\\

We start with the non relativistic Lagrangian of electrodynamics from \cite{LagrangeOfField}:

\begin{equation} \label{LagrangianElectDynClassical}
    \mathcal{L} = \epsilon_0 \frac{(-\nabla\phi - \frac{\partial A}{\partial t})^2 - c^2 (\nabla \times A)^2}{2} - \rho\phi + j \cdot A
\end{equation}


In the sense of section \ref{LagrangeTranformation} we consider the following transfomations of
\begin{itemize}
    \item time $t$ and the three spacial coodinates $x$
    \item the fields $\phi$ and $A$
    \item the charge density $\rho$ and the current density $j$
\end{itemize}

\begin{equation} \label{coordinateTransformClassical}
\left(\begin{array}{c}
          t
          \\
          x_1
          \\
          x_2
          \\
          x_3
\end{array} \right)
= f(x_0,x_1,x_2,x_3)
:=
\begin{pmatrix}
    1/c & 0 & 0 & 0
    \\
    0 & 1 & 0 & 0
    \\
    0 & 0 & 1 & 0
    \\
    0 & 0 & 0 & 1
\end{pmatrix}
\left(\begin{array}{c}
          x_0
          \\
          x_1
          \\
          x_2
          \\
          x_3
\end{array} \right)
=
\left(\begin{array}{c}
          x_0/c
          \\
          x_1
          \\
          x_2
          \\
          x_3
\end{array} \right)
\end{equation}


\begin{equation} \label{fieldTransformClassical}
    \left(\begin{array}{c}
              \phi
              \\
              A_1
              \\
              A_2
              \\
              A_3
    \end{array} \right)
    = F_A(A_0,A_1,A_2,A_3)
    :=
    \begin{pmatrix}
        c & 0 & 0 & 0
        \\
        0 & 1 & 0 & 0
        \\
        0 & 0 & 1 & 0
        \\
        0 & 0 & 0 & 1
    \end{pmatrix}
    \left(\begin{array}{c}
              A_0
              \\
              A_1
              \\
              A_2
              \\
              A_3
    \end{array} \right)
    =
    \left(\begin{array}{c}
              c A_0
              \\
              A_1
              \\
              A_2
              \\
              A_3
    \end{array} \right)
\end{equation}

\begin{equation} \label{currentTransformClassical}
    \left(\begin{array}{c}
              \rho
              \\
              j_1
              \\
              j_2
              \\
              j_3
    \end{array} \right)
    = F_J (J_0,J_1,J_2,J_3)
    :=
    \begin{pmatrix}
        1/c & 0 & 0 & 0
        \\
        0 & 1 & 0 & 0
        \\
        0 & 0 & 1 & 0
        \\
        0 & 0 & 0 & 1
    \end{pmatrix}
    \left(\begin{array}{c}
              J_0
              \\
              J_1
              \\
              J_2
              \\
              J_3
    \end{array} \right)
    =
    \left(\begin{array}{c}
              J_0/c
              \\
              J_1
              \\
              J_2
              \\
              J_3
    \end{array} \right)
\end{equation}


\footnote{
One may argue that this transformation is nothing but a change of variables or a change of units and to treat it as a transformation of the Lagrangian is exaggerated.
This true in the sense that the equations of motions (Maxwell's equations) can be rewritten in the new variables/units straight forward.
But since this paper stressed the importance of the transformation behavior of the Lagrangian and the Euler-Lagrange equations much, we found it adequate to use it at this point, too.

After all when the transformation is defined there is no discussion needed how it applies to the Lagrangian or to the equations of motion.
All there is to do it is to apply rule \ref{LagrTransform}, algebra and calculus.
Furthermore we think it's worth pointing out that a change of units can be interpreted as a coordinate transformation.
As discussed in \cite{WagnerGuthrie} coordinates are not part of nature but are just a human means  to think about nature.
Realizing that this holds for units too, makes the argument relevant to every day life, where units are quite inevitable.
Nonetheless because of their are arbitrariness units can't be part of reality itself.

Those who tend to philosophize may find that units are a suitable means to illustrate the shadows in Plato's Cave.

Anyway we feel it satisfying that the Lagrange formalism restricts and makes clear the influence that coordinates and units have in scientific thinking.
} % end \footnote


The transformed Lagrangian, which we denote $\mathcal{L}_R$, is according to \ref{LagrTransform} given by

\begin{equation}
    \mathcal{L}_R = \frac{1}{c} \mathcal{L} \bigg(F, \frac{\partial F}{\partial f} \bigg) \;,\; \text{$F$ symbolizes $F_A$ and $F_J$}
\end{equation}
The factor $1/c$ comes from the determinant of the matrix in \ref{coordinateTransformClassical}.
This matrix is the Jacobi matrix of $f$ with respect to $x_0,x_1,x_2,x_3$ and thus according to \ref{LagrTransform} the determinant of this matrix has to be included.

Replacing $\mathcal{L}$, $F$ and $f$ by their definitions results in

\begin{equation}
    \mathcal{L}_R = \frac{1}{c} \bigg\{ \frac{\epsilon_0}{2} \bigg[ \bigg(-c \nabla A_0 - \frac{\partial A}{\partial (\frac{x_0}{c})}\bigg)^2 - c^2 \bigg(\nabla \times A\bigg)^2 \bigg]
    - \big( J_0 A_0 - J \cdot A \big) \bigg\}
\end{equation}

\begin{equation} \label{maxwellHalfRelatifistic}
    = \frac{1}{c} \bigg\{ - \frac{c^2 \epsilon_0}{2} \bigg[ -\bigg(\frac{\partial A}{\partial x_0} + \nabla A_0 \bigg)^2 + \bigg(\nabla \times A\bigg)^2 \bigg]
    - \big( J_0 A_0 - J \cdot A \big) \bigg\}
\end{equation}
where $A$ and $J$ without index denote  $\left( \begin{array}{c} A_1 \\ A_2 \\ A_3 \end{array} \right)$ and $\left( \begin{array}{c} J_1 \\ J_2 \\ J_3 \end{array} \right)$ respectively.
For the next steps we will concentrate on the two terms in square brackets.
First we look at $\big(\frac{\partial A}{\partial x_0} + \nabla A_0 \big)^2$:

\begin{equation} \label{A0_Equation}
    \bigg(\frac{\partial A}{\partial x_0} + \nabla A_0 \bigg)^2
      = \bigg(\frac{\partial A_i}{\partial x_0} + \frac{\partial A_0}{\partial x_i} \bigg) \bigg(\frac{\partial A_i}{\partial x_0} + \frac{\partial A_0}{\partial x_i} \bigg)
\end{equation}
where we implicitly sum over the duplicate index $i$ from $1$ to $3$.
\\
\\
\noindent We will now use the following definition of a new symbol $\partial$:

\begin{equation} \label{partialDerivTimesG}
\partial_0 := \frac{\partial}{\partial x_0} \;,\;
\partial_1 := -\frac{\partial}{\partial x_1} \;,\;
\partial_2 := -\frac{\partial}{\partial x_2} \;,\;
\partial_3 := -\frac{\partial}{\partial x_3}
\end{equation}
\\
\noindent
\textbf{Note:} The defintion can also be written as:
\begin{equation}
    \left(\begin{array}{c}
              \partial_0
              \\
              \partial_1
              \\
              \partial_2
              \\
              \partial_3
    \end{array} \right)
    := g
    \left(\begin{array}{c}
              \frac{\partial}{\partial x_0}
              \\
              \frac{\partial}{\partial x_1}
              \\
              \frac{\partial}{\partial x_2}
              \\
              \frac{\partial}{\partial x_3}
    \end{array} \right)
    =
    \left(\begin{array}{c}
              \frac{\partial}{\partial x_0}
              \\
              -\frac{\partial}{\partial x_1}
              \\
              -\frac{\partial}{\partial x_2}
              \\
              -\frac{\partial}{\partial x_3}
    \end{array} \right)
\end{equation}
where the metric tensor $g$ was defined in \ref{generalConditionForLorentzTransformations}.
\\
\\
\noindent
With definition \ref{partialDerivTimesG} equation \ref{A0_Equation} turns into

\begin{equation}
\bigg(\frac{\partial A}{\partial x_0} + \nabla A_0 \bigg)^2
= \big(\partial_0 A_i - \partial_i A_0 \big) \big(\partial_0 A_i - \partial_i A_0 \big)
\end{equation}

\begin{equation}
= \frac{1}{2} \bigg[  \big(\partial_0 A_i - \partial_i A_0 \big) \big(\partial_0 A_i - \partial_i A_0 \big)
                    + \big(\partial_i A_0 - \partial_0 A_i \big) \big(\partial_i A_0 - \partial_0 A_i \big)\bigg]
\end{equation}


with $F_{0i} := \partial_0 A_i - \partial_i A_0 $ and $F_{i0} := \partial_i A_0 - \partial_0 A_i \big$

\begin{equation}
    = \frac{1}{2} \bigg[ F_{0i}F_{0i} + F_{i0}F_{i0} \bigg]
\end{equation}


Next we look at $(\nabla \times A)^2$:

\begin{equation}
    (\nabla \times A)^2 = \epsilon_{ijk} \epsilon_{ilm} \frac{\partial A_k}{\partial x_j} \frac{\partial A_m}{\partial x_l}
\end{equation}


where again we implicitly sum over all duplicate indexes from $1$ to $3$.


\begin{equation}
    = \epsilon_{ijk} \epsilon_{ilm} \partial_j A_k \partial_l A_m
\end{equation}

with the rule $\epsilon_{ijk} \epsilon_{ilm} = \delta_{jl}\delta_{km} - \delta_{jm}\delta_{kl}$ this can be written as

\begin{equation}
    = (\delta_{jl}\delta_{km} - \delta_{jm}\delta_{kl}) \partial_j A_k \partial_l A_m
\end{equation}

\begin{equation}
    = \partial_j A_k \partial_j A_k - \partial_j A_k \partial_k A_j
\end{equation}

\begin{equation}
    = \frac{1}{2} \bigg[   \partial_j A_k \partial_j A_k - \partial_j A_k \partial_k A_j
                         + \underbrace{\partial_k A_j \partial_k A_j}_\text{$= \partial_j A_k \partial_j A_k$}  - \partial_j A_k \partial_k A_j \bigg]
\end{equation}

\begin{equation}
    = \frac{1}{2} \bigg[ (\partial_j A_k - \partial_k A_j) (\partial_j A_k - \partial_k A_j)  \bigg]
\end{equation}

with $F_{jk} := (\partial_j A_k - \partial_k A_j)$

\begin{equation}
    = \frac{1}{2} F_{jk} F_{jk}
\end{equation}

Thus for the term $\bigg[ -\bigg(\frac{\partial A}{\partial x_0} + \nabla A_0 \bigg)^2 + \bigg(\nabla \times A\bigg)^2 \bigg]$
of equation \ref{maxwellHalfRelatifistic} we find:

\begin{equation}
    -\bigg(\frac{\partial A}{\partial x_0} + \nabla A_0 \bigg)^2 + \bigg(\nabla \times A\bigg)^2
    = \frac{1}{2} \bigg[ -(F_{0i} F_{0i} + F_{i0} F_{i0}) + F_{jk} F_{jk}\bigg]
\end{equation}


We define $F_{\mu\nu} := \partial_\mu A_\nu - \partial_\nu A_\mu$ with $\mu, \nu \in \{0,1,2,3\}$.
If we take into account that from this definition $F_{00} = 0$ follows, we can write

\begin{equation}
-\bigg(\frac{\partial A}{\partial x_0} + \nabla A_0 \bigg)^2 + \bigg(\nabla \times A\bigg)^2
= \frac{1}{2} F_{\mu\nu}F_{\alpha\beta} g_{\mu\alpha} g_{\nu\beta}
\end{equation}
where $g_{\mu\nu}$ are the elements of the metric tensor \ref{generalConditionForLorentzTransformations}.
\\
\\
\noindent
\textbf{Note:} For the rest of this paper we will always consider greek indexes to run from $0$ to $3$ and will always implicitly sum over duplicate indexes.
\footnote{A note for experienced readers: In this paper we don't user upper and lower indices, we write metric tensors instead.}
\\
\\
\noindent
We are now ready to put this result back into \ref{maxwellHalfRelatifistic}:

\begin{equation}
    \mathcal{L}_R = \frac{1}{c} \bigg\{ - \frac{c^2 \epsilon_0}{2} \frac{1}{2} F_{\mu\nu}F_{\alpha\beta} g_{\mu\alpha} g_{\nu\beta} - \big( J_0 A_0 - J \cdot A \big) \bigg\}
\end{equation}

with $1/\mu_0=c^2 \epsilon_0 $ this turns into

\begin{equation} \label{electrodynLagrangeRelativistic}
    \mathcal{L}_R = \frac{1}{c} \bigg\{ - \frac{1}{4 \mu_0} F_{\mu\nu}F_{\alpha\beta} g_{\mu\alpha} g_{\nu\beta} - J_\mu A_\nu g_{\mu\nu} \bigg\}
\end{equation}

\subsubsection{Interpretation}
\begin{itemize}
    \item[1.] $\mathcal{L}_R$ in the form \ref{electrodynLagrangeRelativistic} is called the relativistic Lagrangian of electrodynamics.
              The reason why it is called "relativistic" will be explained in the next section.
    \item[2.] \ref{electrodynLagrangeRelativistic} is the result of the transformation given by \ref{coordinateTransformClassical},
              \ref{fieldTransformClassical} and \ref{currentTransformClassical} and the application of the transformation rule \ref{LagrTransform}.
\end{itemize}

\subsubsection{Why the laws of electrodynamics are the same in every inertial reference frame}

For $\mathcal{L}_R$ in the form \ref{electrodynLagrangeRelativistic} we consider another transformation of the coordinates $x_0, x_1, x_2, x_3$ the fields $A_0,A_1,A_2,A_3$ and $J_0,J_1,J_2,J_3$:

\begin{equation} \label{coordinateTransform}
    x_\mu = f(\bar{x})_\mu := \Lambda_{\mu\nu} \bar{x}_\nu
\end{equation}

\begin{equation} \label{fieldTransform}
    A_\mu = F_A(\bar{A})_\mu := \Lambda_{\mu\nu} \bar{A}_\nu
\end{equation}

\begin{equation} \label{currentTransform}
    J_\mu = F_J(\bar{J})_\mu := \Lambda_{\mu\nu} \bar{J}_\nu
\end{equation}
where $\Lambda$ is an arbitrary Lorentz transformation as discussed in section \ref{sectionNotations}.
In the above equations we used the same index based notation as we did in the end of section \ref{sectionRelativisticLagrangianElectrodynamics}.
In the following we will go on to use this notation.
Please be aware that based on this notation equation \ref{generalConditionForLorentzTransformations} can be written as
\begin{equation} \label{invarianceWithMetricTensor}
    g_{\mu\nu} = \Lambda^t_{\mu\alpha} g_{\alpha\beta} \Lambda_{\beta\nu}
    \; \iff \; g_{\mu\nu} = \Lambda_{\alpha\mu} g_{\alpha\beta} \Lambda_{\beta\nu}
\end{equation}
\\
\noindent
\textbf{Note:}

\noindent
First and foremost we are  interested in what happens to $\mathcal{L}_R$ when this transformation is applied using rule \ref{LagrTransform}.
To do so we don't have to be aware of the physical meaning of the transformation.
But for readers who don't appreciate this abstraction it might be helpful that
\begin{itemize}
    \item \ref{coordinateTransform} is the way the coordinates transform in the real physical world.
    This was discussed in section \ref{sectionLorentzTransformation}.
    \item \ref{fieldTransform} was shown to be true in section \ref{sectionConsequencesOfInvarianceLorentzForce}.
    \item \ref{currentTransform} is shown in appendix \ref{appendixConinuity}.
\end{itemize}
\\
\noindent
First we look after the absolute value of the determinant of the Jacobian matrix in \ref{LagrTransform}:

\begin{equation}
    \frac{\partial f}{\partial \bar{x}} = \Lambda \implies \bigg| det \frac{\partial f}{\partial \bar{x}} \bigg| = |det \Lambda|
\end{equation}

from $g = \Lambda^t g \Lambda$ follows
\begin{equation} \label{determinantOfLorentzTransform}
    -1 = det g = det g \;  det^2 \Lambda \; \implies \; 1 = det^2 \Lambda \; \implies \; |det \Lambda | = 1
\end{equation}

As a preparation for writing down $\bar{\mathcal{L}}_R$ we rewrite \ref{electrodynLagrangeRelativistic} with $F$ and $\partial$ replaced by there definitions:

\begin{align}
    \mathcal{L}_R = & \frac{1}{c} \bigg\{ -\frac{1}{4\mu_0} \label{LagrangePartialA}
    \big(\partial_\mu A_\nu - \partial_\nu A_\mu \big)
    g_{\mu\alpha} g_{\nu\beta}
    \big(\partial_\alpha A_\beta - \partial_\beta A_\alpha\big) - J_\mu g_{\mu\nu} A_\nu \bigg\}
    \\
     = & \frac{1}{c} \bigg\{ -\frac{1}{4\mu_0}  \nonumber \\
    & \bigg(g_{\mu\tau} \frac{\partial A_\nu}{\partial x_\tau} - g_{\nu\epsilon} \frac{\partial A_\mu}{\partial x_\epsilon} \bigg)
    g_{\mu\alpha} g_{\nu\beta}
    \bigg(g_{\alpha\pi} \frac{\partial A_\beta}{\partial x_\pi} - g_{\beta\gamma} \frac{\partial A_\alpha}{\partial x_\gamma} \bigg) - J_\mu g_{\mu\nu} A_\nu \bigg\}
\end{align}



By applying \ref{LagrTransform} we find the transformed Lagrangian $\bar{\mathcal{L}}_R$ to be

\begin{align}
    \bar{\mathcal{L}}_R = & \frac{1}{c} \bigg\{ -\frac{1}{4\mu_0}  \nonumber \\
    & \bigg(g_{\mu\tau} \frac{\partial F_A(\bar{A})_\nu}{\partial f(\bar{x})_\tau} - g_{\nu\epsilon} \frac{\partial F_A(\bar{A})_\mu}{\partial f(\bar{x})_\epsilon} \bigg)
    g_{\mu\alpha} g_{\nu\beta}
    \bigg(g_{\alpha\pi} \frac{\partial F_A(\bar{A})_\beta}{\partial f(\bar{x})_\pi} - g_{\beta\gamma} \frac{\partial F_A(\bar{A})_\alpha}{\partial f(\bar{x})_\gamma} \bigg) \nonumber \\
    & - F_J(\bar{J})_\mu g_{\mu\nu} F_A(\bar{A})_\nu \bigg\}
\end{align}

We look for a way to express $g_{\mu\tau} \frac{\partial}{\partial f(\bar{x})_\tau}$ by components of $\frac{\partial}{\partial \bar{x}}$.
To do so we start with the chain rule:

\begin{equation}
    \frac{\partial}{\partial \bar{x}_\alpha} = \frac{\partial f_\nu}{\partial \bar{x}_\alpha} \frac{\partial}{\partial f_\nu}
\end{equation}
because of \ref{coordinateTransform} this turns into

\begin{equation}
    \frac{\partial}{\partial \bar{x}_\alpha} = \Lambda_{\nu\alpha} \frac{\partial}{\partial f_\nu}
\end{equation}
multiplying both sides with $\Lambda^{-1}_{\alpha\mu}$ gives

\begin{equation}
    \Lambda^{-1}_{\alpha\mu} \frac{\partial}{\partial \bar{x}_\alpha} = \frac{\partial}{\partial f_\nu} \delta_{\nu\mu} = \frac{\partial}{\partial f_\mu}
\end{equation}
multiplying both sides with $g_{\mu\beta}$ leads to

\begin{equation}
    \Lambda^{-1}_{\alpha\mu} g_{\mu\beta} \frac{\partial}{\partial \bar{x}_\alpha} = g_{\mu\beta} \frac{\partial}{\partial f_\mu}
    \iff (\Lambda^{-1}  g )_{\alpha\beta} \; \frac{\partial}{\partial \bar{x}_\alpha} = g_{\mu\beta} \frac{\partial}{\partial f_\mu}
\end{equation}
using $g = g^t$ and $(\Lambda^{-1} g)^t = g \Lambda^{-1 t}$ this turns into

\begin{equation}
    (g \Lambda^{-1 t} )_{\beta\alpha} \; \frac{\partial}{\partial \bar{x}_\alpha} = g_{\beta\mu} \frac{\partial}{\partial f_\mu}
\end{equation}
By inverting both sides of \ref{invarianceWithMetricTensor} and using $g=g^{-1}$ we find
$\Lambda^{-1}g\Lambda^{-1t} = g \iff g\Lambda^{-1t} = \Lambda g$.
Thus we can write

\begin{equation}
    (\Lambda g )_{\beta\alpha} \; \frac{\partial}{\partial \bar{x}_\alpha} = g_{\beta\mu} \frac{\partial}{\partial f_\mu}
\end{equation}
We use definition \ref{partialDerivTimesG} for the transformed coordinates and write $\bar{\partial}_\nu := g_{\nu\mu} \frac{\partial}{\partial \bar{x}_\mu}$.
This leads us to

\begin{equation} \label{transformPartial}
    \Lambda_{\beta\nu} \bar{\partial}_\nu = g_{\beta\mu} \frac{\partial}{\partial f_\mu}
    \iff (\Lambda \bar{\partial})_\beta = g_{\beta\mu} \frac{\partial}{\partial f_\mu}
\end{equation}



Now $\bar{\mathcal{L}}_R$ can be written as

\begin{align}
    \bar{\mathcal{L}}_R = \frac{1}{c} \bigg\{ & -\frac{1}{4\mu_0}  \nonumber \\
    & \bigg((\Lambda \bar{\partial})_\mu F_A(\bar{A})_\nu - (\Lambda \bar{\partial})_\nu F_A(\bar{A})_\mu \bigg)
    g_{\mu\alpha} g_{\nu\beta}
    \bigg((\Lambda \bar{\partial})_\alpha F_A(\bar{A})_\beta - (\Lambda \bar{\partial})_\beta F_A(\bar{A})_\alpha \bigg) \nonumber \\
    & - F_J(\bar{J})_\mu g_{\mu\nu} F_A(\bar{A})_\nu \bigg\}
\end{align}
Next we use \ref{fieldTransform} and \ref{currentTransform}:

\begin{align} \label{LagrangeTransformed2}
    \bar{\mathcal{L}}_R = \frac{1}{c} \bigg\{ & -\frac{1}{4\mu_0}  \nonumber \\
    & \bigg((\Lambda \bar{\partial})_\mu (\Lambda\bar{A})_\nu - (\Lambda \bar{\partial})_\nu (\Lambda \bar{A})_\mu \bigg)
    g_{\mu\alpha} g_{\nu\beta}
    \bigg((\Lambda \bar{\partial})_\alpha (\Lambda \bar{A})_\beta - (\Lambda \bar{\partial})_\beta (\Lambda \bar{A})_\alpha \bigg) \nonumber \\
    & - (\Lambda \bar{J})_\mu g_{\mu\nu} (\Lambda \bar{A})_\nu \bigg\}
\end{align}
Multplying out the first summand one of the resulting terms is

\begin{align}
    & (\Lambda \bar{\partial})_\mu (\Lambda\bar{A})_\nu g_{\mu\alpha} g_{\nu\beta} (\Lambda \bar{\partial})_\alpha (\Lambda \bar{A})_\beta \nonumber
    \\
    & =
      \Lambda_{\mu\epsilon} \bar{\partial}_\epsilon \Lambda_{\nu\tau} \bar{A}_\tau
      g_{\mu\alpha} g_{\nu\beta}
      \Lambda_{\alpha\pi} \bar{\partial}_\pi \Lambda_{\beta\sigma} \bar{A}_\sigma \nonumber
    \\
    & =
      \Lambda_{\mu\epsilon} g_{\mu\alpha} \Lambda_{\alpha\pi} \;\;
      \Lambda_{\nu\tau} g_{\nu\beta} \Lambda_{\beta\sigma} \;\;
      \bar{\partial}_\epsilon \bar{A}_\tau \bar{\partial}_\pi \bar{A}_\sigma \nonumber
    \\
      & =
      (\Lambda^tg\Lambda)_{\epsilon\pi} \;\; (\Lambda^tg\Lambda)_{\tau\sigma} \;\;
      \bar{\partial}_\epsilon \bar{A}_\tau \bar{\partial}_\pi \bar{A}_\sigma \nonumber
    \\
    & = g_{\epsilon\pi} \;\; g_{\tau\sigma} \;\; \bar{\partial}_\epsilon \bar{A}_\tau \bar{\partial}_\pi \bar{A}_\sigma
\end{align}
Interchanging indexes the following way $\epsilon \rightarrow \mu, \tau \rightarrow \nu, \pi \rightarrow \alpha, \sigma \rightarrow \beta$ turns this into

\begin{equation}
    g_{\mu\alpha} \;\; g_{\nu\beta} \;\; \bar{\partial}_\mu \bar{A}_\nu \bar{\partial}_\alpha \bar{A}_\beta
\end{equation}
With analogous calculations for the remaining terms \ref{LagrangeTransformed2} turns into

\begin{equation} \label{LagrangeTransformFinal}
    \bar{\mathcal{L}}_R = \frac{1}{c} \bigg\{ -\frac{1}{4\mu_0}
    \big(\bar{\partial}_\mu \bar{A}_\nu - \bar{\partial}_\nu \bar{A}_\mu \big)
    g_{\mu\alpha} g_{\nu\beta}
    \big(\bar{\partial}_\alpha \bar{A}_\beta - \bar{\partial}_\beta \bar{A}_\alpha\big)
    - \bar{J}_\mu g_{\mu\nu} \bar{A}_\nu \bigg\}
\end{equation}

\subsubsection{Interpretation} \label{invarianceMaxwell}

We found that the following equation holds:
\begin{equation}
    \bar{\mathcal{L}}_R(\bar{A}, \frac{\partial \bar{A}}{\partial \bar{x}}) = \mathcal{L}_R(\bar{A}, \frac{\partial \bar{A}}{\partial \bar{x}})
\end{equation}
using \ref{LagrTransform} and the fact that $|det\lambda| = 1$, see \ref{determinantOfLorentzTransform}, we can also write
\begin{equation}
    \bar{\mathcal{L}}_R(\bar{A}, \frac{\partial \bar{A}}{\partial \bar{x}}) = \mathcal{L}_R(A, \frac{\partial A}{\partial x})
\end{equation}
So we find that the Lagrangian of the electromagnetic field satisfies a similar notion of invariance as discussed in sections \ref{sectionInvariance} and \ref{sectionGeneralizationInvariance}
for particle Lagrangians.

Since the Euler-Lagrange equations don't change under transformations at all, it is obvious that the fields' equations of motion also look the same
in the transformed and untransformed coordinates and fields.
That is why the laws of electrodynamics (Maxwell's equations) are the same in every inertial reference frame.



\appendix

\section{Transformation of $J_\mu$} \label{appendixConinuity}


We are going to motivate why \ref{currentTransform} is the physically correct behavior of $J_\mu$ under Lorentz transformations.
We use the empiric fact that the electric charge is conserved in every inertial reference frame.
The formula that describes this fact is the continuity equation:

\begin{equation}
    0 = \frac{\partial \rho}{\partial t} + \nabla \cdot j = \frac{\partial (c \rho)}{\partial (ct)} + \frac{\partial j_i}{\partial x_i}
\end{equation}
with \ref{coordinateTransformClassical}, \ref{currentTransformClassical} and \ref{partialDerivTimesG} this equation can be written as

\begin{equation} \label{relativisticContinuity}
    0 = \partial_0 J_0 - \partial_i J_i = g_{\mu\nu} \partial_\mu J_{\nu}
\end{equation}


Let $T$ and $\bar{T}$ be two inertial reference frames with coordinates $x_0, x_1, x_2, x_3$ and $\bar{x}_0, \bar{x}_1, \bar{x}_2, \bar{x}_3$
If we denote the current in $\bar{T}$ by $\bar{J}$ the continuity equation in $\bar{T}$ reads:

\begin{equation} \label{relativisticContinuityTransfored}
    0 = g_{\mu\nu} \bar{\partial}_\mu \bar{J}_{\nu}
\end{equation}

If $\Lambda$ is the Lorentz transformation that connects the frames' coordinates ($x_\mu = \Lambda_{\mu\nu} \bar{x}_\nu$) then according to \ref{transformPartial}
$\partial_\mu = \Lambda_{\mu\nu} \bar{\partial}_\nu$ is true.

Next we need an expression for the connection between $J$ and $\bar{J}$.
Thus we write $J_\mu=G(\bar{J})_\mu$, where $G$ is some yet undefined function.
With this \ref{relativisticContinuity} can be written as

\begin{align} \label{continuityTransformed}
0 & =  g_{\mu\nu} \partial_\mu J_{\nu} \nonumber \\
  & = g_{\mu\nu} \Lambda_{\mu\alpha} \bar{\partial}_\alpha  G(\bar{J})_\nu \nonumber \\
  & = (g \Lambda)_{\nu\alpha} \bar{\partial}_\alpha  G(\bar{J})_\nu \nonumber \\
  & = (\Lambda^t g )_{\alpha\nu} \bar{\partial}_\alpha  G(\bar{J})_\nu \nonumber \\
\end{align}

The simplest guess to make this formula consistent with \ref{relativisticContinuityTransfored} is $G(\bar{J})_\nu = \Lambda_{\nu\beta} \bar{J}_\beta$.
Because then \ref{continuityTransformed} becomes equivalent to \ref{relativisticContinuityTransfored}:

\begin{align}
    0 & = (\Lambda^t g )_{\alpha\nu} \bar{\partial}_\alpha  \Lambda_{\nu\beta} \bar{J}_\beta \nonumber \\
      & = (\Lambda^t g \Lambda)_{\alpha\beta} \bar{\partial}_\alpha  \bar{J}_\beta \nonumber \\
      & = g_{\alpha\beta} \bar{\partial}_\alpha  \bar{J}_\beta \nonumber \\
      & = g_{\mu\nu} \bar{\partial}_\mu  \bar{J}_\nu \nonumber
\end{align}

\section{Relativistic form of Maxwell's equations}
We calculate the equation of motion \ref{EulerLagrangeField} for an arbitrary component $A_\tau$ of $A$.
In this section we once and again make use of the definition $F_{\mu\nu} := \partial_\mu A_\nu - \partial_\nu A_\mu$ from section \ref{sectionRelativisticLagrangianElectrodynamics}.
We start with

\begin{align}
    \frac{\partial \mathcal{L}}{\partial\frac{\partial A_\tau}{\partial x_\epsilon}}
      = & \frac{\partial}{\partial\frac{\partial A_\tau}{\partial x_\epsilon}}
         \bigg[
            -\frac{1}{4 c \mu_0}
            \big(\partial_\mu A_\nu - \partial_\nu A_\mu \big)
            g_{\mu\alpha} g_{\nu\beta}
            \big(\partial_\alpha A_\beta - \partial_\beta A_\alpha\big)
        \bigg] \\
     = & -\frac{1}{4 c \mu_0} \frac{\partial}{\partial\frac{\partial A_\tau}{\partial x_\epsilon}}
        \bigg[
            \bigg(g_{\mu\sigma} \frac{\partial A_\nu}{\partial x_\sigma} - g_{\nu\pi} \frac{\partial A_\mu}{\partial x_\pi} \bigg)
            g_{\mu\alpha} g_{\nu\beta}
            \bigg(g_{\alpha\eta} \frac{\partial A_\beta}{\partial x_\eta} - g_{\beta\gamma} \frac{\partial A_\alpha}{\partial x_\gamma} \bigg)
        \bigg] \nonumber \\ \\
     = & -\frac{1}{4 c \mu_0}
        \bigg[
             \bigg(g_{\mu\sigma} \delta_{\tau\nu}\delta_{\epsilon\sigma} - g_{\nu\pi}\delta_{\mu\tau}\delta_{\epsilon\pi} \bigg)
                  g_{\mu\alpha} g_{\nu\beta}
             \bigg(g_{\alpha\eta} \frac{\partial A_\beta}{\partial x_\eta} - g_{\beta\gamma} \frac{\partial A_\alpha}{\partial x_\gamma} \bigg) \nonumber \\
        & +  \bigg(g_{\mu\sigma} \frac{\partial A_\nu}{\partial x_\sigma} - g_{\nu\pi} \frac{\partial A_\mu}{\partial x_\pi} \bigg)
                g_{\mu\alpha} g_{\nu\beta}
             \bigg( g_{\alpha\eta}\delta_{\tau\beta}\delta_{\epsilon\eta} - g_{\beta\gamma}\delta_{\tau\alpha}\delta_{\epsilon\gamma} \bigg)
        \bigg] \nonumber \\ \\
    = & -\frac{1}{4 c \mu_0}
    \bigg[
            \bigg(g_{\mu\epsilon} \delta_{\tau\nu} - g_{\nu\epsilon}\delta_{\mu\tau} \bigg)
            g_{\mu\alpha} g_{\nu\beta}
            \bigg(g_{\alpha\eta} \frac{\partial A_\beta}{\partial x_\eta} - g_{\beta\gamma} \frac{\partial A_\alpha}{\partial x_\gamma} \bigg) \nonumber \\
        & +  \bigg(g_{\mu\sigma} \frac{\partial A_\nu}{\partial x_\sigma} - g_{\nu\pi} \frac{\partial A_\mu}{\partial x_\pi} \bigg)
        g_{\mu\alpha} g_{\nu\beta}
        \bigg( g_{\alpha\epsilon}\delta_{\tau\beta} - g_{\beta\epsilon}\delta_{\tau\alpha} \bigg)
    \bigg] \nonumber \\ \\
    = & -\frac{1}{4 c \mu_0}
    \bigg[
            \bigg(\delta_{\alpha\epsilon} g_{\tau\beta} - \delta_{\beta\epsilon}g_{\alpha\tau} \bigg)
            \bigg(g_{\alpha\eta} \frac{\partial A_\beta}{\partial x_\eta} - g_{\beta\gamma} \frac{\partial A_\alpha}{\partial x_\gamma} \bigg) \nonumber \\
        & + \bigg(g_{\mu\sigma} \frac{\partial A_\nu}{\partial x_\sigma} - g_{\nu\pi} \frac{\partial A_\mu}{\partial x_\pi} \bigg)
            \bigg( \delta_{\mu\epsilon}g_{\nu\tau} - \delta_{\nu\epsilon}g_{\mu\tau} \bigg)
    \bigg] \nonumber \\ \\
    = & -\frac{1}{4 c \mu_0}
    \bigg[
        \bigg(
            g_{\epsilon\eta}g_{\tau\beta}\frac{\partial A_\beta}{\partial x_\eta} - \delta_{\tau\gamma}\frac{\partial A_\epsilon}{\partial x_\gamma}
          - \delta_{\tau\eta}\frac{\partial A_\epsilon}{\partial x_\eta} + g_{\epsilon\gamma}g_{\tau\alpha}\frac{\partial A_\alpha}{\partial x_\gamma}
        \bigg) \nonumber \\
    & + \bigg(
           g_{\epsilon\sigma}g_{\nu\tau}\frac{\partial A_\nu}{\partial x_\sigma} - \delta_{\sigma\tau}\frac{\partial A_\epsilon}{\partial x_\sigma}
         - \delta_{\pi\tau}\frac{\partial A_\epsilon}{\partial x_\pi} + g_{\epsilon\pi}g_{\mu\tau}\frac{\partial A_\mu}{\partial x_\pi}
         \bigg)
    \bigg] \nonumber \\ \\
    = & -\frac{1}{4 c \mu_0}
    \bigg[
          g_{\epsilon\eta}g_{\tau\beta}\frac{\partial A_\beta}{\partial x_\eta} - \frac{\partial A_\epsilon}{\partial x_\tau}
          - \frac{\partial A_\epsilon}{\partial x_\tau} + g_{\epsilon\gamma}g_{\tau\alpha}\frac{\partial A_\alpha}{\partial x_\gamma} \nonumber \\
    & +   g_{\epsilon\sigma}g_{\nu\tau}\frac{\partial A_\nu}{\partial x_\sigma} - \frac{\partial A_\epsilon}{\partial x_\tau}
          - \frac{\partial A_\epsilon}{\partial x_\tau} + g_{\epsilon\pi}g_{\mu\tau}\frac{\partial A_\mu}{\partial x_\pi}
    \bigg] \nonumber \\ \\
    = & -\frac{1}{4 c \mu_0}
        \bigg[
          4 \cdot g_{\epsilon\eta}g_{\tau\beta}\frac{\partial A_\beta}{\partial x_\eta} - 4 \cdot \frac{\partial A_\epsilon}{\partial x_\tau}
        \bigg] \nonumber \\
    = & -\frac{1}{c \mu_0}
        \bigg[
          g_{\epsilon\eta}g_{\tau\beta}\frac{\partial A_\beta}{\partial x_\eta} - \frac{\partial A_\epsilon}{\partial x_\tau}
        \bigg] \nonumber \\
    = & -\frac{1}{c \mu_0}
        \bigg[
          g_{\tau\beta} \partial_\epsilon A_\beta - g_{\tau\beta}g_{\beta\alpha}\frac{\partial A_\epsilon}{\partial x_\alpha}
        \bigg] \nonumber \\
    = & -\frac{1}{c \mu_0} g_{\tau\beta}
        \bigg[
           \partial_\epsilon A_\beta - \partial_\beta A_\epsilon
        \bigg] \nonumber \\
    = & -\frac{1}{c \mu_0} g_{\tau\beta} F_{\epsilon\beta} \nonumber
\end{align}

Next we look at
\begin{equation}
    \frac{\partial \mathcal{L}}{\partial A_\tau}
    = \frac{\partial }{\partial A_\tau} \bigg( - \frac{1}{c} J_\mu g_{\mu\nu} A_\nu \bigg)
    = - \frac{1}{c} J_\mu g_{\mu\nu} \delta_{\nu\tau}
    = - \frac{1}{c} J_\mu g_{\mu\tau}
    = - \frac{1}{c} g_{\tau\beta} J_\beta
\end{equation}

Thus according to \ref{EulerLagrangeField} the equation of motion is given by

\begin{equation}
     0 = - \frac{1}{c} g_{\tau\beta} J_\beta
          + \frac{\partial}{\partial x_\epsilon}
            \bigg(
                \frac{1}{c \mu_0} g_{\tau\beta}
                \bigg(
                \partial_\epsilon A_\beta - \partial_\beta A_\epsilon
                \bigg)
            \bigg)
\end{equation}
\begin{equation}
    \iff 0 = g_{\tau\beta} \bigg(J_\beta
             - \frac{1}{\mu_0}  \frac{\partial}{\partial x_\epsilon}\big( \partial_\epsilon A_\beta - \partial_\beta A_\epsilon \big) \bigg)
\end{equation}

\begin{equation}
    \iff 0 = J_\beta - \frac{1}{\mu_0}  \frac{\partial}{\partial x_\epsilon}\big( \partial_\epsilon A_\beta - \partial_\beta A_\epsilon \big)
\end{equation}

\begin{equation}
    \iff \mu_0 J_\beta = g_{\epsilon\alpha}g_{\alpha\pi}\frac{\partial}{\partial x_\pi}\big( \partial_\epsilon A_\beta - \partial_\beta A_\epsilon \big)
\end{equation}

\begin{equation} \label{equationOfMotionUntransformed}
    \iff \mu_0 J_\beta = g_{\epsilon\alpha}  \partial_\alpha \big(\partial_\epsilon A_\beta - \partial_\beta A_\epsilon \big)
                       = g_{\epsilon\alpha}  \partial_\alpha F_{\epsilon\beta}
\end{equation}

This is the relativistic from of Maxwell's equations.

\begin{thebibliography}{9}

\bibitem{LandauInterval} L.D. Landau and E.M. Lifshitz, The classical theory of fields, Chapter 1 Paragraph 2 Intervals

\bibitem{WagnerGuthrie} Gerd Wagner and Matt Guthrie, Demystifying the Lagrangian of Classical Mechanics

\bibitem {SusskindRelativisticLagrange} Susskind Lectures, Special Relativity, Lecture 3, \url{https://theoreticalminimum.com/courses/special-relativity-and-classical-field-theory/2012/spring/lecture-3}

\bibitem{LandauRelativisticLagrange} L.D. Landau and E.M. Lifshitz, The classical theory of fields, Chapter 2 Paragraph 1 The principle of least action

\bibitem{LagrangeOfField} Gerd Wagner, Demystifying the Lagrange formalism of field theory

\bibitem{EinsteinSpecialRelativity} A. Einstein, Zur Elektrodynamik bewegter K\"{o}rper (English Translation: On the Electrodynamics of Moving Bodies), Annalen der Physik 1905, 17, 891--921

\end{thebibliography}



\end{document}
