\documentclass{article}
%\documentclass[journal]{IEEEtran}
%\documentclass{report}
%\documentclass{acta}

\usepackage{hyperref}
\usepackage{amsmath}
\usepackage{amssymb}
\usepackage{amsthm}




\DeclareMathOperator{\dd}{d\!}
\DeclareMathOperator{\ddd}{\mathrm{d}}


\begin{document}

\title{Demystifying the Lagrange formalism of field theory}
\author{Gerd Wagner}

\maketitle

\begin{abstract} This paper takes the same approach as \cite{WagnerGuthrie} to derive and motivate the Lagrangian formulation of field theories. To do so we take the following steps:
\begin{itemize}
\item Give the definition of the action and derive the Euler-Lagrange equations for field theories
\item Prove the Euler-Lagrange equations are independent of arbitrary coordinate transformations and motivate that this independence is desirable for field theories in physics.
\item As an example: Provide the Lagrangian for Electrodynamics as a definition and prove that its Euler-Lagrange equations lead to Maxwell's equations.
\end{itemize}
\end{abstract}


\section{Introduction}

When Lagrangian field theory is introduced it is often presented as a generalization of the Lagrange formalism of classical mechanics, for example see \cite{Goldstein}. The approach of this paper is different: We start by presenting the Lagrangian formulation of field theories as a purely mathematical formalism. We find that it has very well defined coordinate and field transformation properties. Since we consider these properties as valuable for physical field theories the desire to find Lagrangians for these theories in order to turn their field equations into Euler-Lagrange equations is well motivated. 

As a prove of concept we provide the Lagrangian for Electrodynamics as a definition and show that it leads to Maxwell's equations.

This paper is purely non relativistic which means neither the Lagrangian formulation of field theory nor the treatment of Electrodynamics requires concepts of special relativity.

\section{Definition of the Lagrangian formalism for fields}\label{definition}

Note: The experienced reader may well recognize the symbols and names used in this section. Nonetheless she is requested to consider this section as purely mathematical definitions and conclusions. \\

A field is given by a function $\psi(t,x)$ of time $t$ and of three spacial coordinates denoted by $x$. The field's values may be of multiple dimensions. A well known example is the electric field which has a direction in space and whose values are thus three dimensional.

A Lagrangian $\mathcal{L}$ of $\psi$ is defined as a function that may depend on $\psi$ itself as well as on its time and spacial derivatives: 
\begin{equation}
\mathcal{L} = \mathcal{L}\bigg(\psi, \frac{\partial \psi}{\partial t}, \frac{\partial \psi}{\partial x}\bigg)
\end{equation}

The action $S$ for two points $t_1$ and $t_2$ in time and a three dimensional area of space $A$  is defined as the following integral of the Lagrangian:

\begin{equation}
S := \int\limits_{t_1}^{t_2} \int\limits_{A} \mathcal{L}\bigg(\psi, \frac{\partial \psi}{\partial t}, \frac{\partial \psi}{\partial x}\bigg) \dd x^3 \dd t
\end{equation}


Next we are interested in the conditions that $\mathcal{L}$ has to fulfill to make $S$ stationary. To do so we consider arbitrary but small variations $\delta\psi$ of the field and calculate the resulting variation $\delta S$ of $S$. We require the variations $\delta\psi$ to vanish at $t_1$ and $t_2$ as well as on the surface of $A$.

\begin{equation}
\delta S = \int\limits_{t_1}^{t_2} \int\limits_{A} 
\frac{\partial \mathcal{L}}{\partial \psi} \cdot \delta \psi
+ \frac{\partial \mathcal{L}}{\partial \frac{\partial \psi}{\partial t}} \cdot \delta \bigg(\frac{\partial \psi} {\partial t}\bigg)
+ \frac{\partial \mathcal{L}}{\partial \frac{\partial \psi}{\partial x}} \cdot \delta \bigg(\frac{\partial \psi} {\partial x}\bigg)
\dd x^3 \dd t
\end{equation}

If we consider the possibly multidimensional values of $\psi$ indexed by $j$ and the three spacial dimensions indexed by $i$ these summands mean:
\begin{equation}
\frac{\partial \mathcal{L}}{\partial \psi} \cdot \delta \psi 
= \sum_{j} \frac{\partial \mathcal{L}}{\partial \psi_{j}} \; \delta \psi_{j} 
\end{equation}
\begin{equation}
\frac{\partial \mathcal{L}}{\partial \frac{\partial \psi}{\partial t}} \cdot \delta \bigg(\frac{\partial \psi} {\partial t}\bigg)
= \sum_{j} \frac{\partial \mathcal{L}}{\partial \frac{\partial \psi_{j}}{\partial t}} \; \delta \bigg(\frac{\partial \psi_{j}} {\partial t}\bigg)
\end{equation}
\begin{equation}
\frac{\partial \mathcal{L}}{\partial \frac{\partial \psi}{\partial x}} \cdot \delta \bigg(\frac{\partial \psi} {\partial x}\bigg)
= \sum_{i,j} \frac{\partial \mathcal{L}}{\partial \frac{\partial \psi_{j}}{\partial x_{i}}} \; \delta \bigg(\frac{\partial \psi_{j}} {\partial x_{i}}\bigg)
\end{equation}

We now integrate the second summand by parts over time and the third summand by parts over space. To do so we use the identities 
$\delta \bigg(\frac{\partial \psi} {\partial t}\bigg) 
= \frac{\partial \psi_2} {\partial t} - \frac{\partial \psi_1} {\partial t}
= \frac{\partial (\psi_2 - \psi_1)} {\partial t}
= \frac{\partial \delta \psi} {\partial t}$ 
and
$\delta \bigg(\frac{\partial \psi} {\partial x}\bigg) 
= \frac{\partial \psi_2} {\partial x} - \frac{\partial \psi_1} {\partial x}
= \frac{\partial (\psi_2 - \psi_1)} {\partial x}
= \frac{\partial \delta \psi} {\partial x}$ 

\begin{equation} \label{calcDeltaSSection2}
\begin{split}
\delta S = \int\limits_{t_1}^{t_2} \int\limits_{A} 
\frac{\partial \mathcal{L}}{\partial \psi} \cdot \delta \psi
-\frac{\partial}{\partial t} \bigg( \frac{\partial \mathcal{L}}{\partial \frac{\partial \psi}{\partial t}} \bigg) \cdot \delta \psi
-\bigg(\frac{\partial}{\partial x} \cdot \bigg( \frac{\partial \mathcal{L}}{\partial \frac{\partial \psi}{\partial x}} \bigg)\bigg) \cdot \delta \psi
\dd x^3 \dd t \\
+ \int\limits_{A} \int\limits_{t_1}^{t_2} \frac{\partial}{\partial t} \bigg(\frac{\partial \mathcal{L}}{\partial \frac{\partial \psi}{\partial t}} \cdot \delta \psi \bigg) \dd t \dd x^3
+ \int\limits_{t_1}^{t_2} 
\int\limits_{A} \frac{\partial}{\partial x} \cdot \bigg( \frac{\partial \mathcal{L}}{\partial \frac{\partial \psi}{\partial x}} \cdot \delta \psi \bigg) \dd x^3  \dd t
\end{split}
\end{equation}

The second integral vanishes because of the fundamental theorem of calculus and $\delta \psi(t_1) = \delta \psi(t_2) = 0$ the third vanishes because of Gauss's theorem and $\delta \psi(x) = 0$ for any $x$ on the surface of $A$.

\begin{equation}
\delta S = \int\limits_{t_1}^{t_2} \int\limits_{A} 
\frac{\partial \mathcal{L}}{\partial \psi} \cdot \delta \psi
-\frac{\partial}{\partial t} \bigg( \frac{\partial \mathcal{L}}{\partial \frac{\partial \psi}{\partial t}} \bigg) \cdot \delta \psi
-\bigg(\frac{\partial}{\partial x} \cdot \bigg( \frac{\partial \mathcal{L}}{\partial \frac{\partial \psi}{\partial x}} \bigg)\bigg) \cdot \delta \psi
\dd x^3 \dd t
\end{equation}

If we use the same index conventions for the field and space as we did above the last two terms mean

\begin{equation}
\frac{\partial}{\partial t} \bigg( \frac{\partial \mathcal{L}}{\partial \frac{\partial \psi}{\partial t}} \bigg) \cdot \delta \psi
= \sum_j \frac{\partial}{\partial t} \bigg( \frac{\partial \mathcal{L}}{\partial \frac{\partial \psi_j}{\partial t}} \bigg) \; \delta \psi_j
\end{equation}

\begin{equation}
\bigg(\frac{\partial}{\partial x} \cdot \bigg( \frac{\partial \mathcal{L}}{\partial \frac{\partial \psi}{\partial x}} \bigg)\bigg) \cdot \delta \psi
= \sum_j \bigg(\sum_i \frac{\partial}{\partial x_i} \; \bigg( \frac{\partial \mathcal{L}}{\partial \frac{\partial \psi_j}{\partial x_i}} \bigg)\bigg) \; \delta \psi_j
\end{equation}

where the sum over $i$ is called the divergence of $\partial \mathcal{L} / \partial \frac{\partial \psi_j}{\partial x}$.

The last rewrite of $\delta S$ we wish to do is 

\begin{equation}
\delta S = \int\limits_{t_1}^{t_2} \int\limits_{A} 
\bigg(
\frac{\partial \mathcal{L}}{\partial \psi}
-\frac{\partial}{\partial t} \bigg( \frac{\partial \mathcal{L}}{\partial \frac{\partial \psi}{\partial t}} \bigg) 
-\frac{\partial}{\partial x} \cdot \bigg( \frac{\partial \mathcal{L}}{\partial \frac{\partial \psi}{\partial x}} \bigg)\bigg) \cdot \delta \psi
\dd x^3 \dd t
\end{equation}

Since $\delta \psi$ is arbitrary (except from its border conditions) the only way to make $S$ stationary (which is equivalent to require $\delta S = 0$) is that $\mathcal{L}$ fulfills the condition


\begin{equation}
0 = \frac{\partial \mathcal{L}}{\partial \psi}
-\frac{\partial}{\partial t} \bigg( \frac{\partial \mathcal{L}}{\partial \frac{\partial \psi}{\partial t}} \bigg) 
-\frac{\partial}{\partial x} \cdot \bigg( \frac{\partial \mathcal{L}}{\partial \frac{\partial \psi}{\partial x}} \bigg) 
\end{equation}

This is the Euler-Lagrange equation for field theory. It can be seen as a counterpart of the Euler-Lagrange equation for particles we discussed in \cite{WagnerGuthrie}. The procedure of looking for a condition to make $S$ stationary under a Lagrangian $\mathcal{L}\bigg(\psi, \frac{\partial \psi}{\partial t}, \frac{\partial \psi}{\partial x}\bigg)$ is called the Lagrangian formalism for field theory.

\section{Invariance of the Euler-Lagrange equation for field theory under transformations}

Let $x=f(\bar{x},t)$ be an invertible and differentiable transformation of the spacial coordinates and $\psi=F(\bar{\psi})$ be an invertible and differentiable transformation of the field. We define the transformed Lagrangian  $\bar{\mathcal{L}}$ by

\begin{equation} \label{LagrTransform}
\bar{\mathcal{L}}\bigg(\bar{\psi}, \frac{\partial \bar{\psi}}{\partial t}, \frac{\partial \bar{\psi}}{\partial \bar{x}}\bigg) 
:= \mathcal{L}\bigg(F(\bar{\psi}), \frac{\partial F(\bar{\psi})}{\partial t}, \frac{\partial F(\bar{\psi})}{\partial f}\bigg) 
\bigg| det \frac{\partial f}{\partial \bar{x}} \bigg|
\end{equation}

where $\bigg| det \frac{\partial f}{\partial \bar{x}} \bigg|$ is the absolute value of the determinant of the Jacobian matrix of $f$ with respect to the spacial coordinates $\bar{x}$. \\

We will prove that from requiring S to be stationary the two equations

\begin{equation} \label{ELGTransformed}
0 = \frac{\partial \bar{\mathcal{L}}}{\partial \bar{\psi}}
-\frac{\partial}{\partial t} \bigg( \frac{\partial \mathcal{\bar{L}}}{\partial \frac{\partial \bar{\psi}}{\partial t}} \bigg) 
-\frac{\partial}{\partial \bar{x}} \cdot \bigg( \frac{\partial \mathcal{\bar{L}}}{\partial \frac{\partial \bar{\psi}}{\partial \bar{x}}} \bigg) 
\end{equation}

and

\begin{equation} \label{ELGUntransformed}
0 = \frac{\partial \mathcal{L}}{\partial \psi}
-\frac{\partial}{\partial t} \bigg( \frac{\partial \mathcal{L}}{\partial \frac{\partial \psi}{\partial t}} \bigg) 
-\frac{\partial}{\partial x} \cdot \bigg( \frac{\partial \mathcal{L}}{\partial \frac{\partial \psi}{\partial x}} \bigg) 
\end{equation}

follow and thus that the Euler-Lagrange equation for field theory is independent of arbitrary coordinate and field transformations as long as the transformation of the Lagrangian is given by \ref{LagrTransform}. \\

To do so we consider arbitrary but small variations $\delta \bar{\psi}$ of the field $\bar{\psi}$ that vanish at $t_1$, $t_2$ and on the surface of an area of space $\bar{A}$. These we use to find the condition for  

\begin{equation}
S = \int\limits_{t_1}^{t_2} \int\limits_{\bar{A}} \bar{\mathcal{L}}\bigg(\bar{\psi}, \frac{\partial \bar{\psi}}{\partial t}, \frac{\partial \bar{\psi}}{\partial \bar{x}}\bigg) \dd \bar{x}^3 \dd t 
= \int\limits_{t_1}^{t_2} \int\limits_{\bar{A}} \mathcal{L}\bigg(F(\bar{\psi}), \frac{\partial F(\bar{\psi})}{\partial t}, \frac{\partial F(\bar{\psi})}{\partial f}\bigg) 
\bigg| det \frac{\partial f}{\partial \bar{x}} \bigg| \dd \bar{x}^3 \dd t 
\end{equation}

to become stationary.\\

\ref{ELGTransformed} just follows from repeating the considerations of section \ref{definition}. \\

To prove \ref{ELGUntransformed} we look at 

\begin{equation}
S = \int\limits_{t_1}^{t_2} \int\limits_{\bar{A}} \mathcal{L}\bigg(F(\bar{\psi}), \frac{\partial F(\bar{\psi})}{\partial t}, \frac{\partial F(\bar{\psi})}{\partial f}\bigg) 
\bigg| det \frac{\partial f}{\partial \bar{x}} \bigg| \dd \bar{x}^3 \dd t 
\end{equation}

which by using the transformation formula of multidimensional integrals (see Appendix \ref{TranformationFormula}) can be turned into

\begin{equation}
S = \int\limits_{t_1}^{t_2} \int\limits_{f(\bar{A})} \mathcal{L}\bigg(F(\bar{\psi}), \frac{\partial F(\bar{\psi})}{\partial t}, \frac{\partial F(\bar{\psi})}{\partial f}\bigg) 
\dd f^3 \dd t 
\end{equation}
where $f(\bar{A})$ is the picture of $\bar{A}$ under the coordinate transformation $f$.

Based on this formula the variation $\delta S$ of $S$ is given by

\begin{equation}
\delta S = \int\limits_{t_1}^{t_2} \int\limits_{f(\bar{A})} 
\frac{\partial \mathcal{L}}{\partial F} \cdot \delta F
+ \frac{\partial \mathcal{L}}{\partial \frac{\partial F}{\partial t}} \cdot \delta \bigg(\frac{\partial F}{\partial t}\bigg)
+ \frac{\partial \mathcal{L}}{\partial \frac{\partial F}{\partial f}} \cdot \delta \bigg(\frac{\partial F} {\partial f}\bigg)
\dd f^3 \dd t
\end{equation}

where 

\begin{equation} \label{deltaFDefinition}
\delta F = \frac{\partial F}{\partial \bar{\psi}} \delta \bar{\psi}
\end{equation}


Integration by parts of the second and the third term leads to

\begin{equation} \label{calcDeltaSSection3}
\begin{split}
\delta S = \int\limits_{t_1}^{t_2} \int\limits_{f(\bar{A})} 
\frac{\partial \mathcal{L}}{\partial F} \cdot \delta F
-\frac{\partial}{\partial t} \bigg( \frac{\partial \mathcal{L}}{\partial \frac{\partial F}{\partial t}} \bigg) \cdot \delta F
-\bigg(\frac{\partial}{\partial f} \cdot \bigg( \frac{\partial \mathcal{L}}{\partial \frac{\partial F}{\partial f}} \bigg)\bigg) \cdot \delta F
\dd x^3 \dd t \\
+ \int\limits_{f(\bar{A})} \int\limits_{t_1}^{t_2} \frac{\partial}{\partial t} \bigg(\frac{\partial \mathcal{L}}{\partial \frac{\partial F}{\partial t}} \cdot \delta F \bigg) \dd t \dd f^3
+ \int\limits_{t_1}^{t_2} 
\int\limits_{f(\bar{A})} \frac{\partial}{\partial x} \cdot \bigg( \frac{\partial \mathcal{L}}{\partial \frac{\partial F}{\partial f}} \cdot \delta F \bigg) \dd x^3 \dd t
\end{split}
\end{equation}

where the identities $\delta \bigg(\frac{\partial F} {\partial t}\bigg) 
= \frac{\partial F_2} {\partial t} - \frac{\partial F_1} {\partial t}
= \frac{\partial (F_2 - F_1)} {\partial t}
= \frac{\partial \delta F} {\partial t}$ 
and
$\delta \bigg(\frac{\partial F} {\partial f}\bigg) 
= \frac{\partial F_2} {\partial f} - \frac{\partial F_1} {\partial f}
= \frac{\partial (F_2 - F_1)} {\partial f}
= \frac{\partial \delta F} {\partial f}$ 
were used. \\

Because of \ref{deltaFDefinition}
$\delta F$ vanishes at $t_1$, $t_2$ as $\delta \bar{\psi}$ does. So the second term is zero because of the fundamental theorem of calculus.

The third term can be transformed into an integral over the surface of $f(\bar{A})$ which we denote by $\partial (f(\bar{A}))$. This surface is the same as the picture of the surface of $\bar{A}$ under $f$: 
\begin{equation}
\partial (f(\bar{A})) = f(\partial \bar{A})
\end{equation}

To show that the third term vanishes, we will prove, that $\delta F$ is zero for any $x \in \partial (f(\bar{A}))$:
\\ 
\\

\noindent \textbf{Begin proof}
\\
Let $x$ be an element of $\partial (f(\bar{A}))$. Then for $x$ there exists an unique $\bar{x} \in \partial \bar{A}$ which is defined by $x=f(\bar{x})$. We are going to use the fact from above that $\delta \bar{\psi}(\bar{x}) = 0$. 
We recall that the variation $\delta \bar{\psi}$ is a difference between two fields which we name $\bar{\psi}_1$ and $\bar{\psi}_2$ such that
\begin{equation}
\delta \bar{\psi} = \bar{\psi}_2 - \bar{\psi}_1
\end{equation}
The value of $F$ considered as a function of $x$ is given by 
\begin{equation}
F(x) = F(\bar{\psi}(\bar{x})) \;\; \text{with} \;\; \bar{x} \;\; \text{defined through} \;\; x=f(\bar{x}) 
\iff \bar{x} = f^{-1}(x)
\end{equation}
The variation $\delta F$ that results from the difference $\delta \bar{\psi}$ between $\bar{\psi}_1$ and $\bar{\psi}_2$ is given by
\begin{equation}
\delta F(x) = F(\bar{\psi}_2(\bar{x})) - F(\bar{\psi}_1(\bar{x})) 
= F(\bar{\psi}_1(\bar{x}) + \delta \bar{\psi} (\bar{x})) - F(\bar{\psi}_1(\bar{x})) 
= \frac{\partial F}{\partial \bar{\psi}} \delta \bar{\psi} (\bar{x})
\end{equation}
Since $\delta \bar{\psi}(\bar{x})$ is zero by assumption, $\delta F(x)$ is zero, too, which finishes the proof.
\\
\textbf{End proof}
\\
\\

\noindent As to $\delta S$ we are now left with


\begin{equation}
\delta S = \int\limits_{t_1}^{t_2} \int\limits_{f(\bar{A})} 
\bigg(
\frac{\partial \mathcal{L}}{\partial F}
-\frac{\partial}{\partial t} \bigg( \frac{\partial \mathcal{L}}{\partial \frac{\partial F}{\partial t}} \bigg) 
-\frac{\partial}{\partial f} \cdot \bigg( \frac{\partial \mathcal{L}}{\partial \frac{\partial F}{\partial f}} \bigg)\bigg) \cdot \delta F
\dd f^3 \dd t
\end{equation}

Because of \ref{deltaFDefinition} $\delta F$ is equally arbitrary as $\delta \bar{\psi}$. Thus the only way for $\delta S$ to become zero is

\begin{equation}
0 = 
\frac{\partial \mathcal{L}}{\partial F}
-\frac{\partial}{\partial t} \bigg( \frac{\partial \mathcal{L}}{\partial \frac{\partial F}{\partial t}} \bigg) 
-\frac{\partial}{\partial f} \cdot \bigg( \frac{\partial \mathcal{L}}{\partial \frac{\partial F}{\partial f}} \bigg)
\end{equation}

If we now replace $F$ and $f$ according to their definitions by $\psi$ and $x$ this equation turns into \ref{ELGUntransformed} and thus finishes the proof.




\section{Application to physic: The Lagrangian for Electrodynamics}
The transformation properties for the Euler-Lagrange equations for field theories that we found on a purely mathematical basis makes the Euler-Lagrange formalism for field theories desirable for physics. To make the formalism available for physics we needed to turn the physical field equations into Euler-Lagrange equations. First and foremost this means to find a Lagrangian for the physical field theory in question. This has in fact been done for any existing physical field theory, e.g. Electrodynamics, General relativity, Schr\"odinger's equation, Dirac's equation, the Klein-Gordon equation and the Standard model of particle physics.

As an example we are going to define the Lagrangian of Electrodynamics and show that Maxwell's equations can be derived from its Euler-Lagrange equations. We start with some remarks on Maxwell's equations that can be found in several text books, for example \cite{Jackson}, or at \cite{Stackexchange}.

Maxwell's equations in empty space are 


\begin{equation} \label{Faraday}
\nabla \times E = - \frac{\partial B}{\partial t}
\end{equation}

\begin{equation}
\nabla \times B = \mu_0 j +  \frac{1}{c^2} \frac{\partial E}{\partial t}
\end{equation}

\begin{equation}
\nabla \cdot E = \frac{\rho}{\epsilon_0}
\end{equation}

\begin{equation} \label{noMonopole}
\nabla \cdot B = 0
\end{equation}

where

\begin{equation}
\mu_0 = \frac{1}{\epsilon_0 c^2}
\end{equation}

Because of $\nabla \cdot B = 0$ there exists a vector potential $A$ such that

\begin{equation} \label{BbyA}
\boxed{B = \nabla \times A}
\end{equation}

With that \ref{Faraday} can be turned into

\begin{equation}
\nabla \times \bigg( E + \frac{\partial A}{\partial t} \bigg) = 0
\end{equation}

When the rotation of a field is zero the field can be expressed by the gradient of a scalar potential $\phi$. Thus we can write

\begin{equation} \label{EbyAPhi}
E + \frac{\partial A}{\partial t} = - \nabla \phi
\iff
\boxed{E = - \nabla \phi - \frac{\partial A}{\partial t}}
\end{equation}

To find \ref{BbyA} and \ref{EbyAPhi} we consumed \ref{noMonopole} and \ref{Faraday}. We now use \ref{BbyA} and \ref{EbyAPhi} to express the remaining two Maxwell equations by $A$ and $\phi$ only:

\begin{equation} \label{MaxwellAPhi1}
\nabla \times ( \nabla \times A) = \mu_0 j + \frac{1}{c^2} \frac{\partial}{\partial t} \bigg(- \nabla \phi - \frac{\partial A}{\partial t} \bigg)
\end{equation}

\begin{equation} \label{MaxwellAPhi2}
\nabla \cdot \bigg( -\nabla \phi - \frac{\partial A}{\partial t} \bigg) = \frac{\rho}{\epsilon_0} 
\end{equation}
\\

Assertion: These equations are the Euler-Lagrange equations of the Lagrangian

\begin{equation}
\mathcal{L} = \epsilon_0 \frac{E^2 - c^2 B^2}{2} - \rho\phi + j \cdot A
= \epsilon_0 \frac{(-\nabla\phi - \frac{\partial A}{\partial t})^2 - c^2 (\nabla \times A)^2}{2} - \rho\phi + j \cdot A
\end{equation}
\\

To prove this we first calculate the Euler-Lagrange equation for $\phi$

\begin{equation} \label{ELPhi}
0 = \frac{\partial \mathcal{L}}{\partial \phi}
-\frac{\partial}{\partial t} \bigg( \frac{\partial \mathcal{L}}{\partial \frac{\partial \phi}{\partial t}} \bigg) 
-\frac{\partial}{\partial x} \cdot \bigg( \frac{\partial \mathcal{L}}{\partial \frac{\partial \phi}{\partial x}} \bigg) 
\end{equation}

We calculate the terms separately

\begin{equation}
\frac{\partial \mathcal{L}}{\partial \phi} = -\rho
\end{equation}

\begin{equation}
\frac{\partial \mathcal{L}}{\partial \frac{\partial \phi}{\partial t}} = 0 
\implies \frac{\partial}{\partial t} \bigg( \frac{\partial \mathcal{L}}{\partial \frac{\partial \phi}{\partial t}}\bigg) = 0 
\end{equation}



\begin{equation}
\frac{\partial \mathcal{L}}{\partial \frac{\partial \phi}{\partial x}} 
= \frac{\epsilon_0}{2} \cdot 2 \bigg(-\nabla \phi - \frac{\partial A}{\partial t} \bigg) (-1) 
= \epsilon_0 \bigg(\nabla \phi + \frac{\partial A}{\partial t} \bigg)
\end{equation}

\begin{equation}
\frac{\partial}{\partial x} \cdot \bigg(\frac{\partial \mathcal{L}}{\partial \frac{\partial \phi}{\partial x}} \bigg) 
= \nabla \cdot \bigg[\epsilon_0 \bigg(\nabla \phi + \frac{\partial A}{\partial t} \bigg)\bigg]
\end{equation}

Plugin these results into \ref{ELPhi} results in

\begin{equation}
0 = -\rho - 0 -\nabla \cdot \bigg[\epsilon_0 \bigg(\nabla \phi + \frac{\partial A}{\partial t} \bigg)\bigg]
\end{equation}

which is equivalent to \ref{MaxwellAPhi2}. With that the first part of the proof is done. \\

For the second part we have to prove that 

\begin{equation} \label{ELA}
0 = \frac{\partial \mathcal{L}}{\partial A}
-\frac{\partial}{\partial t} \bigg( \frac{\partial \mathcal{L}}{\partial \frac{\partial A}{\partial t}} \bigg) 
-\frac{\partial}{\partial x} \cdot \bigg( \frac{\partial \mathcal{L}}{\partial \frac{\partial A}{\partial x}} \bigg) 
\end{equation}

is equivalent to \ref{MaxwellAPhi1}. We do this for the first component $A_1$ of $A$ only. (Note that \ref{ELA} actually represents one equation for each component of $A$.) Again we calculate the terms separately:

\begin{equation}
\frac{\partial \mathcal{L}}{\partial A_1} = j_1
\end{equation}

\begin{equation}
\frac{\partial \mathcal{L}}{\partial \frac{\partial A_1}{\partial t} } 
= \frac{\epsilon_0}{2} \cdot 2 \bigg(-\frac{\partial \phi}{\partial x_1} - \frac{\partial A_1}{\partial t}\bigg) (-1)
\end{equation}

\begin{equation}
\frac{\partial}{\partial t} \bigg(\frac{\partial \mathcal{L}}{\partial \frac{\partial A_1}{\partial t} } \bigg)
= \epsilon_0 \frac{\partial}{\partial t} \bigg(\frac{\partial \phi}{\partial x_1} + \frac{\partial A_1}{\partial t}\bigg)
\end{equation}


\begin{equation}
\begin{split}
\frac{\partial \mathcal{L}}{\partial \frac{\partial A_1}{\partial x} }
= - \frac{\epsilon_0 c^2}{2} \frac{\partial}{\partial \frac{\partial A_1}{\partial x}}
\bigg[
\bigg(\frac{\partial A_3}{\partial x_2} - \frac{\partial A_2}{\partial x_3} \bigg)^2
+ \bigg(\frac{\partial A_1}{\partial x_3} - \frac{\partial A_3}{\partial x_1} \bigg)^2
+ \bigg(\frac{\partial A_2}{\partial x_1} - \frac{\partial A_1}{\partial x_2} \bigg)^2
\bigg]
\\
= 
- \frac{\epsilon_0 c^2}{2} 
\left(
\begin{array}{c} 
0
\\
2 \bigg(\frac{\partial A_2}{\partial x_1} - \frac{\partial A_1}{\partial x_2} \bigg) (-1)
\\
2 \bigg(\frac{\partial A_1}{\partial x_3} - \frac{\partial A_3}{\partial x_1} \bigg)
\end{array}
\right)
= 
- \epsilon_0 c^2
\left(
\begin{array}{c} 
0
\\
\bigg(\frac{\partial A_1}{\partial x_2} - \frac{\partial A_2}{\partial x_1} \bigg)
\\
\bigg(\frac{\partial A_1}{\partial x_3} - \frac{\partial A_3}{\partial x_1} \bigg)
\end{array}
\right)
\end{split}
\end{equation}

\begin{equation}
\frac{\partial}{\partial x} \cdot \frac{\partial \mathcal{L}}{\partial \frac{\partial A_1}{\partial x} }
= - \epsilon_0 c^2 
\bigg( \frac{\partial^2 A_1}{\partial x_2^2} - \frac{\partial^2 A_2}{\partial x_2 \partial x_1}
+ \frac{\partial^2 A_1}{\partial x_3^2} - \frac{\partial^2 A_3}{\partial x_3 \partial x_1} \bigg)
\end{equation}

Plugin these into \ref{ELA} results in

\begin{equation}
0= j_1 
- \epsilon_0 \frac{\partial}{\partial t} \bigg(\frac{\partial \phi}{\partial x_1} + \frac{\partial A_1}{\partial t} \bigg)
+ \epsilon_0 c^2 
\bigg( \frac{\partial^2 A_1}{\partial x_2^2} - \frac{\partial^2 A_2}{\partial x_2 \partial x_1}
+ \frac{\partial^2 A_1}{\partial x_3^2} - \frac{\partial^2 A_3}{\partial x_3 \partial x_1} \bigg)
\end{equation}

using $\mu_0 = \frac{1}{\epsilon_0 c^2}$ this can be turned into

\begin{equation}
 - \frac{\partial^2 A_1}{\partial x_2^2} + \frac{\partial^2 A_2}{\partial x_2 \partial x_1}
- \frac{\partial^2 A_1}{\partial x_3^2} + \frac{\partial^2 A_3}{\partial x_3 \partial x_1}
= \mu_0 j_1 
+ \frac{1}{c^2} \frac{\partial}{\partial t} \bigg(-\frac{\partial \phi}{\partial x_1} - \frac{\partial A_1}{\partial t} \bigg)
\end{equation}

The right hand side of this equation equals the first component of the right hand side of \ref{MaxwellAPhi1}. Now we need to prove that the first component of $\nabla \times ( \nabla \times A)$ equals the left hand side of this equation. To do so we use the formula 
$\nabla \times ( \nabla \times A) = \nabla(\nabla \cdot A) - \Delta A$:

\begin{equation}
\begin{split}
[\nabla(\nabla \cdot A) - \Delta A]_1 = 
\frac{\partial}{\partial x_1} \bigg( \frac{\partial A_1}{\partial x_1} + \frac{\partial A_2}{\partial x_2} + \frac{\partial A_3}{\partial x_3} \bigg)
- \bigg( \frac{\partial^2 A_1}{\partial x_1^2} + \frac{\partial^2 A_1}{\partial x_2^2} + \frac{\partial^2 A_1}{\partial x_3^2} \bigg) 
\\
= 
\frac{\partial^2 A_2}{\partial x_1 \partial x_2} + \frac{\partial^2 A_3}{\partial x_1 \partial x_3}
- \frac{\partial^2 A_1}{\partial x_2^2} - \frac{\partial^2 A_1}{\partial x_3^2} 
\end{split}
\end{equation}

So we found that the Euler-Lagrange equation for $A_1$ is equivalent to the first component of equation \ref{MaxwellAPhi1}. To finish the prove the last calculation just has to be repeated for the remaining Euler-Lagrange equations and components of \ref{MaxwellAPhi1}.

\section{Conclusion}
The Euler-Lagrange formalism for field theory was presented as a purely mathematical framework that provides us with field equations which are invariant under any coordinate and field transformation as long as the associated Lagrangian $\mathcal{L}$ has  the very simple and well defined transformation property given by \ref{LagrTransform}.

Based on this mathematical result we are well motivated to reformulate physical field equations in such a way that they become Euler-Lagrange equations. The crucial step to do this is to find the Lagrangian for the field theory that we want to reformulate.

\section{Appendix: Alternative to the transformation formula of multidimensional integrals} \label{TranformationFormula}

The proof of the transformation formula of multidimensional integrals is not trivial. 
A nice heuristic explanation is given by \cite{Sterman}. There

\begin{equation}
\bigg|det \frac{\partial f}{\partial \bar{x}} \bigg| =  \bigg|det \frac{\partial x}{\partial \bar{x}} \bigg|
\end{equation}

is written as $\dd x^3 / \dd \bar{x}^3$ and is explained to be the ratio of the differentials in the transformed and untransformed coordinates.


\begin{thebibliography}{9}

\bibitem{WagnerGuthrie} Gerd Wagner and Matt Guthrie, Demystifying the Lagrangian of Classical Mechanics

\bibitem{Goldstein} Goldstein, Poole and Safko, Classical Mechanics, Chapter 13 

\bibitem{Jackson} J.D. Jackson, Classical Electrodynamics, section 6.4

\bibitem{Stackexchange} \url{https://physics.stackexchange.com/questions/34241/deriving-lagrangian-density-for-electromagnetic-field}

\bibitem{Sterman} George Sterman - An Introduction to Quantum Field Theory - Section 1.3 Invariance and conservation


\end{thebibliography}



\end{document}
